\documentclass[]{article}

% PACKAGES

% correct encoding and typography
\usepackage[english]{babel}
\usepackage[utf8]{inputenc}
\usepackage[T1]{fontenc}

% math
\usepackage{amsfonts, latexsym, mathtools, amsthm, amssymb, amscd}
\usepackage{euscript}
\usepackage{esvect}
\usepackage{bm}
\usepackage{physics}

% aesthetics color options
\usepackage{graphicx}
\usepackage[dvipsnames]{xcolor}                           

% correct encoding and typography
\usepackage[english]{babel}
\usepackage[utf8]{inputenc}
\usepackage[T1]{fontenc}

\usepackage{tikz}
\usetikzlibrary{arrows.meta,positioning,calc, matrix}
\usepackage{pgfplots}
\pgfplotsset{compat=1.18}
\usepgfplotslibrary{groupplots}
\usepackage{pgfplotstable}


% math
\usepackage{amsfonts, latexsym, mathtools, amsthm, amssymb, amscd}
\usepackage{euscript}
\usepackage{esvect}
\usepackage{bm}
\usepackage{physics}

% aesthetics color options
\usepackage{graphicx}
\usepackage{pdfpages}
\usepackage{microtype}

% to access the last page
\usepackage{lastpage}

% better document formatting
\usepackage{fancyhdr}
\usepackage[a4paper, right=1.2cm, left=1.2cm, bottom=2cm, top=2cm, centering]{geometry}
\usepackage{fullpage}

\usepackage{datetime2}

% better stylistic formatting
\usepackage{parskip}
\usepackage{enumitem}
\usepackage{framed}
\usepackage{longtable}
\usepackage{csquotes}

% embedding and referencing
\usepackage{hyperref}
\hypersetup{colorlinks=true,citecolor=blue,urlcolor =black,linkbordercolor={1 0 0}}

\usepackage{booktabs,siunitx, array,longtable}


%%%%%%%%%%%%%%%%%%%%%%%%%%%%%%%%%%%%%%%%%%%%%%%%%%%%%%%%%

%%%%%%%%%%%%%%%%%%%%%%%%%%%%%%%%%%%%%%%%%%%%%%%%%%%%%%%%%
% Command formatting for ease-of-life

% new math symbols taking no arguments
\newcommand{\minus}{\smallsetminus}
\newcommand{\dotp}{\boldsymbol{\cdot}}

%redefined math symbols taking no arguments
\newcommand{\<}{\langle}
\renewcommand{\>}{\rangle}
\renewcommand{\iff}{\Leftrightarrow}
\renewcommand{\implies}{\Rightarrow}

% Arrows
\newcommand{\from}{\leftarrow}
\newcommand{\ffrom}{\longleftarrow}
\newcommand{\goesto}{\rightsquigarrow}
\newcommand{\into}{\hookrightarrow}

% Basic bbface (blackboard bold) lettering command:
\newcommand{\bbb}[1]{\ensuremath{\mathbb{#1}}}

% Boldface lettering command used for vectors:
\newcommand{\bvec}[1]{\ensuremath{\mathbf{#1}}}

\newcommand{\bN}{\bbb{N}}
\newcommand{\bR}{\bbb{R}}
\newcommand{\bZ}{\bbb{Z}}
\newcommand{\bH}{\bbb{H}}
\newcommand{\oo}{\ensuremath{\emptyset}}
\newcommand{\bQ}{\bbb{Q}}
\newcommand{\bC}{\bbb{C}}
\newcommand{\bF}{\bbb{F}}
\newcommand{\bI}{\bbb{I}}
\newcommand{\bP}{\bbb{P}}
\newcommand{\bE}{\bbb{E}}
\newcommand{\bT}{\bbb{T}}


\newcommand{\vx}{\bvec{x}}
\newcommand{\vy}{\bvec{y}}
\newcommand{\vuu}{\bvec{u}}
\newcommand{\vvv}{\bvec{v}}
\newcommand{\vw}{\bvec{w}}
\newcommand{\vaa}{\bvec{a}}
\newcommand{\vbb}{\bvec{b}}
\newcommand{\vo}{\bvec{o}}
\newcommand{\ve}{\bvec{e}}
\newcommand{\vp}{\bvec{p}}
\newcommand{\vi}{\bvec{i}}
\newcommand{\vz}{\bvec{z}}
\newcommand{\vt}{\bvec{t}}


% Basic cal lettering
\newcommand{\ccc}[1]{\ensuremath{\mathcal{#1}}}

\newcommand{\cE}{\ccc{E}}
\newcommand{\cM}{\ccc{M}}
\newcommand{\cP}{\ccc{P}}
\newcommand{\cV}{\ccc{V}}
\newcommand{\cO}{\ccc{O}}


% Basic frak lettering
\newcommand{\fff}[1]{\ensuremath{\mathfrak{#1}}}

\newcommand{\mm}{\mathfrak{m}}
\newcommand{\pp}{\mathfrak{p}}

% Other lettering
\newcommand{\GL}{\mathit{GL}}
\newcommand{\FL}{{\mathcal{} F}{\ell}_n}
\newcommand{\SO}{\mathit{SO}}
\newcommand{\eps}{\varepsilon}
\newcommand{\ind}{\mathbf{1}}

% some operators and delimiters
\DeclareMathOperator{\proj}{proj}
\DeclareMathOperator{\im}{im}
\DeclareMathOperator{\coker}{coker}
\DeclareMathOperator{\coim}{coim}
\DeclareMathOperator{\Span}{Span}
\DeclareMathOperator{\Stab}{Stab}
\DeclareMathOperator{\Orb}{Orb}
\DeclareMathOperator{\EE}{E}
\DeclareMathOperator{\Fun}{Fun}
\DeclareMathOperator{\CSet}{Set}
\DeclareMathOperator{\Cmap}{map}
\DeclareMathOperator{\colim}{colim}
\DeclareMathOperator{\Hom}{Hom}
\DeclareMathOperator{\res}{res}
\DeclareMathOperator{\Spec}{Spec}
\DeclareMathOperator{\Int}{int}

% Other more involved shortcuts

%for overline math
\newcommand{\ol}[1]{{\overline{#1}}}

%redefined math symbols taking arguments
\renewcommand{\mod}[1]{\ (\mathrm{mod}\ #1)}

%for easy 2 x 2 matrices
\newcommand{\twobytwo}[1]{\left[\begin{array}{@{}cc@{}}#1\end{array}\right]}

%for easy column vectors of size 2
\newcommand{\tworow}[1]{\left[\begin{array}{@{}c@{}}#1\end{array}\right]}

%%%%%%%%%%%%%%%%%%%%%%%%%%%%%%%%%%%%%%%%%%%%%%%%%%%%%%%%%

%%%%%%%%%%%%%%%%%%%%%%%%%%%%%%%%%%%%%%%%%%%%%%%%%%%%%%%%%
%Below are the theorem, definition, example, lemma, etc. body types.

\newtheorem{theorem}{Theorem}[section]
\newtheorem{proposition}[theorem]{Proposition}
\newtheorem{lemma}[theorem]{Lemma}
\newtheorem{corollary}[theorem]{Corollary}
\theoremstyle{definition}
\newtheorem{definition}[theorem]{Definition}
\newtheorem{assumption}[theorem]{Assumption}
\newtheorem{remark}[theorem]{Remark}
\newtheorem{example}[theorem]{Example}


%%%%%%%%%%%%%%%%%%%%%%%%%%%%%%%%%%%%%%%%%%%%%%%%%%%%%%%%%
%Some formatting tidbits for margins, paragraphs, and removing orphans

% make widows and orphans rare
\clubpenalty =10000
\widowpenalty =10000

% formatting of paragraphs and separation
\setlength{\parindent}{0pt}
\setlength{\parskip}{5pt plus 1pt}
\setlength{\headheight}{13.6pt}

\setlength\marginparwidth{2.2in}
\setlength\marginparsep{1mm}




%%%%%%%%%%%%%%%%%%%%%%%%%%%%%%%%%%%%%%%%%%%%%%%%%%%%%%%%%

% Styles and colors
\tikzset{
	state/.style   ={draw,rounded corners,thick,align=center,
		minimum width=3.2cm,minimum height=1.15cm,fill=white},
	input/.style   ={draw,rounded corners,thick,dashed,align=center,
		minimum width=3.2cm,minimum height=1.0cm,fill=gray!10},
	derived/.style ={draw,rounded corners,thick,densely dashed,align=center,
		minimum width=3.2cm,minimum height=1.0cm,fill=gray!5},
	edgeplus/.style={-Stealth,very thick,draw=ForestGreen,shorten >=2pt,shorten <=2pt},
	edgeminus/.style={-Stealth,very thick,draw=BrickRed,shorten >=2pt,shorten <=2pt},
	% If you *do* want tiny '+'/'−' labels back on edges, uncomment:
	% lab/.style={font=\scriptsize,fill=white,inner sep=1pt}
}

%%%%%%%%%%%%%%%%%%%%%%%%%%%%%%%%%%%%%%%%%%%%%%%%%%%%%%%%%


\title{\vspace{-1.0em}A Coupled Six--State Athlete Model for Training, Sleep, Recovery, and Risk\\
	\large (Sections 1--3: Introduction, Overview, Assumptions)}
\author{}
\date{}

\begin{document}

	\maketitle
	\vspace{-2.0em}
	
	\tableofcontents
	
	\newpage
	
	\section{Introduction}
	Athletic performance emerges from the three-pronged tug-of-war between \emph{training stimulus}, \emph{recovery}, and \emph{risk}. 
	
	Our goal is to build a compact, mechanistic model that can evaluate \emph{training--rest regimes} and answer operational questions such as: When does a given microcycle peak performance? How costly is late--evening high intensity on next--day readiness? What taper length best converts prior load into performance at a target event while respecting injury risk?
	
	\paragraph{Problem framing.}
	The problem statement proposes comparing qualitatively distinct regimes (e.g., high--intensity early vs.\ late sessions, split sessions vs.\ single sessions, alternating hard/easy days, dedicated recovery or taper periods). We formalize these as exogenous, time--varying input functions for training, sleep, and context (stress, nutrition), then follow their consequences through a system of coupled ODE states. The model is designed to be \emph{interpretable}, \emph{calibratable on athlete logs}, and \emph{portable} across sports.
	
	\paragraph{Design philosophy and precedent.}
	We draw on established ideas from training--response modeling (fitness--fatigue/impulse--response), tapering, concurrent training interactions, sleep effects on performance, and load--related injury risk \cite{Banister1975,Busso2003,Mujika2003,Hickson1980,Fullagar2015,Buchheit2014,Gabbett2016,Skiba2012}.
	
	
	\section{Brief overview of our dynamic model}
	\label{sec:model-overview}
	
	\subsection*{System architecture and figure}
	Our system is organized as a pipeline with three layers:  
	\emph{(i) exogenous inputs} that the coach/athlete controls (left),  
	\emph{(ii) a six-state ODE core} capturing trainable capacity, fatigue, sleep and risk (center), and  
	\emph{(iii) a derived readiness/output} for decision-making (right). The signed wiring diagram in
	Figure~\ref{fig:overview-wiring} makes these couplings explicit: \textcolor{ForestGreen}{green} = positive effect; \textcolor{BrickRed}{red} = negative effect.
	
	\begin{figure}[h!]
		\centering
		\includegraphics[width=\linewidth]{coupling_diagram_1.png}
		\caption{Left–center–right architecture. Left: five exogenous inputs; Center: six ODE states grouped by row; Right: readiness $P(t)$. Colors denote effect signs.}
		\label{fig:overview-wiring}
	\end{figure}

	
	\subsection*{Left column: exogenous inputs (what the coach controls)}
	Each input is a bounded, time-varying control signal. We keep units flexible and normalize when needed for calibration.
	\begin{description}[leftmargin=1.7em]
		\item[\textbf{Training composition \boldmath$u_E(t),u_H(t),u_S(t)$}.] Session intensity/volume streams for endurance (e.g., Zone time or TRIMP), high‑intensity/anaerobic work (intervals/HIIT), and strength/plyometrics (e.g., tonnage or explosive contacts). These are the \emph{primary} stimuli for adaptation and the main drivers of acute fatigue and micro‑damage \cite{Banister1975,Busso2003,Hickson1980}.
		\item[\textbf{Bedtime proximity kernel \boldmath$B(t)$}.] A short memory of training near lights‑out that reduces sleep efficiency that night \cite{Fullagar2015}. Operationally, $B(t)$ will later be computed by convolving recent intensity with an exponentially decaying kernel that weights the final hours before bedtime more heavily.
		\item[\textbf{Sleep schedule \boldmath$s(t)\!\in\![0,1]$}.] An on/off indicator of sleep opportunity (night sleep and optional naps). During $s(t)=1$ the model pays down fatigue/sleep debt and accelerates tissue repair.
		\item[\textbf{Nutrition/availability \boldmath$n(t)$}.] A compact proxy for energy/protein availability and timing (e.g., carbohydrate after HIIT, protein after strength). We use it to gate remodeling and reduce damage accumulation during sleep and rest.
		\item[\textbf{Context stress \boldmath$x(t)$}.] Non‑training stressors (travel, exams, heat, life stress). This input increases central load and impairs sleep quality; it is an external “tax” on recovery \cite{Buchheit2014}.
	\end{description}

	\subsection*{Center: the six‑state ODE core (what the system does)}
	The central panel contains six dynamical states grouped by theme (rows). We postpone explicit forms until Section~4; here we state what each encodes and how it is observed.
	
	\paragraph{Top row — Trainable capacities.}
	\begin{description}[leftmargin=1.7em]
		\item[\textbf{Aerobic adaptation \boldmath$A(t)$}.] Normalized $(0\!-\!1)$ “engine” for endurance performance (e.g., \%~$\dot{V}\!O_2$ improvements, time‑to‑exhaustion). Stimulated mainly by $u_E$ and $u_H$ with diminishing returns; gated by recovery. Proxies: best‑effort curves, critical‑power modeling, heart‑rate kinetics \cite{Banister1975,Skiba2012}.
		\item[\textbf{Neuromuscular/strength adaptation \boldmath$N(t)$}.] Normalized $(0\!-\!1)$ capacity for force/power (e.g., 1RM, CMJ, sprint split). Stimulated by $u_S$ and partly by $u_H$; subject to endurance–strength interference when $u_E$ is high \cite{Hickson1980}. Also gated by recovery.
	\end{description}
	
	\paragraph{Middle row — Fatigue (two time scales).}
	\begin{description}[leftmargin=1.7em]
		\item[\textbf{Acute fatigue \boldmath$F_a(t)$}.] Fast time scale (hours–days). Rises with session load ($u_E,u_H,u_S$), clears quickly (especially during sleep).
		\item[\textbf{Chronic fatigue \boldmath$F_c(t)$}.] Slow time scale (days–weeks). Accumulates when $F_a$ is repeatedly unresolved (monotony), clears slowly with sustained good sleep and lighter training \cite{Busso2003}.
	\end{description}
	
	\paragraph{Bottom row — Sleep and tissue risk.}
	\begin{description}[leftmargin=1.7em]
		\item[\textbf{Sleep debt / quality \boldmath$S(t)$}.] Larger $S$ means worse cumulative sleep state (more debt/lower quality). Increases while awake and after heavy training; decreases during sleep with an efficiency reduced by $B(t)$ \cite{Fullagar2015}.
		\item[\textbf{Injury micro‑damage / hazard \boldmath$I(t)$}.] A continuous proxy for tissue stress/inflammation (not a discrete injury). Rises with high‑impact/HIIT/strength loading and with fatigue‑mediated poor mechanics; falls with time, sleep, and nutrition \cite{Gabbett2016}.
	\end{description}

	\subsection*{Right column: derived readiness/output (what we optimize)}
	\textbf{Readiness \boldmath$P(t)$} aggregates sport‑specific performance potential from the states above. We use $P(t)$ to compare regimes, design tapers, and schedule recovery days; detailed forms appear in Section~4.
	
	\subsection*{Coupling map (how pieces talk)}
	The directed arrows in Figure~\ref{fig:overview-wiring} implement the following sign‑rules and qualitative nonlinearities:
	\begin{enumerate}
		\item \textbf{Training stimulates capacity} (\textcolor{ForestGreen}{+}): $u_E,u_H \!\to\! A$; $u_S,u_H \!\to\! N$ with saturating gains and \emph{diminishing returns}. High $u_E$ mildly interferes with $N$ (concurrent‑training effect) \cite{Hickson1980}.
		\item \textbf{Training creates load} (\textcolor{ForestGreen}{+}): all $u_\cdot \!\to\! F_a$ and, via accumulation, $F_a\!\to\!F_c$.
		\item \textbf{Load creates micro‑damage} (\textcolor{ForestGreen}{+}): $u_\cdot$, $F_a$, and $F_c$ raise $I$ (mechanical + metabolic + poor‑mechanics channels).
		\item \textbf{Sleep repairs} (\textcolor{ForestGreen}{+} into recovery, \textcolor{BrickRed}{-} into debts): $s(t)$ \emph{reduces} $S$, $F_a$, $F_c$ and $I$. But late training worsens that repair: larger $B(t)$ \emph{reduces} the sleep‑driven clearance of $S$, $F_a$, $F_c$, $I$ \cite{Fullagar2015}.
		\item \textbf{Sleep debt throttles adaptation} (\textcolor{BrickRed}{-}): larger $S$ suppresses gains in $A,N$ and increases $F_a,F_c$ (more wakefulness/poorer sleep $\Rightarrow$ higher perceived load) \cite{Busso2003,Fullagar2015}.
		\item \textbf{Micro‑damage suppresses adaptation} (\textcolor{BrickRed}{-}): high $I$ reduces realized gains in $A,N$ and contributes to readiness penalties \cite{Gabbett2016}.
		\item \textbf{Context stress loads the system} (\textcolor{ForestGreen}{+} into $F_c,S$): travel/heat/psychological load raises central fatigue and impairs sleep quality \cite{Buchheit2014}.
		\item \textbf{Nutrition improves remodeling} (\textcolor{BrickRed}{-} into $I$): adequate energy/protein reduces tissue damage and speeds clearance.
		\item \textbf{Readiness aggregation}: $A,N$ contribute positively; $F_a,F_c,S,I$ subtract with task‑specific weights.
	\end{enumerate}
	All couplings will be implemented with smooth, saturating response functions to ensure state positivity and realistic ceilings \cite{Busso2003}.
	
	\subsection*{Regimes as inputs (how we encode the schedules)}
	We represent regimes by specifying the shapes and timing of the five input streams. Below are canonical examples we will test, using the exact left‑column elements of Figure~\ref{fig:overview-wiring}.
	
	\paragraph{1. Early‑morning HIIT (7–9 AM), no nap.}
	\emph{Encoding:} A short $u_H$ pulse near wake time; low $B(t)$; $s(t)$ is one nightly block.
	\emph{Expected signature:} Strong $N$ and $A$ stimulus; modest $F_a$ spike; minimal impact on that night’s sleep; next‑day $P$ depends on preceding night’s $S$.
	
	\paragraph{2. Evening moderate/high intensity (7–9 PM).}
	\emph{Encoding:} $u_E$ or $u_H$ pulse ending near lights-out $\Rightarrow$ large $B(t)$; standard $s(t)$.  
	\emph{Expected signature:} Reduced sleep‑efficiency that night (slower decay of $S,F_a,F_c,I$); next‑day $P$ depressed; cumulative late‑evening sessions elevate chronic load.
	
	\paragraph{3. Split session (light AM endurance + PM strength).}
	\emph{Encoding:} Small morning $u_E$ pulse; larger afternoon/evening $u_S$ pulse raising $B(t)$.  
	\emph{Expected signature:} Good $N$ gains with some interference from $u_E$; higher $I$ and $S$ on PM‑strength days; performance trade‑off between power gains and sleep.
	
	\paragraph{4. Alternating days (hard/easy microcycle).}
	\emph{Encoding:} Hard day: large $u_\cdot$ pulses; Easy day: minimal $u_\cdot$, $s(t)$ may include a nap; $B(t)$ small on easy days.  
	\emph{Expected signature:} $F_a$ rises on hard days then decays; $F_c$ stabilizes or falls; $I$ accumulates more slowly; $P$ shows saw‑tooth with higher weekly average.
	
	\paragraph{5. Midday training (1–3 PM).}
	\emph{Encoding:} $u_E$ or mixed session far from bedtime $\Rightarrow$ small $B(t)$.  
	\emph{Expected signature:} Balanced load–recovery; relatively low $S$; favorable steady‑state $P$ with low $I$ accrual.
	
	\paragraph{6. Taper into event week (volume down, intensity maintained).}
	\emph{Encoding:} Multiply $u_E,u_S$ volumes by a decaying factor; maintain short $u_H$ stimuli; enforce early‑day sessions to keep $B(t)$ small.  
	\emph{Expected signature:} $F_a\downarrow$, then $F_c\downarrow$; $S$ improves; $A,N$ maintained; $I$ decays; peak in $P$ near event \cite{Mujika2003}.
	
	\paragraph{7. Sleep‑extension and nap policy.}
	\emph{Encoding:} Increase nightly $s(t)$ duration and add a short post‑lunch nap block; enforce low‑$B(t)$ by moving $u_\cdot$ earlier.  
	\emph{Expected signature:} Faster clearance of $F_a,F_c,I$; sustained reduction in $S$; higher readiness envelope for the same weekly load \cite{Fullagar2015}.
	
	\paragraph{8. High‑volume polarized vs.\ pyramidal endurance blocks.}
	\emph{Encoding:} Shift weight among $u_E$ (easy volume) and $u_H$ (interval density) with identical weekly “TRIMP”.  
	\emph{Expected signature:} Comparable $A$ gains but different $F_a,S$ trajectories; polarized blocks target higher $P$ with lower $I$ at the same load.
	
	\medskip
	These regime encodings are \emph{inputs only}; the explicit ODEs that transform them into state trajectories will be written in Section~\ref{sec:model-development}. Our analysis will compare regimes by their steady‑state $P$ envelopes, peaks, time‑to‑peak, and risk measures (e.g., time above $I$ thresholds).
	
	
	
	\section{Assumptions}
	We separate assumptions into: (i) global modeling assumptions; (ii) regime/input assumptions; and (iii) state--specific assumptions that will directly inform Section~4 when we write the ODEs.
	
	\subsection{Global modeling assumptions}
	\begin{enumerate}[label=\textbf{G\arabic*.}]
		\item \textbf{Single ``well--mixed'' athlete:} we model one athlete as a single dynamical unit; tissue and organ micro--heterogeneity are absorbed into parameters.
		\item \textbf{Time scales:} processes evolve on hours--to--weeks; we do not include circannual or multi--year remodeling here.
		\item \textbf{Non--dimensionalization:} states (\(A,N,S,I\)) are scaled to \([0,1]\); \(F_a,F_c\) are nonnegative with practical upper bounds from data.
		\item \textbf{Regularity:} inputs \(u_E,u_H,u_S,s,n,x\) are piecewise continuous and bounded; regime switches are scheduled or threshold--triggered (Section~4).
		\item \textbf{Saturations:} all response functions are monotone and saturating (e.g., Hill/Michaelis--Menten--like) to enforce physiological ceilings and diminishing returns \cite{Busso2003,Mujika2003}.
		\item \textbf{Positivity and invariance:} the ODE right--hand sides are constructed to keep physically meaningful ranges invariant (no negative sleep debt or negative injury, etc.).
		\item \textbf{No explicit delays (first pass):} distributed training effects are approximated by multiple time scales (acute \(\to\) chronic) rather than explicit delay differential equations \cite{Busso2003}.
		\item \textbf{Observables:} we map proxies to states for calibration: critical power/\(W'\) or best efforts to \(A\), jump/1RM surrogates to \(N\), session RPE and neuromuscular decrements to \(F_a\), HRV/sleep metrics to \(R/S\), soreness/incident logs to \(I\) \cite{Buchheit2014,Skiba2012,Fullagar2015}.
		\item \textbf{Noise and shocks:} stochastic shocks (illness, travel) are represented through \(x(t)\); we neglect process noise in the first pass.
	\end{enumerate}
	
	\subsection{Regime and input assumptions}
	\begin{enumerate}[label=\textbf{R\arabic*.}]
		\item \textbf{Training decomposition:} total load is \(u(t)=u_E(t)+u_H(t)+u_S(t)\); each component differs in \emph{how} it stimulates capacity vs.\ damage and in energy cost \cite{Banister1975,Hickson1980}.
		\item \textbf{Sleep window:} \(s(t)=1\) during scheduled sleep (including naps); nightly sleep efficiency is reduced by a \emph{bedtime--proximity} kernel \(B(t)\) that integrates training intensity close to bedtime.
		\item \textbf{Nutrition simplification:} \(n(t)\) represents energy/protein availability; we will later let \(n(t)\) gate recovery and reduce damage accrual.
		\item \textbf{Context stress:} \(x(t)\) aggregates non--training stressors; it increases fatigue and sleep debt and (weakly) raises micro--damage (e.g., travel).
		\item \textbf{Hybrid switching (optional):} regimes are either prescribed on a calendar or triggered by internal thresholds, e.g., if a hazard score from \(F_a,F_c,S,I\) exceeds a limit, switch to recovery.
	\end{enumerate}
	
	\subsection{State--specific assumptions (to guide the ODE forms later)}
	\paragraph{Aerobic adaptation \(A(t)\).}
	\begin{enumerate}[label=\textbf{A\arabic*.}]
		\item Stimulated primarily by \(u_E\) and \(u_H\); the effect is saturating and subject to diminishing returns.
		\item Gains are \emph{gated} by recovery: high \(S\) (poor sleep) and high \(F_c\) reduce effective adaptation \cite{Busso2003,Fullagar2015}.
		\item Detrains slowly toward a baseline in the absence of stimulus.
		\item Elevated damage \(I\) suppresses realized gains (e.g., protective downregulation) \emph{and} can transiently impede training quality.
	\end{enumerate}
	
	\paragraph{Neuromuscular adaptation \(N(t)\).}
	\begin{enumerate}[label=\textbf{N\arabic*.}]
		\item Stimulated by \(u_S\) and, secondarily, by \(u_H\) (shared neuromuscular stress).
		\item Endurance load \(u_E\) causes a modest \emph{interference} with strength/power gains (modeled later as a damping factor) \cite{Hickson1980}.
		\item Gains are gated by \(S\) and \(F_c\) (poor sleep/central fatigue slow synthesis and motor learning).
		\item Detrains with a time constant distinct from \(A\) (typically faster).
		\item Elevated \(I\) directly suppresses \(N\) gains (pain/inflammation limiting heavy work).
	\end{enumerate}
	
	\paragraph{Acute fatigue \(F_a(t)\).}
	\begin{enumerate}[label=\textbf{F\arabic*.}]
		\item Increases with all training components; intensity--heavy work contributes disproportionately \((u_H,u_S)\).
		\item Clears quickly with time and \emph{faster} under good sleep (\(s(t)\)) and good recovery state.
		\item Low energy/nutrition (via \(n(t)\)) and high \(S\) blunt clearance.
	\end{enumerate}
	
	\paragraph{Chronic fatigue \(F_c(t)\).}
	\begin{enumerate}[label=\textbf{C\arabic*.}]
		\item Accumulates from unresolved \(F_a\) (low--pass filtered fatigue).
		\item Clears slowly with time and sleep; sensitive to monotony/psychological factors (absorbed into \(x(t)\)).
		\item High \(F_c\) gates down \(A\) and \(N\) gains \cite{Busso2003}.
	\end{enumerate}
	
	\paragraph{Sleep debt / quality \(S(t)\).}
	\begin{enumerate}[label=\textbf{S\arabic*.}]
		\item Accumulates while awake and with strenuous training days (via arousal and thermoregulation burdens).
		\item Decreases during sleep; the nightly paydown is reduced when \(B(t)\) is large (late training) \cite{Fullagar2015}.
		\item Higher \(S\) raises \(F_a\) and \(F_c\) (worse sleep \(\Rightarrow\) more fatigue) and gates down capacity gains.
	\end{enumerate}
	
	\paragraph{Injury micro--damage \(I(t)\).}
	\begin{enumerate}[label=\textbf{I\arabic*.}]
		\item Increases with mechanical/metabolic stress, particularly \(u_S\) and high--intensity efforts \(u_H\).
		\item Accrual is amplified by high \(F_a\) or \(F_c\) (poor mechanics, compromised tissue resilience).
		\item Clears with time, sleep, and adequate nutrition \(n(t)\) (remodeling).
		\item A (soft) hazard from \(I\) contributes to regime switching/guard conditions in Section~4 and to performance penalties \cite{Gabbett2016}.
	\end{enumerate}
	
	\paragraph{Derived outputs (for later use).}
	We will define sport--specific \emph{readiness} signals \(P_{\text{end}}\) and \(P_{\text{str}}\) as functions of \(A,N,F_a,F_c,S,I\), to be used for evaluation and optimization in later sections.
	
	\section{Model development: a six--state ODE with exogenous controls}
	\label{sec:model-development}
	
	In line with the wiring diagram in Figure~\ref{fig:overview-wiring}, we now move from concepts to a concrete
	six--equation dynamical system driven by five exogenous inputs (training, bedtime proximity, sleep
	opportunity, nutrition, and context stress). The model is intentionally \emph{mechanistic but light}:
	each term has a physiological interpretation, obeys sign rules from Section~\ref{sec:model-overview},
	and uses smooth saturating (logistic/Hill) responses so states remain in meaningful ranges
	(\emph{positivity} and \emph{boundedness}). We keep the time unit as \textbf{days}; sub–day effects (e.g., a 
	late session) enter via the bedtime kernel.
	
	Throughout, we adopt the training–response perspective pioneered by Banister and co‑authors and
	expanded by many others \cite{Banister1975,Busso2003,Skiba2012,Mujika2003}, the interference
	literature for concurrent strength and endurance \cite{Hickson1980}, sleep–performance links \cite{Fullagar2015,Buchheit2014}, and
	load–injury risk ideas \cite{Gabbett2016}.
	
	\subsection{Preliminaries: inputs, normalizations, and helper functions}
	
	\paragraph{Inputs (left column of Fig.~\ref{fig:overview-wiring}).}
	\begin{itemize}
		\item \(u_E(t),u_H(t),u_S(t)\) --- endurance, HIIT/anaerobic, and strength/plyometric session intensity--volume
		streams (e.g., zone minutes/TRIMP; interval load; tonnage or contact count). They are bounded and piecewise
		continuous. We use \(\|u\|_\text{day}=\int_t^{t+1}(u_E+u_H+u_S)\,d\tau\) as the day’s gross load.
		\item \(B(t)\) --- the \textbf{bedtime proximity kernel}. It compresses “how late” training occurred into a scalar
		that modulates that night’s sleep quality. We define
		\[
		B(t)=\int_{-\infty}^t \big(\beta_E u_E(\tau)+\beta_H u_H(\tau)+\beta_S u_S(\tau)\big)\,K_b\!\big(t-\tau\big)\,d\tau,
		\]
		where \(K_b(\Delta)=\exp(-\Delta/\sigma_b)\mathbf{1}_{0<\Delta\le h_b}\) weights the last \(h_b\) hours before lights--out;
		\(\sigma_b\) is a decay constant. Physiologically: high intensity close to bedtime leaves more arousal/heat, impairing
		early sleep \cite{Fullagar2015}.
		\item \(s(t)\in[0,1]\) --- \textbf{sleep opportunity}; \(s=1\) during night sleep or planned naps, else \(0\).
		\item \(n(t)\in[0,1]\) --- \textbf{nutrition/availability}; a compact proxy for energy/protein availability and timing
		(e.g., carbohydrate after HIIT; protein after strength); higher \(n\) improves repair.
		\item \(x(t)\ge0\) --- \textbf{context stress} (travel/heat/psychological load); larger \(x\) adds central load and
		impairs sleep quality \cite{Buchheit2014}.
	\end{itemize}
	
	\paragraph{Gates and saturations.}
	We repeatedly use three small building blocks in our equations:
	
	\begin{align}
		\textit{(i) Recovery gate:}\quad & G_\mathrm{rec}(S,F_c)=\frac{1}{1+c_S S+c_c F_c}\in(0,1],                                            \label{eq:recgate}\\
		\textit{(ii) Interference gate:}\quad & G_\mathrm{int}(u_E)=\frac{1}{1+\mu_E\,u_E}\in(0,1],                                             \label{eq:intgate}\\
		\textit{(iii) Sleep efficiency:}\quad & q(B)=\frac{q_0}{1+\eta\,B(t)}\in(0,q_0], \qquad \eta\ge0, \; 0<q_0\le1.                          \label{eq:sleepeff}
	\end{align}
	
	The \emph{recovery gate} \eqref{eq:recgate} (so named to emphasize that poor sleep/central fatigue throttle supercompensation) embodies the empirically supported idea that poor sleep and
	unresolved central fatigue throttle adaptation \cite{Busso2003,Fullagar2015}. 
	
	The \emph{interference gate}
	\eqref{eq:intgate} (named afater Hickson's concurrent training effect) encodes the classic endurance–strength interference effect \cite{Hickson1980}. 
	
	The
	\emph{sleep efficiency} \eqref{eq:sleepeff} implements “late training hurts tonight’s recovery” via \(B(t)\)
	\cite{Fullagar2015}. Sleep has both a \emph{quantity} (opportunity \(s(t)\)) and \emph{quality} dimension; late, intense sessions degrade
	the latter \cite{Fullagar2015}. A short exponentially weighted memory captures “how late was the hard work”
	without a full circadian submodel. 
	
	We also use standard diminishing–returns saturations for the two other capacities.
	
	\subsection{The six ODEs (center of the figure)}
	\label{sec:six-odes}
	
	We now present the forced system \(\dot{\mathbf{z}}=f(\mathbf{z},\mathbf{u})\) with
	\(\mathbf{z}(t)=[A,N,F_a,F_c,S,I]^\top\) and inputs \(\mathbf{u}(t)=[u_E,u_H,u_S,B,s,n,x]^\top\).
	Time is measured in \emph{days}. Each right–hand side is constructed so that: (i) sign conventions match the wiring diagram in Fig.~\ref{fig:overview-wiring}; (ii) states remain nonnegative (forward‑invariance); (iii) capacities saturate and detraining is first order, consistent with decades of training‑response literature \cite{Banister1975,Busso2003,Mujika2003}; (iv) sleep opportunity and bedtime proximity modulate nightly clearance via the sleep‑efficiency \(q(B)\) \cite{Fullagar2015}; and (v) the endurance–strength interference documented by Hickson is represented parsimoniously \cite{Hickson1980}.
	
	\vspace{0.3em}
	\noindent\textbf{Notation recap and units.} We take \(A,N,S,I,F_a,F_c\ge0\). \(A,N\) are normalized to \([0,1]\) by asymptotes \(K_A,K_N\) (dimensionless). Fatigues \(F_a,F_c\) and sleep‑debt \(S\) are dimensionless load states; injury \(I\) is a dimensionless hazard/micro‑damage proxy. Training streams \(u_E,u_H,u_S\) have units “training impulse per day” (e.g., TRIMP‑like for endurance, interval‑load for HIIT, tonnage/contacts for strength); all gains that multiply them carry day\(^{-1}\) per (unit of impulse). Sleep \(s(t)\in\{0,1\}\) (night sleep and naps); \(B(t)\ge0\) is the bedtime‑proximity kernel (Sec.~4.1); \(n(t)\in[0,1]\) is a nutrition/availability proxy; \(x(t)\ge0\) is context stress (travel/heat/psych load).
	
	\paragraph{Top row: trainable capacities.}
	\begin{align}
		\dot A &= 
		\underbrace{k_A\big(\alpha_E u_E+\alpha_H u_H\big)}_{\substack{\text{endurance and HIIT stimuli} \\ \text{(units: day}^{-1}\!)}} 
		\,\underbrace{G_\mathrm{rec}(S,F_c)}_{\text{recovery gate}}\,
		\underbrace{\Big(1-\frac{A}{K_A}\Big)}_{\text{diminishing returns}}
		\;-\;\underbrace{\frac{A}{\tau_A}}_{\text{detraining}}
		\;-\;\underbrace{\theta_{AI}\,I}_{\text{injury penalty}} ,
		\label{eq:A-expanded}\\[0.25em]
		\dot N &= 
		\underbrace{k_N\big(\alpha_S u_S+\alpha_{HN} u_H\big)}_{\substack{\text{strength + HIIT stimuli}\\\text{(day}^{-1}\!)}} 
		\,G_\mathrm{rec}(S,F_c)\,
		\underbrace{G_\mathrm{int}(u_E)}_{\text{concurrent interference}}
		\Big(1-\frac{N}{K_N}\Big)
		\;-\;\frac{N}{\tau_N}
		\;-\;\theta_{NI}\,I .
		\label{eq:N-expanded}
	\end{align}
	
	\emph{Design Choices and Evidence.}
	\begin{enumerate}
		\item \emph{Stimulus terms.} Endurance \(u_E\) and HIIT \(u_H\) are the primary drivers of central/aerobic adaptation \(A\); strength \(u_S\) (and, secondarily, HIIT) drive neuromuscular adaptation \(N\). The linear dependence in the low‑to‑moderate range reflects the impulse‑response tradition \cite{Banister1975,Busso2003}; higher‑order effects are carried by the saturation \((1-A/K_A)\) and \((1-N/K_N)\).
		\item \emph{Diminishing returns.} The logistic–like factor ensures that identical stimuli produce smaller increments when close to capacity—empirically consistent with taper/peaking observations \cite{Mujika2003}.
		\item \emph{Recovery gate \(G_\mathrm{rec}(S,F_c)=1/(1+c_S S+c_c F_c)\).} Poor sleep state \(S\) and unresolved central fatigue \(F_c\) throttle supercompensation; the multiplicative gate makes this explicit and keeps units tidy (dimensionless factor \( \in (0,1]\)) \cite{Busso2003,Fullagar2015}.
		\item \emph{Detraining.} In the absence of stimulus, \(A\) and \(N\) decay exponentially with time constants \(\tau_A,\tau_N\) (20–90 days typical), matching longitudinal observations \cite{Busso2003}.
		\item \emph{Injury penalty.} A high hazard state \(I\) reduces \emph{realized} gains (pain/inflammation limiting quality and protective down‑regulation) \cite{Gabbett2016}. We model this as a linear sink; in Sec.~5 we will check sensitivity to \(\theta_{AI},\theta_{NI}\).
		\item \emph{Interference gate \(G_\mathrm{int}(u_E)=1/(1+\mu_E u_E)\).} The classic endurance‑strength interference is represented as a simple monotone damping of strength gains when concurrent endurance load is high; \(\mu_E\) sets how much endurance “soaks up” resources otherwise available to \(N\) \cite{Hickson1980}. 
	\end{enumerate}
	
	\emph{Limits.}
	(i) If \(u_E=u_H=0\), \(A\) detains to baseline with time constant \(\tau_A\); similarly for \(N\) if \(u_S=u_H=0\).  
	(ii) If sleep/recovery are excellent (\(S=F_c=0\)), the gate is \(1\) and the model reduces to a classic stimulus–response with saturation.  
	(iii) If \(I\) is large, both capacities stall; this is intentional and encourages recovery regimes in Sec.~6.
	
	\paragraph{Middle row: fatigue on two time scales.}
	\begin{align}
		\dot F_a &= 
		\underbrace{\gamma_E u_E+\gamma_H u_H+\gamma_S u_S}_{\text{training\,$\to$\,acute load}}
		\;-\;\underbrace{\big(\tfrac{1}{\tau_{fa}}+\rho_a\,s(t)\,q(B)\big)F_a}_{\substack{\text{baseline + sleep‑accelerated}\\\text{clearance (reduced by late sessions)}}},
		\label{eq:Fa-expanded}\\[0.25em]
		\dot F_c &= 
		\underbrace{\varepsilon\,F_a}_{\text{unresolved fatigue spills over}}
		\;+\;\underbrace{\xi_c\,x(t)}_{\text{context stress}}
		\;-\;\underbrace{\big(\tfrac{1}{\tau_{fc}}+\rho_c\,s(t)\,q(B)\big)F_c}_{\text{slow clearance, helped by sleep}} .
		\label{eq:Fc-expanded}
	\end{align}
	
	\emph{Design choices and evidence.}
	\begin{itemize}[leftmargin=1.6em]
		\item \emph{Additive inputs, multiplicative dissipation.} All training modalities raise acute fatigue \(F_a\). Both \(F_a\) and \(F_c\) clear with baseline first‑order dissipation (\(\tau_{fa},\tau_{fc}\)) \emph{plus} a sleep‑accelerated component, scaled by \(s(t)\,q(B)\). Making clearance proportional to the current level (\(\propto F_a,F_c\)) guarantees nonnegativity: if a state hits zero, it cannot go negative (forward‑invariance).
		\item \emph{Late‑session penalty.} Sleep efficiency \(q(B)=q_0/(1+\eta B)\) is reduced when late training makes \(B\) large; hence, clearance at night is smaller after evening intensity \cite{Fullagar2015}.
		\item \emph{From acute to chronic.} The spillover parameter \(\varepsilon\) is the “low‑pass filter” linking day‑to‑day load to weeks‑scale central fatigue; it reproduces the fast/slow components in impulse‑response models \cite{Busso2003}.
		\item \emph{Context stress \(x(t)\).} Travel, heat, exams and logistics add central load and degrade sleep \cite{Buchheit2014}; here \(x(t)\) feeds \(F_c\) via \(\xi_c\), and will also enter \(S\) below.
	\end{itemize}
	
	\emph{Limits.}
	(i) If no sleep is taken (\(s=0\)) both fatigues still clear slowly (via \(1/\tau\)), preventing blow‑up.  
	(ii) If an athlete sleeps well and early (so \(s=1\) and \(B\approx0\)), night clearance is maximal \(\approx(1/\tau+\rho)\,F\).  
	(iii) If there is no training nor stress, \(F_a,F_c\to0\) exponentially.
	
	\paragraph{Bottom row: sleep debt/quality and micro‑damage (risk).}
	\begin{align}
		\dot S &= 
		\underbrace{\lambda_w\big(1-s(t)\big)}_{\text{wake accrual}}
		\;+\;\underbrace{\lambda_T\big(\gamma_E u_E+\gamma_H u_H+\gamma_S u_S\big)}_{\text{training raises need for sleep}}
		\;+\;\underbrace{\xi_s\,x(t)}_{\text{stress impairs sleep}}
		\;-\;\underbrace{\big(\tfrac{1}{\tau_S}+\mu_s\,s(t)\,q(B)\big)S}_{\text{baseline + sleep‑accelerated paydown}},
		\label{eq:S-expanded}\\[0.25em]
		\dot I &= 
		\underbrace{\psi_0\big(\kappa_E u_E+\kappa_H u_H+\kappa_S u_S\big)}_{\text{mechanical/metabolic stress}}
		\underbrace{\big(1+\chi_a F_a+\chi_c F_c\big)}_{\text{fatigue amplifies damage}}
		\;-\;\underbrace{\big(\tfrac{1}{\tau_I}+\psi_s\,s(t)\,q(B)+\psi_n\,n(t)\big)\,I}_{\text{time, sleep, nutrition repair}} .
		\label{eq:I-expanded}
	\end{align}
	
	\emph{Design choices and evidence.}
	\begin{itemize}[leftmargin=1.6em]
		\item \emph{Sleep debt \(S\).} While awake, debt accumulates at \(\lambda_w\); hard training days add an extra term \(\lambda_T(\ldots)\) (arousal/thermoregulation burden). During sleep, debt is paid down at a rate that is larger when sleep is available (\(s=1\)) and efficient (\(q(B)\) close to \(q_0\)). Late sessions (\(B\!\uparrow\)) reduce the nightly paydown \cite{Fullagar2015}.
		\item \emph{Micro‑damage \(I\).} Tissue stress rises with modality‑weighted load \(\kappa_E,\kappa_H,\kappa_S\); fatigue multiplies the effect (\(\chi_a,\chi_c\)) because poor mechanics and slower remodeling under fatigue are well documented \cite{Gabbett2016}. Repair is first‑order with time and is accelerated by sleep (via \(q(B)\)) and by nutrition \(n(t)\) (energy/protein availability).
		\item \emph{Nutrition as clearance.} Modeling \(n(t)\) as an additive clearance coefficient \(\psi_n n(t)\) keeps dimensions clean and matches the idea that adequate fueling speeds remodeling rather than instantaneously erasing damage.
	\end{itemize}
	
	\emph{Limits.}
	(i) If an athlete extends sleep opportunity (larger \(s\)) and trains earlier (smaller \(B\)), the term \((\mu_s s q(B))S\) pulls \(S\) down faster.  
	(ii) If nutrition is chronically low (\(n\approx0\)), \(I\) clears slowly, raising injury risk for the same external load.
	
	\paragraph{Readiness aggregator (right column).}
	We combine states into a sport‑specific readiness signal
	\begin{equation}
		P(t)\;=\;w_{\mathrm{end}}A+w_{\mathrm{str}}N\;-\;\lambda_a F_a-\lambda_c F_c-\lambda_s S-\lambda_i I,
		\label{eq:P-expanded}
	\end{equation}
	where weights encode the event profile (e.g., \(w_{\mathrm{end}}\gg w_{\mathrm{str}}\) for a marathon). The sign structure matches Fig.~\ref{fig:overview-wiring}. In Sec.~5 we will calibrate \(\lambda_\cdot,w_\cdot\) so that \(P(t)\) predicts best‑effort tests; in Sec.~6 we will use time‑averaged \(P\) and its peak timing to rank regimes and to design tapers \cite{Mujika2003}.
	
	\subsubsection*{Why these forms?}
	\paragraph{Positivity and boundedness without ad hoc clipping.}
	All sinks in \eqref{eq:Fa-expanded}–\eqref{eq:I-expanded} are proportional to the current state, so the nonnegative orthant is forward‑invariant: if a state hits zero, it cannot cross negative. For capacities, the saturations \((1-A/K_A),(1-N/K_N)\) and linear detraining keep \(A\in[0,K_A]\), \(N\in[0,K_N]\) if initialized inside the interval—no artificial “max/min” operators are needed.
	
	\paragraph{Clear separation between \emph{stimulus}, \emph{modulation}, and \emph{dissipation}.}
	Each equation decomposes as: \(\text{gain from inputs}\times\text{gates/saturations}\;-\;\text{(baseline + sleep + nutrition) clearance}\).
	This mirrors the impulse–response view \cite{Banister1975,Busso2003}, isolates where sleep acts (always in the clearance terms through \(q(B)\)), and makes the late–session penalty precise.
	
	\paragraph{Bedtime proximity as a short memory instead of a full circadian submodel.}
	The kernel \(B(t)=\int (\beta_E u_E+\beta_H u_H+\beta_S u_S)\,K_b(t-\tau)\,d\tau\) (Sec.~4.1) captures “how late was the hard work?” while keeping the model small. The concavity \(q(B)=q_0/(1+\eta B)\) ensures diminishing sleep impairment as \(B\) increases (a large late session is worse than two modest sessions).
	
	\paragraph{Interference via a \emph{gate} rather than via cross‑terms in \(\dot N\).}
	We use \(G_\mathrm{int}(u_E)=1/(1+\mu_E u_E)\) to encode the endurance–strength interference \cite{Hickson1980}. This makes the trade‑off explicit, dimensionless, and easy to calibrate from blocks that vary the amount of concurrent endurance.
	
	\paragraph{Minimal parameter collinearity.}
	Stimulus gains \(k_A,k_N\) are separated from modality weights \(\alpha_{\cdot}\), the recovery gate uses distinct couplings \(c_S,c_c\), and sleep efficiency \(q(B)\) has its own baseline/penalty \((q_0,\eta)\). This reduces confounding in calibration (Sec.~5).
	
	\subsubsection*{Dimensional consistency and quick reference}
	Table~\ref{tab:dimensions} shows the units of the main multipliers so each product in \eqref{eq:A-expanded}–\eqref{eq:I-expanded} is in day\(^{-1}\) times a state (as required for an ODE in days).
	\begin{center}
		\begin{tabular}{lll}
			\hline
			Term & Role & Units \\
			\hline
			$k_A\,\alpha_E u_E$,\; $k_A\,\alpha_H u_H$;\; $k_N\,\alpha_S u_S$,\; $k_N\,\alpha_{HN} u_H$ & stimulus gains & day$^{-1}$ \\
			$1/\tau_A,1/\tau_N,1/\tau_{fa},1/\tau_{fc},1/\tau_S,1/\tau_I$ & baseline decay & day$^{-1}$ \\
			$\rho_a s q(B),\rho_c s q(B),\mu_s s q(B),\psi_s s q(B),\psi_n n$ & sleep/nutrition clearance & day$^{-1}$ \\
			$\gamma_E u_E,\gamma_H u_H,\gamma_S u_S$ & training $\to$ fatigue & day$^{-1}$ \\
			$\varepsilon$ & acute $\to$ chronic fatigue spillover & day$^{-1}$ \\
			$\psi_0(\kappa_E u_E+\kappa_H u_H+\kappa_S u_S)$ & load $\to$ damage & day$^{-1}$ \\
			$c_S S+c_c F_c$ & recovery gate (dimensionless) & -- \\
			$\mu_E u_E$ & interference gate (dimensionless) & -- \\
			\hline
		\end{tabular}
	\end{center}
	\label{tab:dimensions}
				
	\subsection{Well‑posedness and qualitative properties}
	The right–hand side of our ODE system is locally Lipschitz in the states for bounded inputs, so
	solutions exist and are unique. For \(A,N\), the saturations and linear decay give \(0\le A\le K_A\), \(0\le N\le K_N\) when initialized in range. For \(F_a,F_c,S,I\), the linear dissipation terms imply the nonnegative orthant is
	positively invariant and each state is ultimately bounded by an affine function of the input magnitudes
	(\(\|u\|_\text{day},x\)). Under stationary, \(T\)-periodic, or weekly repeating inputs we can study equilibria or the Poincaré map.
		
	\subsection{From regimes to inputs (how schedules drive the ODE)}
	Each regime from the problem statement becomes a specification of the input streams:
	timestamps and magnitudes for $(u_E,u_H,u_S)$, the sleep window $s(t)$ (and naps), and the
	lights--out time used to compute $B(t)$. Nutrition $n(t)$ and context stress $x(t)$ can be held at
	nominal profiles or varied to match scenarios (travel, heat). Representative codings:
	\begin{itemize}
		\item \textbf{Early AM HIIT:} a short $u_H$ pulse soon after wake; standard $s(t)$; small $B(t)$.
		\item \textbf{Evening intensity:} $u_E$ or $u_H$ pulse that ends near lights--out; large $B(t)$ reduces $q(B)$ that night.
		\item \textbf{Split day (AM endurance, PM strength):} two pulses with the PM $u_S$ increasing $B(t)$; useful to probe
		the $G_\mathrm{int}$ effect on $N$.
		\item \textbf{Alternating hard/easy days:} large pulses on hard days, minimal load and optional nap ($s=1$ block) on easy days.
		\item \textbf{Midday training:} pulses centered 6--8\,h before bedtime to keep $B(t)$ small.
		\item \textbf{Taper:} multiply volumes by a decaying factor while retaining brief $u_H$ to preserve $N$/$A$
		(``maintain intensity, reduce volume'' \cite{Mujika2003}); schedule earlier in the day to minimize $B(t)$.
		\item \textbf{Sleep extension/nap policy:} lengthen nightly $s(t)$ and add a short post‑lunch nap; ensure late‑day
		sessions are moved earlier (low $B$). Expect lower $S$, faster $F_a/F_c$ clearance, and lower $I$ \cite{Fullagar2015}.
	\end{itemize}

	\section{Model summary and parameter inventory}
	\label{sec:final-model}
	
	This section restates the full model in a compact form and inventories every state, parameter, and input used
	in the six–equation ODE. \emph{For readability we suppress the explicit time dependence}: all symbols
	$A,N,F_a,F_c,S,I,u_E,u_H,u_S,s,B,n,x$ below should be read as functions of time $t$ unless otherwise stated.
	
	\subsection{Compact statement of the six ODEs}
	\label{sec:compact-odes}
	\begingroup\small
	\begin{align}
		\dot A &= k_A(\alpha_E u_E+\alpha_H u_H)\,G_\mathrm{rec}(S,F_c)\Bigl(1-\tfrac{A}{K_A}\Bigr)
		\;-\;\tfrac{A}{\tau_A}\;-\;\theta_{AI} I, \tag{$\mathcal{A}$} \label{eq:compactA}\\
		\dot N &= k_N(\alpha_S u_S+\alpha_{HN} u_H)\,G_\mathrm{rec}(S,F_c)\,G_\mathrm{int}(u_E)\Bigl(1-\tfrac{N}{K_N}\Bigr)
		\;-\;\tfrac{N}{\tau_N}\;-\;\theta_{NI} I, \tag{$\mathcal{N}$} \label{eq:compactN}\\
		\dot F_a &= \gamma_E u_E+\gamma_H u_H+\gamma_S u_S
		\;-\;\Bigl(\tfrac{1}{\tau_{fa}}+\rho_a\, s\,q(B)\Bigr)F_a, \tag{$\mathcal{F}_a$} \label{eq:compactFa}\\
		\dot F_c &= \varepsilon F_a + \xi_c\,x
		\;-\;\Bigl(\tfrac{1}{\tau_{fc}}+\rho_c\, s\,q(B)\Bigr)F_c, \tag{$\mathcal{F}_c$} \label{eq:compactFc}\\
		\dot S &= \lambda_w(1-s)+\lambda_T(\gamma_E u_E+\gamma_H u_H+\gamma_S u_S)+\xi_s\,x
		\;-\;\Bigl(\tfrac{1}{\tau_S}+\mu_s\, s\,q(B)\Bigr)S, \tag{$\mathcal{S}$} \label{eq:compactS}\\
		\dot I &= \psi_0(\kappa_E u_E+\kappa_H u_H+\kappa_S u_S)\bigl(1+\chi_a F_a+\chi_c F_c\bigr)
		\;-\;\Bigl(\tfrac{1}{\tau_I}+\psi_s\, s\,q(B)+\psi_n\, n\Bigr)I. \tag{$\mathcal{I}$} \label{eq:compactI}
	\end{align}
	\endgroup
	
	\subsection{Inputs and helper relations (definitions)}
	\label{sec:inputs-helpers}
	\begingroup\small
	\begin{align}
		\text{Bedtime kernel: } \quad
		B &= \int_{0}^{h_b}\!\!\bigl(\beta_E u_E(t-\Delta)+\beta_H u_H(t-\Delta)+\beta_S u_S(t-\Delta)\bigr)\,K_b(\Delta)\,d\Delta, 
		\quad K_b(\Delta)=e^{-\Delta/\sigma_b}, \\
		\text{Sleep efficiency: } \quad
		q(B) &= \frac{q_0}{1+\eta B}, 
		\qquad 0<q_0\le 1,\ \eta\ge 0,\\
		\text{Recovery gate: } \quad
		G_\mathrm{rec}(S,F_c) &= \frac{1}{1+c_S S+c_c F_c},\\
		\text{Interference gate: } \quad
		G_\mathrm{int}(u_E) &= \frac{1}{1+\mu_E u_E}.
	\end{align}
	\endgroup
	
	\subsection{Output (readiness)}
	\label{sec:output-P}
	\begin{equation}
		P \;=\; w_{\mathrm{end}}A+w_{\mathrm{str}}N-\lambda_a F_a-\lambda_c F_c-\lambda_s S-\lambda_i I.
	\end{equation}
	
	\subsection{Variables and parameters (grouped inventory)}
	\label{sec:inventory}
	Table~\ref{tab:inventory} lists the six equations and all symbols they introduce or require. We include the input/helper relations and the readiness output for completeness. Units follow Section~\ref{sec:model-development}.
	\begin{center}
		\renewcommand{\arraystretch}{1.18}
		\setlength{\tabcolsep}{6pt}
		\begin{longtable}{>{\raggedright\arraybackslash}p{0.20\linewidth} >{\raggedright\arraybackslash}p{0.22\linewidth} >{\raggedright\arraybackslash}p{0.52\linewidth}}
			\caption{Variables and parameters used in the six–equation ODE and associated input/output relations (time dependence suppressed for readability).}
			\label{tab:inventory}\\
			\toprule
			\textbf{Equation} & \textbf{Symbol} & \textbf{Definition / role (units)} \\
			\midrule
			\endfirsthead
			\multicolumn{3}{l}{\small\itshape Table~\thetable\ (continued): Variables and parameters used in the model.}\\
			\toprule
			\textbf{Equation} & \textbf{Symbol} & \textbf{Definition / role (units)} \\
			\midrule
			\endhead
			\midrule
			\multicolumn{3}{r}{\small\itshape Continued on next page} \\
			\bottomrule
			\endfoot
			\bottomrule
			\endlastfoot
			
			\textbf{($\mathcal{A}$) Aerobic adaptation} & $A$ & Aerobic/endurance capacity (normalized $0\!-\!1$).\\
			& $k_A$ & Adaptation gain (day$^{-1}$ per unit of stimulus).\\
			& $\alpha_E,\alpha_H$ & Weights from endurance/HIIT stimulus to $A$ (dimensionless).\\
			& $G_\mathrm{rec}(S,F_c)$ & Recovery gate $=1/(1+c_S S+c_c F_c)$ (dimensionless).\\
			& $c_S,c_c$ & Couplings of sleep debt and chronic fatigue in recovery gate (dimensionless).\\
			& $K_A$ & Aerobic asymptote (often normalized to $1$).\\
			& $\tau_A$ & Detraining time constant of $A$ (days).\\
			& $\theta_{AI}$ & Injury penalty on realized gains (day$^{-1}$).\\
			\addlinespace[1mm]
			
			\textbf{($\mathcal{N}$) Neuromuscular adaptation} & $N$ & Strength/power capacity (normalized $0\!-\!1$).\\
			& $k_N$ & Adaptation gain (day$^{-1}$ per unit of stimulus).\\
			& $\alpha_S,\alpha_{HN}$ & Weights from strength/HIIT stimulus to $N$ (dimensionless).\\
			& $G_\mathrm{rec}(S,F_c)$ & Recovery gate as above.\\
			& $G_\mathrm{int}(u_E)$ & Interference gate $=1/(1+\mu_E u_E)$ (dimensionless).\\
			& $\mu_E$ & Strength of endurance–strength interference (dimensionless).\\
			& $K_N$ & Neuromuscular asymptote (often normalized to $1$).\\
			& $\tau_N$ & Detraining time constant of $N$ (days).\\
			& $\theta_{NI}$ & Injury penalty on realized gains (day$^{-1}$).\\
			\addlinespace[1mm]
			
			\textbf{($\mathcal{F}_a$) Acute fatigue} & $F_a$ & Acute/session fatigue (dimensionless).\\
			& $\gamma_E,\gamma_H,\gamma_S$ & Modality gains: training $\to F_a$ (day$^{-1}$ per input unit).\\
			& $\tau_{fa}$ & Baseline acute fatigue clearance (days).\\
			& $\rho_a$ & Sleep–accelerated clearance coefficient for $F_a$ (day$^{-1}$).\\
			& $s$ & Sleep opportunity indicator (1 during sleep; 0 otherwise).\\
			& $q(B)$ & Sleep efficiency, see “helpers” (dimensionless).\\
			\addlinespace[1mm]
			
			\textbf{($\mathcal{F}_c$) Chronic fatigue} & $F_c$ & Chronic/central fatigue (dimensionless).\\
			& $\varepsilon$ & Spillover from $F_a$ to $F_c$ (day$^{-1}$).\\
			& $\xi_c$ & Context stress $\to F_c$ gain (day$^{-1}$ per unit $x$).\\
			& $\tau_{fc}$ & Baseline chronic fatigue clearance (days).\\
			& $\rho_c$ & Sleep–accelerated clearance coefficient for $F_c$ (day$^{-1}$).\\
			& $x$ & Context stress input (travel/heat/psych load; exogenous).\\
			\addlinespace[1mm]
			
			\textbf{($\mathcal{S}$) Sleep debt / quality} & $S$ & Sleep debt / sleep quality state (larger $S$ is worse).\\
			& $\lambda_w$ & Wake accrual rate of sleep debt (debt units per day).\\
			& $\lambda_T$ & Training–day accrual scaling (debt per unit of training load).\\
			& $\xi_s$ & Context stress $\to S$ gain (per unit $x$).\\
			& $\tau_S$ & Baseline sleep debt decay (days).\\
			& $\mu_s$ & Sleep–accelerated paydown for $S$ (day$^{-1}$).\\
			& $s,\,q(B)$ & Sleep opportunity and efficiency as above.\\
			\addlinespace[1mm]
			
			\textbf{($\mathcal{I}$) Injury micro–damage} & $I$ & Micro–damage / injury hazard proxy (dimensionless).\\
			& $\psi_0$ & Base damage gain (day$^{-1}$ per input unit).\\
			& $\kappa_E,\kappa_H,\kappa_S$ & Modality weights in load $\to I$ (dimensionless).\\
			& $\chi_a,\chi_c$ & Fatigue amplification of damage (dimensionless).\\
			& $\tau_I$ & Baseline repair/remodeling time constant (days).\\
			& $\psi_s$ & Sleep–accelerated repair gain (day$^{-1}$).\\
			& $\psi_n$ & Nutrition–accelerated repair gain (day$^{-1}$).\\
			& $n$ & Nutrition/availability input (0–1).\\
			\addlinespace[1mm]
			
			\textbf{Inputs and helper relations} & $u_E,u_H,u_S$ & Training composition (endurance/HIIT/strength); exogenous, piecewise‑continuous.\\
			& $B$ & Bedtime kernel (recent late‑day training), $B=\int_0^{h_b}(\beta_E u_E+\beta_H u_H+\beta_S u_S)K_b(\Delta)d\Delta$.\\
			& $\beta_E,\beta_H,\beta_S$ & Modality weights in the bedtime kernel (dimensionless).\\
			& $K_b(\Delta)$ & Memory kernel $e^{-\Delta/\sigma_b}$ on $\Delta\in(0,h_b]$ (decay $\sigma_b$, horizon $h_b$).\\
			& $q(B)$ & Sleep efficiency $=q_0/(1+\eta B)$; $0<q_0\le1$, $\eta\ge0$.\\
			& $G_\mathrm{rec}(S,F_c)$ & Recovery gate $=1/(1+c_S S+c_c F_c)$.\\
			& $G_\mathrm{int}(u_E)$ & Interference gate $=1/(1+\mu_E u_E)$.\\
			& $s$ & Sleep opportunity schedule (0/1; nights and naps).\\
			& $x$ & Context stress (travel/heat/psychological load).\\
			& $n$ & Nutrition/availability proxy (0–1).\\
			\addlinespace[1mm]
			
			\textbf{Output (readiness)} & $P$ & Readiness $=w_{\mathrm{end}}A+w_{\mathrm{str}}N-\lambda_a F_a-\lambda_c F_c-\lambda_s S-\lambda_i I$.\\
			& $w_{\mathrm{end}},w_{\mathrm{str}}$ & Positive weights for endurance vs.\ strength contributions.\\
			& $\lambda_a,\lambda_c,\lambda_s,\lambda_i$ & Penalty weights on fatigue, debt, and damage.\\
		\end{longtable}
	\end{center}
	
	\paragraph{Notes on usage.}
	The **inputs** $u_E,u_H,u_S,s,n,x$ are set by the chosen regime (Section~\ref{sec:model-overview}); the
	**helpers** $B,q,G_\mathrm{rec},G_\mathrm{int}$ are computed from them and from the states; and the six ODEs
	\eqref{eq:compactA}–\eqref{eq:compactI} evolve the **states**. The right‑hand side is locally Lipschitz for bounded inputs and keeps states nonnegative by construction.
	
	\section{Deterministic parameter selection and synthetic-athlete regimes}
	\label{sec:deterministic-params}
	
	\paragraph{Deterministic calibration.}
	We now fix \emph{deterministic point estimates} for all parameters. Each choice is anchored to established evidence: the impulse–response canon and its two‑component refinements guide month‑scale adaptation gains and detraining time constants for $A$ and $N$ \cite{Banister1975,Busso2003}; taper practice constrains plausible magnitudes and time courses of performance changes and fatigue dissipation \cite{Mujika2003}; the endurance–strength interference literature motivates a monotone gate $G_{\mathrm{int}}(u_E)$ with sensitivity $\mu_E$ \cite{Hickson1980}; sleep and recovery studies support the nightly clearance multiplier $s\,q(B)$ and the bedtime‑proximity penalty $q(B)=q_0/(1+\eta\,B)$ \cite{Fullagar2015,Buchheit2014}; and load–injury work justifies a damage state $I$ with week‑scale repair and amplification by fatigue \cite{Gabbett2016}. We retain the sign structure, gates, and day‑level time units introduced in Sections~\ref{sec:model-development}–\ref{sec:final-model}, so that with daily inputs $(u_E,u_H,u_S,s,n,x)$ the model deterministically evolves $(A,N,F_a,F_c,S,I)$ and the readiness output $P(t)$ without additional statistical machinery.
	
	
	\paragraph{Normalization and units.}
	Daily training streams are scaled so that a “typical hard day” satisfies
	\(\int (u_E+u_H+u_S)\,dt \approx 1\). Capacities \(A,N\in[0,1]\) (dimensionless), fatigues \(F_a,F_c\), sleep
	debt \(S\) and micro‑damage \(I\) are nonnegative dimensionless states. Time constants are in days; the
	bedtime kernel acts on hours then contributes to daily \(q(B)\) as in \S\ref{sec:inputs-helpers}.
	
	\subsection{Nominal (all‑round) athlete: fixed parameter values}
	\label{sec:nominal-values}
	
	Tables~\ref{tab:deterministic-core} and \ref{tab:deterministic-helpers} list all parameters used by the six ODEs and
	helpers. Values were chosen to reproduce month‑scale adaptation, day‑to‑week fatigue, sleep effects on
	clearance, plausible injury‑hazard dynamics, and sensible readiness aggregation. Each block is followed by
	a brief rationale. (All symbols match Section~\ref{sec:final-model}.)
	
	\begin{center}
		\renewcommand{\arraystretch}{1.18}
		\setlength{\tabcolsep}{6pt}
		\begin{longtable}{@{}l l l l@{}}
			\caption{Deterministic values for the six ODEs (Nominal athlete). Units: day$^{-1}$ unless noted.}
			\label{tab:deterministic-core}\\
			\toprule
			\textbf{Equation} & \textbf{Parameter} & \textbf{Value} & \textbf{Meaning} \\
			\midrule
			\endfirsthead
			\toprule
			\textbf{Equation} & \textbf{Parameter} & \textbf{Value} & \textbf{Meaning} \\
			\midrule
			\endhead
			\midrule
			\multicolumn{4}{r}{\small\itshape Continued on next page}\\
			\bottomrule
			\endfoot
			\bottomrule
			\endlastfoot
			
			\textbf{$\dot A$ (capacity)} 
			& $k_A$ & 0.035 & stimulus gain \\
			& $\alpha_E,\alpha_H$ & 0.60,\ 0.40 & Endurance/HIIT weights to $A$ \\
			& $K_A$ & 1.00 & asymptote (normalized) \\
			& $\tau_A$ & 50\,(d) & detraining time constant \\
			& $\theta_{AI}$ & 0.030 & injury penalty on realized gains \\
			& $c_S,c_c$ & 0.50,\ 0.40 & recovery‑gate couplings in $G_{\mathrm{rec}}$ \\[2pt]
			
			\textbf{$\dot N$ (capacity)} 
			& $k_N$ & 0.040 & stimulus gain \\
			& $\alpha_S,\alpha_{HN}$ & 0.70,\ 0.30 & Strength/HIIT weights to $N$ \\
			& $K_N$ & 1.00 & asymptote (normalized) \\
			& $\tau_N$ & 35\,(d) & detraining time constant \\
			& $\theta_{NI}$ & 0.030 & injury penalty on realized gains \\
			& $\mu_E$ & 0.30 & interference gate $G_{\mathrm{int}}=1/(1+\mu_E u_E)$ \\[2pt]
			
			\textbf{$\dot F_a$ (acute fatigue)} 
			& $\gamma_E,\gamma_H,\gamma_S$ & 0.55,\ 0.90,\ 0.45 & load $\to F_a$ gains \\
			& $\tau_{fa}$ & 1.0\,(d) & baseline clearance \\
			& $\rho_a$ & 0.50 & sleep‑accelerated clearance (multiplies $s\,q(B)$) \\[2pt]
			
			\textbf{$\dot F_c$ (chronic fatigue)} 
			& $\varepsilon$ & 0.18 & spillover $F_a \to F_c$ \\
			& $\xi_c$ & 0.20 & context stress $\to F_c$ gain \\
			& $\tau_{fc}$ & 10\,(d) & baseline clearance \\
			& $\rho_c$ & 0.40 & sleep‑accelerated clearance \\[2pt]
			
			\textbf{$\dot S$ (sleep debt/quality)} 
			& $\lambda_w$ & 0.30 & wake accrual \\
			& $\lambda_T$ & 0.20 & training‑day accrual scaling \\
			& $\xi_s$ & 0.15 & stress $\to S$ gain \\
			& $\tau_S$ & 6\,(d) & baseline paydown \\
			& $\mu_s$ & 0.30 & sleep‑accelerated paydown \\[2pt]
			
			\textbf{$\dot I$ (micro‑damage)} 
			& $\psi_0$ & 0.40 & base damage gain \\
			& $\kappa_E,\kappa_H,\kappa_S$ & 0.30,\ 0.35,\ 0.35 & modality weights in load $\to I$ \\
			& $\chi_a,\chi_c$ & 0.25,\ 0.25 & fatigue amplification of damage \\
			& $\tau_I$ & 14\,(d) & baseline repair \\
			& $\psi_s$ & 0.25 & sleep‑accelerated repair \\
			& $\psi_n$ & 0.20 & nutrition‑accelerated repair \\
		\end{longtable}
	\end{center}
	
	\begin{center}
		\renewcommand{\arraystretch}{1.18}
		\setlength{\tabcolsep}{6pt}
		\begin{longtable}{@{}l l l l@{}}
			\caption{Helpers, kernel, readiness weights, and initial conditions (Nominal athlete).}
			\label{tab:deterministic-helpers}\\
			\toprule
			\textbf{Block} & \textbf{Parameter} & \textbf{Value} & \textbf{Meaning / units} \\
			\midrule
			\endfirsthead
			\toprule
			\textbf{Block} & \textbf{Parameter} & \textbf{Value} & \textbf{Meaning / units} \\
			\midrule
			\endhead
			\midrule
			\multicolumn{4}{r}{\small\itshape Continued on next page}\\
			\bottomrule
			\endfoot
			\bottomrule
			\endlastfoot
			\textbf{Sleep efficiency} & $q_0$ & 0.88 & baseline sleep efficiency \\
			& $\eta$ & 1.00 & bedtime penalty in $q(B)=q_0/(1+\eta B)$ \\[2pt]
			\textbf{Bedtime kernel} & $\beta_E,\beta_H,\beta_S$ & 0.30,\ 0.40,\ 0.30 & modality weights \\
			& $\sigma_b$ & 1.5\,h & exponential decay of kernel \\
			& $h_b$ & 6\,h & horizon (final hours before lights‑out) \\[2pt]
			\textbf{Readiness $P$} & $w_{\mathrm{end}},w_{\mathrm{str}}$ & 0.50,\ 0.50 & positive weights \\
			& $\lambda_a,\lambda_c,\lambda_s,\lambda_i$ & 0.30,\ 0.50,\ 0.40,\ 0.60 & penalties on $F_a,F_c,S,I$ \\[2pt]
			\textbf{Initial states} & $A_0,N_0$ & 0.65,\ 0.65 & preseason capacities (normalized) \\
			& $F_{a0},F_{c0},S_0,I_0$ & 0.20,\ 0.20,\ 0.20,\ 0.15 & latent recovery/risk states \\
		\end{longtable}
	\end{center}
	
	Month–scale adaptation with first–order detraining for the two capacities is standard in the training–response canon, so we set time constants to reflect typical endurance (\(\tau_A\!\approx\!50\,\mathrm{d}\)) and neuromuscular (\(\tau_N\!\approx\!35\,\mathrm{d}\)) decay and choose stimulus gains that yield realistic multi‑week improvements under sustained load \cite{Banister1975,Busso2003}. These magnitudes are also consistent with taper/peaking evidence in which performance rebounds over 1–3 weeks as training load is reduced \cite{Mujika2003}. The endurance–strength \emph{interference} effect is captured with a damping gate on \(N\) that depends on concurrent endurance work; taking \(\mu_E\!=\!0.30\) produces a \(\sim 20\text{–}35\%\) reduction in realized strength gains when endurance volume is high, in line with classic concurrent‑training findings \cite{Hickson1980}. 
	
	Fatigue operates on two time scales: acute clears in roughly a day (\(\tau_{fa}\!\approx\!1\,\mathrm{d}\)), chronic in roughly a week to fortnight (\(\tau_{fc}\!\approx\!10\,\mathrm{d}\)), and both are accelerated during sleep via coefficients \(\rho_a,\rho_c\). Night‑to‑night modulation uses a sleep‑efficiency factor \(q(B)=q_0/(1+\eta B)\); with \(\sigma_b\!=\!1.5\,\mathrm{h}\) and \(h_b\!=\!6\,\mathrm{h}\) for the bedtime kernel and \(\eta\!=\!1\), a late, intense evening session lowers efficiency by \(\mathcal{O}(10\text{–}25\%)\), which matches applied sleep literature on training close to lights‑out \cite{Fullagar2015,Buchheit2014}. 
	
	Damage/repair dynamics follow load–injury associations: base damage gain and modality weights make high‑impact or high‑intensity work more injurious; fatigue multipliers \(\chi_a,\chi_c\!\approx\!0.25\) encode poorer mechanics and slower remodeling under load; and week‑scale repair (\(\tau_I\!\approx\!14\,\mathrm{d}\)) is accelerated by sleep and adequate nutrition (\(\psi_s,\psi_n>0\)) \cite{Gabbett2016}. Finally, readiness weights treat \(A\) and \(N\) symmetrically in the Nominal profile, while penalties are heavier for chronic fatigue and micro‑damage (\(\lambda_c,\lambda_i>\lambda_a\)), reflecting their outsized impact on next‑day performance and availability noted in the applied literature \cite{Buchheit2014,Fullagar2015}.
	
	
	
	\subsection{Synthetic athletes: three deterministic parameter regimes}
	\label{sec:synthetic-athletes}
	
	To emulate inter‑individual variability \emph{without} a Bayesian layer, we define three fixed parameter
	regimes. Each regime is a small, physiologically coherent deviation from the Nominal values in
	Tables~\ref{tab:deterministic-core}–\ref{tab:deterministic-helpers}. These synthetic athletes can be used to later compare training schedules (regimes) and illustrate how recommendations depend on athlete type.
	
	
	
	\paragraph{SA‑E: Aerobic responder, durable sleeper.}
	Stronger response to endurance/HIIT stimuli, faster clearances, less sensitive to late sessions; slightly
	endurance‑weighted readiness.
	
	\begin{center}
		\small
		\begin{tabular}{@{}l l@{}}
			\toprule
			\textbf{Parameter overrides (vs.\ Nominal)} & \textbf{New value} \\
			\midrule
			$k_A=0.045$;\ \ $\alpha_E,\alpha_H=(0.70,0.30)$;\ \ $k_N=0.035$ & (stronger A, modest N)\\
			$\tau_{fa}=0.9$\,d,\ $\rho_a=0.60$;\ \ $\tau_{fc}=9$\,d,\ $\rho_c=0.50$ & (faster clearance)\\
			$\mu_E=0.18$ & (weaker interference) \\
			$\eta=0.70$ & (late‑session penalty milder) \\
			$w_{\mathrm{end}},w_{\mathrm{str}}=(0.65,0.35)$ & (readiness preference) \\
			\bottomrule
		\end{tabular}
	\end{center}
	
	\paragraph{SA‑S: Power/neuromuscular responder, interference‑sensitive.}
	Stronger response to strength, larger endurance–strength interference, slightly higher damage from strength.
	
	\begin{center}
		\small
		\begin{tabular}{@{}l l@{}}
			\toprule
			\textbf{Parameter overrides (vs.\ Nominal)} & \textbf{New value} \\
			\midrule
			$k_N=0.055$;\ \ $\alpha_S,\alpha_{HN}=(0.80,0.20)$;\ \ $k_A=0.030$ & (stronger N) \\
			$\mu_E=0.55$ & (strong interference from endurance) \\
			$\gamma_S=0.60$;\ \ $\kappa_S=0.45$;\ \ $\psi_0=0.45$;\ \ $\chi_a=\chi_c=0.30$ & (strength loads damage a bit more) \\
			$w_{\mathrm{end}},w_{\mathrm{str}}=(0.35,0.65)$ & (readiness preference) \\
			\bottomrule
		\end{tabular}
	\end{center}
	
	\paragraph{SA‑R: Recovery‑limited, stress‑sensitive sleeper.}
	Sleep and central fatigue throttle gains more; clearances slower; late training hurts sleep more; higher readiness
	penalties on sleep and damage.
	
	\begin{center}
		\small
		\begin{tabular}{@{}l l@{}}
			\toprule
			\textbf{Parameter overrides (vs.\ Nominal)} & \textbf{New value} \\
			\midrule
			$c_S,c_c=(0.80,0.60)$ & (recovery gate tighter) \\
			$\tau_{fa}=1.3$\,d,\ $\rho_a=0.35$;\ \ $\tau_{fc}=14$\,d,\ $\rho_c=0.30$ & (slower clearance) \\
			$\tau_S=8$\,d,\ $\mu_s=0.25$;\ \ $\xi_s=0.25$ & (sleep debt clears slowly; stress loads sleep more) \\
			$\eta=1.50$;\ \ $q_0=0.82$ & (late sessions more punitive; lower baseline sleep efficiency) \\
			$\psi_s=0.20$;\ \ $\psi_n=0.15$ & (repair less helped by sleep/nutrition) \\
			$\lambda_s=0.60$;\ \ $\lambda_i=0.70$ & (higher readiness penalties) \\
			Initial: $S_0=0.30$ & (starts with higher sleep debt) \\
			\bottomrule
		\end{tabular}
	\end{center}
			
	\subsection{Model illustration and analysis}
	\label{sec:deterministic-runs}
	
	We illustrate the point--estimate model by simulating several canonical regimes on the Nominal athlete
	and on the three synthetic profiles from §6.2. Each run reports:
	(i) peak readiness $P_{\max}$ and its time;
	(ii) the time--averaged readiness $\overline{P}$ over the horizon;
	and (iii) the fraction of saved time points with micro--damage $I$ above a ``high'' threshold ($I>0.8$).\footnote{%
		Metrics are computed from the simulated trajectory using the readiness aggregator
		$P(t)=w_{\text{end}}A+w_{\text{str}}N-\lambda_aF_a-\lambda_cF_c-\lambda_sS-\lambda_iI$
		and the six ODEs (A)--(I) in §4--§5; the model uses the deterministic parameter sets of §6.}
	
	\begin{table}[h!]
		\centering
		\small
		\begin{tabular}{llllrrr}
			\toprule
			Athlete & Regime & Horizon & Description & $P_{\max}$ & $t_{\text{peak}}$ (d) & $\overline{P}$ \\
			\midrule
			Nominal & Early--AM HIIT & 42 d &
			Short HIIT AM $+$ small $E$, low $B(t)$ & 0.494 & 4.29 & 0.345 \\
			Nominal & Evening intensity & 42 d &
			$E{+}H$ in evening $\Rightarrow$ larger $B(t)$ & 0.494 & 3.79 & 0.332 \\
			SA--E & Polarized (midday) & 42 d &
			High $E$, tiny $H$, both midday & 0.522 & 4.58 & 0.368 \\
			SA--S & Split day (AM $E$, PM $S$) & 42 d &
			AM endurance, PM strength & 0.500 & 4.58 & 0.343 \\
			SA--R & Alternating hard/easy & 42 d &
			Hard: $E{+}H{+}S$; Easy: minimal $+$ nap & 0.422 & 7.00 & 0.291 \\
			Nominal & Taper (early sessions) & 21 d &
			Volume $\downarrow$, intensity kept, early day & 0.518 & 5.04 & 0.452 \\
			SA--R & Sleep extension $+$ nap & 28 d &
			Longer nightly $s(t)$ $+$ short nap & 0.464 & 6.62 & 0.384 \\
			\bottomrule
		\end{tabular}
		\caption{Deterministic runs (fixed parameters and regimes). All runs had
			$\mathrm{frac\_time}(I>0.8)=0.0$ at the default hazard threshold.}
		\label{tab:runs}
	\end{table}

	\begin{figure}
		\centering
	\begin{tikzpicture}
		\begin{groupplot}[
			group style={group size=2 by 1, horizontal sep=18mm},
			width=0.46\textwidth, height=0.34\textwidth,
			grid=both,
			xlabel={Day}, ylabel={$P(t)$ (readiness)},
			% --- legend settings (shared across the group) ---
			legend to name=combinedLegend,         % collect legend entries here
			legend columns=3,                       % tweak to 2/3/4 as you prefer
			legend style={
				draw=none, fill=none, font=\footnotesize,
				/tikz/every even column/.style={column sep=0.8em},
				/tikz/every odd column/.style={}
			}
			]
			
			% ---------- Left panel ----------
			\nextgroupplot[
			title={(a) 42-day blocks (training schedules)}
			]
			\addplot+[thick, mark=none] table[col sep=comma,x index=0,y index=1]{figs/nominal_early_am_hiit.csv};
			\addlegendentry{Nominal: Early–AM HIIT}
			
			\addplot+[thick, mark=none] table[col sep=comma,x index=0,y index=1]{figs/nominal_evening_intensity.csv};
			\addlegendentry{Nominal: Evening intensity}
			
			\addplot+[thick, mark=none, teal!75!black] table[col sep=comma,x index=0,y index=1]{figs/sae_polarized_midday.csv};
			\addlegendentry{SA‑E: Polarized (midday)}
			
			\addplot+[thick, mark=none, violet!80!black] table[col sep=comma,x index=0,y index=1]{figs/sas_split_day.csv};
			\addlegendentry{SA‑S: Split day}
			
			% SA‑R — Alternating hard/easy
			\addplot+[thick, mark=none, orange!85!black] table[x index=0, y index=1, col sep=comma]{figs/sar_alt_hard_easy.csv};
			
			% ---------- Right panel ----------
			\nextgroupplot[
			title={(b) Event-facing \& sleep-hygiene blocks},
			yticklabels=\empty
			]
			\addplot+[thick, mark=none, black] table[col sep=comma,x index=0,y index=1]{figs/nominal_taper.csv};
			
			\addplot+[thick, mark=none, green!55!black] table[col sep=comma,x index=0,y index=1]{figs/sar_sleep_extension.csv};
			
		\end{groupplot}
		
		% Place the shared legend centered below both panels (adjust vspace as needed).
		\node at ($(group c1r1.south)!0.5!(group c2r1.south)$) [below=1.0em]
		{\pgfplotslegendfromname{combinedLegend}};
		

		
	\end{tikzpicture}
		\caption{\textbf{Readiness trajectories $P(t)$ for representative regime$\times$athlete scenarios.}
	Panel~(a) overlays 42‑day training blocks. Moving intensity earlier (blue) improves the readiness
	envelope relative to evening intensity (red) by preserving nightly clearance.
	An endurance‑leaning responder benefits most from a midday polarized block (teal), while a strength‑leaning
	responder retains strong peaks under a split‑day schedule (violet) at some average‑readiness cost from late
	strength sessions. Recovery‑limited athletes prefer alternating microcycles (orange) to avoid accumulation
	of chronic load. Panel~(b) shows event‑facing and sleep‑hygiene blocks with shorter horizons: a classic
	taper (black) yields a readiness rebound near day~5, and extending sleep plus a short nap (green) raises the
	average readiness for a recovery‑limited profile. (Exact peaks and means are reported in
	Table~\ref{tab:runs}.)}
	\label{fig:pgfplots_readiness}
	\end{figure}


	
	\paragraph{Summary.}
	Across 42--day microcycles, the \emph{midday polarized} block for the aerobic responder (SA--E)
	achieves the highest envelope ($P_{\max}=0.522$) and the largest mean readiness ($\overline{P}=0.368$).
	On the Nominal athlete, \emph{evening intensity} matches the early--morning peak height
	($P_{\max}=0.494$) yet lowers the mean ($0.332$ vs.\ $0.345$), consistent with a sleep--efficiency
	penalty after late sessions. The \emph{split--day} pattern (AM $E$, PM $S$) fits the neuromuscular
	responder (SA--S) but the evening strength block trims the average ($\overline{P}=0.343$).
	The recovery--limited athlete (SA--R) performs poorly on an \emph{alternating hard/easy} schedule
	($\overline{P}=0.291$), yet improves markedly under a \emph{sleep--extension + nap} policy
	($\overline{P}=0.384$). Finally, the 3--week \emph{taper} in the Nominal athlete exhibits the expected
	profile: mean readiness rises to $0.452$ with a peak around day~5.
	
	\paragraph{Mechanistic interpretation (link to the ODE terms).}
	The observed patterns follow directly from the signed couplings and gates in §4:
	(i) \textbf{Evening vs.~early sessions.} Evening $E/H$ increases the bedtime kernel $B(t)$ and reduces
	that night’s sleep efficiency $q(B)=q_0/(1+\eta B)$, so the sleep--accelerated clearance terms in
	(\textit{Fa})--(\textit{I}) are smaller. This depresses $\overline{P}$ while leaving short--term peaks similar.
	(ii) \textbf{Midday polarized (SA--E).} Stronger aerobic stimulus gain and faster clearances (SA--E overrides)
	combined with small $B(t)$ raise $A$, lower $F_c$ and $S$, and yield the largest envelope.
	(iii) \textbf{Split day (SA--S).} The PM strength pulse raises $B(t)$; despite stronger $N$ stimulus, the evening
	penalty reduces nightly repair, so the mean is comparable to Nominal early--AM HIIT rather than superior.
	(iv) \textbf{Recovery--limited (SA--R).} Tighter recovery gate $G_{\mathrm{rec}}(S,F_c)=1/(1+c_SS+c_cF_c)$,
	slower clearances, and harsher late--session penalty (larger $\eta$) push the peak later (day~7) and lower
	$\overline{P}$ on alternating hard/easy; by contrast, longer $s(t)$ plus a short nap increases the
	sleep--accelerated clearance $(\cdot)\,s\,q(B)$ in (\textit{Fa}),(\textit{Fc}),(\textit{S}),(\textit{I}), lifting the entire envelope.
	(v) \textbf{Taper.} Volume reduction lowers forcing into $F_a$ and spillover to $F_c$ while earlier sessions keep
	$B(t)$ small; capacities are maintained by the saturations $(1-A/K_A)$ and $(1-N/K_N)$ while
	$S$ and $I$ decay, producing a peak near day~5.
	
	\paragraph{Risk and availability.}
	At the default hazard threshold ($I>0.8$), all runs spent \emph{zero} proportion of time in high risk, consistent
	with week--scale repair in (\textit{I}) and with the moderate loads used here.
	
	\paragraph{Practical takeaways for regime design.}
	(i) For identical work late vs.\ early, expect a small but systematic drop in mean readiness; move
	key intensity earlier when possible. (ii) Midday, high--volume polarized endurance is a strong default
	for aerobic responders, avoiding both the interference gate $G_{\mathrm{int}}(u_E)$ and the bedtime penalty.
	(iii) Recovery--limited profiles benefit disproportionately from increased sleep opportunity and naps.
	(iv) For event week, the taper confirms common practice: reduce volume, retain brief intensity, and
	run sessions well before bedtime.
	
	\section{Long‑term behavior of solutions}
	\label{sec:longterm}
	
	In this section we analyze the qualitative (long‑time) behavior of the six–state system introduced in
	Sections 4 and 5.% 
	Due to time constraints, we keep the discussion computation‑free:
	where numerical values would normally be inserted (nullcline intersections, eigenvalues), we state the
	algebraic forms and indicate the expected behavior under our chosen parameter magnitudes.%
	This section relies on the same sign structure, gates, and positivity/forward‑invariance properties
	established earlier (Sec.~4.3).

	
	\subsection{Setups: stationary vs.\ weekly‑repeating inputs}
	
	The model is a forced ODE $\dot z = f(z,u)$ with state 
	$z=(A,N,F_a,F_c,S,I)^{\top}$ and exogenous inputs 
	$u=(u_E,u_H,u_S,B,s,n,x)^{\top}$ (Sec.~5.2).
	We consider two practically relevant long‑time regimes:
	
	\paragraph{(i) Stationary inputs (constant-in-time).}
	This is a deterministic abstraction of a “typical” training climate in which daily
	inputs are held at constant effective levels (e.g., daily averages or block plateaus): 
	$u_E(t)\equiv \bar u_E$, $u_H\equiv \bar u_H$, $u_S\equiv \bar u_S$, $s\equiv \bar s$, 
	$n\equiv \bar n$, $x\equiv \bar x$, and $B(t)\equiv \bar B$ computed from the fixed training levels via the kernel in~(11).
	Under bounded inputs the right‑hand side is locally Lipschitz, the nonnegative orthant is forward‑invariant, and 
	$A,N$ are bounded by their asymptotes $K_A,K_N$ (Sec.~4.3). 
	Therefore at least one steady solution (equilibrium) $z^\star$ exists.
	
	\paragraph{(ii) Weekly‑repeating inputs (calendar regimes).}
	Most regimes in Sec.~2 repeat by design (e.g., 7‑day microcycles, tapers). For a period $T$ (e.g., $T\!=\!7$\,d)
	let $\mathcal{P}:\mathcal{X}\to\mathcal{X}$ be the Poincaré map that advances the solution one period.
	Over a compact, convex, positively invariant state set $\mathcal{X}$ the map $\mathcal{P}$ is continuous; 
	by Brouwer’s fixed‑point theorem it has at least one fixed point $z^\star$ corresponding to a $T$‑periodic solution
	$z^\star(t)=z^\star(t+T)$. In practice this solution is the \emph{entrained} weekly trajectory under the repeated regime.

	
	\subsection{Closed‑form nullclines and steady–state formulas (stationary inputs)}
	\label{sec:nullclines}
	
	With stationary inputs we can write the six equilibrium relations by setting $\dot z = 0$.  
	For readability we denote the stationary inputs with bars (e.g., $\bar u_E$) and the stationary sleep‑efficiency 
	$q(\bar B)=\bar q$.
	
	\paragraph{Fatigue and sleep states.} 
	These equations are affine in the state, yielding explicit nullclines:
	\begin{align}
		F_a^\star 
		&= \frac{\gamma_E\bar u_E+\gamma_H\bar u_H+\gamma_S\bar u_S}
		{\frac{1}{\tau_{fa}}+\rho_a\,\bar s\,\bar q},  \label{eq:Fa-star}\\[2mm]
		F_c^\star 
		&= \frac{\varepsilon F_a^\star + \xi_c\,\bar x}
		{\frac{1}{\tau_{fc}}+\rho_c\,\bar s\,\bar q}, \label{eq:Fc-star}\\[2mm]
		S^\star 
		&= \frac{\lambda_w(1-\bar s) + \lambda_T(\gamma_E\bar u_E+\gamma_H\bar u_H+\gamma_S\bar u_S)
			+ \xi_s\,\bar x}
		{\frac{1}{\tau_S}+\mu_s\,\bar s\,\bar q}. \label{eq:S-star}
	\end{align}
	Each fraction is well‑defined and positive because the denominators are strictly positive (baseline clearance plus sleep‑accelerated paydown).%
	These expressions formalize the intuitive trade‑offs discussed in Sec.~4.2: larger nightly sleep $\bar s$ and
	lower bedtime proximity $\bar B$ (hence larger $\bar q$) reduce the steady \(F_a,F_c,S\).
	
	\paragraph{Micro‑damage.}
	The $I$ equation is linear in $I$ but multiplicatively coupled to $F_a,F_c$:
	\begin{equation}
		I^\star \;=\; 
		\frac{\psi_0 \left(\kappa_E\bar u_E+\kappa_H\bar u_H+\kappa_S\bar u_S\right)
			\left(1+\chi_a F_a^\star+\chi_c F_c^\star\right)}
		{\frac{1}{\tau_I}+\psi_s\,\bar s\,\bar q+\psi_n\,\bar n}. \label{eq:I-star}
	\end{equation}
	As expected, better sleep ($\bar s\uparrow$, $\bar q\uparrow$) and better fueling ($\bar n\uparrow$) reduce $I^\star$, while greater load
	and/or fatigue amplification increase it (via the numerator).
	
	\paragraph{Trainable capacities.}
	Let
	\[
	G_{\mathrm{rec}}^\star \equiv \frac{1}{1+c_S S^\star + c_c F_c^\star}, 
	\qquad
	G_{\mathrm{int}}^\star \equiv \frac{1}{1+\mu_E \bar u_E},
	\]
	and define the effective capacity stimuli
	\[
	G_A \equiv k_A(\alpha_E\bar u_E+\alpha_H\bar u_H)\,G_{\mathrm{rec}}^\star, 
	\qquad
	G_N \equiv k_N(\alpha_S\bar u_S+\alpha_{HN}\bar u_H)\,G_{\mathrm{rec}}^\star G_{\mathrm{int}}^\star.
	\]
	Rearranging the logistic‑with‑detraining forms (A)–(N) yields closed‑form steady states:
	\begin{align}
		A^\star &= 
		\frac{G_A - \theta_{AI} I^\star}{\tfrac{G_A}{K_A} + \tfrac{1}{\tau_A}}, 
		\qquad
		0 \le A^\star \le K_A, \label{eq:A-star}\\[1mm]
		N^\star &= 
		\frac{G_N - \theta_{NI} I^\star}{\tfrac{G_N}{K_N} + \tfrac{1}{\tau_N}}, 
		\qquad
		0 \le N^\star \le K_N. \label{eq:N-star}
	\end{align}
	Two mechanisms depress realized capacity at steady state: (i) poor recovery (small $G_{\mathrm{rec}}^\star$) via large $S^\star,F_c^\star$ 
	and (ii) the injury penalty ($\theta_{\cdot I} I^\star$). 
	The endurance–strength interference appears only in $N^\star$ through $G_{\mathrm{int}}^\star$ (smaller when $\bar u_E$ is high).
	
	\paragraph{Readiness at steady state.} 
	Substituting \eqref{eq:Fa-star}–\eqref{eq:N-star} into the aggregator gives
	\begin{equation}
		P^\star \;=\; w_{\mathrm{end}}A^\star + w_{\mathrm{str}}N^\star 
		- \lambda_a F_a^\star - \lambda_c F_c^\star - \lambda_s S^\star - \lambda_i I^\star.
		\label{eq:P-star}
	\end{equation}
	This expression provides a transparent algebraic summary of how constant training, sleep, and context
	determine the long‑run readiness envelope for a given parameterization (Sec.~6).%

	
	\subsection{Jacobian linearization and local stability (what to expect)}
	\label{sec:jacobian}
	
	Let $J(z^\star)$ be the Jacobian of the right‑hand side at an equilibrium $z^\star$.
	Its sign structure is informative even without numerical values:
	
	\begin{itemize}
		\item \textbf{Diagonal terms negative.}  
		Each equation contains a linear sink in its own state (detraining for $A,N$; 
		baseline+\;sleep/nutrition clearance for $F_a,F_c,S,I$). Therefore $\partial \dot A/\partial A<0$, 
		$\partial \dot N/\partial N<0$, $\partial \dot F_a/\partial F_a=-(1/\tau_{fa}+\rho_a\bar s\bar q)<0$, etc.
		
		\item \textbf{Upper–triangular fatigue–risk block.}
		$F_c$ depends on $F_a$ ($\partial \dot F_c/\partial F_a=\varepsilon>0$) and itself (negative diagonal),
		$S$ is self‑damped only, and $I$ depends on $F_a,F_c$ through the fatigue amplifiers ($\partial \dot I/\partial F_a=\psi_0\chi_a\cdot\text{load}>0$, 
		$\partial \dot I/\partial F_c=\psi_0\chi_c\cdot\text{load}>0$) and on itself (negative diagonal).
		Thus the $(F_a,F_c,S,I)$ submatrix is lower‑block‑triangular with strictly negative diagonal entries.
		
		\item \textbf{Recovery and injury gates into capacities.}
		$A$ and $N$ receive \emph{inhibitory} couplings from $S,F_c$ via $G_{\mathrm{rec}}$ and from $I$ via $\theta_{\cdot I}$
		(all partials $\le 0$), and $N$ receives an additional inhibitory coupling from $\bar u_E$ via $G_{\mathrm{int}}$.
	\end{itemize}
	
	\paragraph{Sufficient conditions for local asymptotic stability.}
	A Gershgorin–type bound shows that if each diagonal sink dominates its row’s (absolute) cross‑couplings,
	all eigenvalues have negative real parts. Concretely:
	\begin{align*}
		\left(\tfrac{1}{\tau_{fa}}+\rho_a\bar s\bar q\right) &>\; 0 \quad\text{(already true)},\\
		\left(\tfrac{1}{\tau_{fc}}+\rho_c\bar s\bar q\right) &>\; \varepsilon,\\
		\left(\tfrac{1}{\tau_I}+\psi_s\bar s\bar q+\psi_n\bar n\right) &>\; 
		\psi_0\big(\chi_a\lvert \partial I\!/\partial F_a\rvert+\chi_c\lvert \partial I\!/\partial F_c\rvert\big),\\
		\left(\tfrac{G_A}{K_A}+\tfrac{1}{\tau_A}\right) &>\; 
		\left\lvert \frac{\partial \dot A}{\partial S}\right\rvert+
		\left\lvert \frac{\partial \dot A}{\partial F_c}\right\rvert+
		\theta_{AI},\qquad
		\left(\tfrac{G_N}{K_N}+\tfrac{1}{\tau_N}\right) \;>\;
		\left\lvert \frac{\partial \dot N}{\partial S}\right\rvert+
		\left\lvert \frac{\partial \dot N}{\partial F_c}\right\rvert+
		\theta_{NI}.
	\end{align*}
	At our nominal magnitudes (Sec.~6), the fatigue and risk sinks are intentionally large
	(1–14\,d scales plus sleep/nutrition multipliers), while cross‑couplings are modest;
	hence we expect the Jacobian at the stationary equilibrium to be Hurwitz (all eigenvalues with negative
	real parts).%
	
	\paragraph{Weekly‑repeating inputs.}
	Linearization about a $T$‑periodic solution $z^\star(t)$ involves the monodromy matrix $\Phi(T)$ (state‑transition over one period).
	Local stability corresponds to all Floquet multipliers inside the unit circle. 
	Given the strong sinks in each state during sleep ($s=1$, $q(B)$ near $q_0$), and the block‑triangular structure noted above, 
	we expect $\rho(\Phi(T))<1$ (contractivity) under moderate loads—i.e., the weekly Poincaré fixed point is locally attracting.
	
	\subsection{Bounds, invariance, and uniqueness heuristics}
	
	\paragraph{Forward invariance and boundedness.}
	With nonnegative inputs, the nonnegative orthant is positively invariant (all sinks are proportional to the state),
	and $A,N$ remain in $[0,K_A]\times[0,K_N]$ by construction. Thus all trajectories enter and remain in a compact set.%
	These properties were established when we formulated the right‑hand side and carry over here.
	
	\paragraph{Uniqueness of steady behavior (heuristic).}
	Because the $(F_a,F_c,S,I)$ block is driven \emph{linearly} by exogenous inputs with strictly negative diagonals,
	and the capacities receive only inhibitory feedback from these states, multiple isolated equilibria are unlikely
	under stationary inputs. We therefore \emph{expect} a unique equilibrium $z^\star$ and a unique $T$‑periodic attractor under
	weekly‑repeating inputs for the parameter ranges of Sec.~6, with global attraction from physiologically reasonable initial
	conditions. (A formal proof would check monotonicity/contractivity of the Poincaré map; we do not pursue it here.)
	
	\subsection{Interpretation for coaching decisions}
	
	Even without numerics, Eqs.~\eqref{eq:Fa-star}–\eqref{eq:P-star} support clear design rules:
	
	\begin{itemize}
		\item \textbf{Move intensity away from bedtime.} 
		$B\!\downarrow\Rightarrow \bar q\!\uparrow$ improves the denominators of 
		\eqref{eq:Fa-star}–\eqref{eq:S-star} and \eqref{eq:I-star}, reducing steady $F_a,F_c,S,I$ and increasing $P^\star$.
		This formalizes the “evening penalty” described qualitatively in Sec.~2 and implemented in $q(B)$.
		
		\item \textbf{Sleep and fueling leverage.}
		Longer sleep windows ($\bar s\uparrow$) and adequate nutrition ($\bar n\uparrow$) raise the clearance terms in 
		\eqref{eq:S-star} and \eqref{eq:I-star}, respectively, lowering steady sleep debt and micro‑damage.
		
		\item \textbf{Concurrent‑training interference.}
		High steady endurance work ($\bar u_E$) damps $N^\star$ through $G_{\mathrm{int}}^\star$, 
		capturing the classic interference effect; practical implication: separate heavy strength from high endurance load 
		to raise $N^\star$ without sacrificing $A^\star$.
		
		\item \textbf{Risk throttles capacity.}
		Large $I^\star$ pulls down both $A^\star$ and $N^\star$ (\eqref{eq:A-star}–\eqref{eq:N-star}) 
		and also reduces $P^\star$ directly; thus regimes that keep $I$ low (earlier sessions, sleep extension, nutrition) 
		should have higher long‑run readiness envelopes.
	\end{itemize}
	
	\subsection{What we would compute next}
	As numerical computations, one would:
	(i) evaluate \eqref{eq:Fa-star}–\eqref{eq:N-star} at the deterministic parameter sets of Sec.~6 
	(Nominal, SA‑E, SA‑S, SA‑R) and at regime‑specific stationary inputs (or weekly averages);
	(ii) assemble $J(z^\star)$ and report its six eigenvalues (stationary) or the Floquet spectrum (weekly);
	and (iii) compare $P^\star$ across regimes and athlete types.  
	Based on our parameterization (Sec.~6), we \emph{expect} all eigenvalues to have negative real parts 
	(moderate off‑diagonal couplings, strong diagonal sinks), the weekly Poincaré map to be a contraction 
	under the loads we explore, and $P^\star$ to follow the qualitative ordering already observed in Sec.~6.3’s 
	simulation narratives (early/midday intensity and sleep‑hygiene policies raising the envelope).%
	Those narratives and parameter magnitudes are documented in Sec.~6 and the accompanying code 
	module (\texttt{AthleteModel}), which implements the identical right‑hand side Eqs.~(A)–(I). 
	
	% =========================
	% 11. Solution and results
	% =========================
	\section{Solution and results}\label{sec:solution-results}
	
	\paragraph{What counts as a “solution” here.}
	For our purposes a \emph{solution} is a fully specified evaluation of a coach-controlled \emph{regime} (left column of Fig.~\ref{fig:wiring}) when passed through the six-state ODE (center column) to produce the derived readiness signal \(P(t)\) (right column).
	A complete solution is characterized by:
	(i) the regime’s five input streams \(u_E,u_H,u_S,s,n\) and the bedtime kernel \(B(t)\);
	(ii) the parameter set \(\theta\) (nominal or one of the synthetic athletes from \S6.2);
	(iii) the state trajectory \(z(t) = (A,N,F_a,F_c,S,I)\) produced by the ODE; and
	(iv) summary decision metrics computed on \(P(t)\), e.g., peak value and timing, mean readiness over a horizon, and simple risk proxies such as time with \(I\) above a threshold.
	
	\paragraph{What we deliver in this section.}
	Because of time constraints we do not recompute long‑term equilibria or eigenstructures for each regime (cf.\ Section~\ref{sec:long-term}, which lays out the qualitative analysis to be done). Instead, we summarize the deterministic runs already reported in \S6.3 (Table~4 and Fig.~2) and organize those findings into practical “solution patterns’’ that a coach can act on. Where a computation would be expected (e.g., an eigenvalue check near a regime’s quasi‑steady weekly pattern), we clearly mark it as future work.
	
	\subsection{Transferability: generalizing the solution to other sports and event goals}
	Our architecture is explicitly \emph{left–center–right}: inputs \(\rightarrow\) six ODE states \(\rightarrow\) readiness. This separation makes the approach portable:
	
	\begin{itemize}
		\item \textbf{Map session types.} Endurance (\(u_E\)), HIIT/anaerobic (\(u_H\)) and strength/plyometrics (\(u_S\)) cover the dominant adaptation pathways across running, cycling, field sports, and court sports. In sport‑specific settings, replace the raw units with a calibrated impulse (e.g., TRIMP or zone minutes for \(u_E\), interval score for \(u_H\), tonnage/contacts for \(u_S\)).
		\item \textbf{Keep the sleep layer intact.} The bedtime kernel \(B(t)\) and sleep opportunity \(s(t)\) are sport‑agnostic; only the kernel weights \((\beta_E,\beta_H,\beta_S)\) and the efficiency slope \(\eta\) need retuning to new populations.
		\item \textbf{Retarget readiness.} Endurance‑major events use \(w_{\text{end}}\gg w_{\text{str}}\); strength‑power events reverse that. Penalties \((\lambda_a,\lambda_c,\lambda_s,\lambda_i)\) can be tuned to the event’s “availability” constraints (e.g., team sports may penalize \(I\) more heavily than \(F_a\)).
		\item \textbf{Weekly invariance.} If a regime repeats weekly, the Poincaré map over 7 days is the natural long‑term object. In future work we will compute the fixed point(s) and their stability for each canonical regime to fully parallel the reference report’s long‑term analysis.
	\end{itemize}
	
	\subsection{Recap of deterministic runs and practical interpretation}\label{subsec:recap-runs}
	Table~4 and Fig.~2 in \S6.3 already illustrate the model’s behavior for representative (athlete, regime) pairs over 21–42 days. We restate and interpret those findings here by mechanism, to make their origin in the ODE transparent:
	
	\begin{description}
		\item[Early vs.\ evening intensity (Nominal).] The early‑AM HIIT and evening‑intensity schedules attained comparable short‑term peaks \(P_{\max}\), but the \emph{mean} readiness was lower for evening sessions. Mechanistically this is the \(q(B)\) penalty: late work increases \(B(t)\), diminishes sleep efficiency, and shrinks the sleep‑accelerated clearance terms in \((F_a),(F_c),(S),(I)\), depressing the envelope on following days. \emph{Expected quantitative check (future):} Compare weekly mean \(P\) across months while holding \(\int(u_E+u_H+u_S)\) fixed; compute the eigenvalues of the 7‑day Poincaré map to confirm stability.
		\item[Midday polarized block (SA‑E).] For an aerobic responder, a high‑volume \(u_E\) with a small midday \(u_H\) gave the largest envelope and highest mean \(P\). Here stronger \(k_A\), milder \(G_{\text{int}}(u_E)\), and small \(B(t)\) combine to raise \(A\) while keeping \(F_c,S\) in check. \emph{Expected quantitative check:} sensitivity of the envelope to \(\mu_E\) (interference) and \(\rho_c\) (sleep‑helped clearance).
		\item[Split day (SA‑S).] AM \(u_E\) + PM \(u_S\) leverages larger \(k_N\) and \(u_S\)-weighting but pays a sleep cost (higher \(B(t)\)): peaks remain strong, while the mean \(P\) is similar to the Nominal early‑AM condition. \emph{Expected check:} how much earlier must the \(u_S\) block move to recoup the mean without sacrificing the \(N\) stimulus?
		\item[Alternating hard/easy (SA‑R).] With tighter recovery gate \(G_{\text{rec}}(S,F_c)\) and slower clearances, the alternating microcycle peaks later and lower. The same SA‑R athlete benefits substantially from a \emph{sleep‑extension + nap} block, which expands \(s(t)\) and thereby multiplies the nightly clearance terms. \emph{Expected check:} quantify the marginal gain in weekly mean \(P\) per extra 30–45 minutes of nightly sleep.
		\item[Taper (Nominal).] A three‑week volume reduction with preserved short \(u_H\) and earlier sessions exhibits the expected rebound: \(F_a\downarrow\to F_c\downarrow\), \(S,I\downarrow\) while \(A,N\) are maintained by saturation. \emph{Expected check:} explore time‑to‑peak as a function of taper half‑life, confirming event‑week planning heuristics.
	\end{description}
	
	\subsection{In‑practice workflow (deterministic use)}
	A practical three‑step workflow emerges from the architecture:
	\begin{enumerate}
		\item \textbf{Pick a parameter set} (Nominal or a synthetic athlete from \S6.2) that best matches the athlete’s phenotype.
		\item \textbf{Encode the candidate training block} as input streams \(u_E,u_H,u_S\), a sleep schedule \(s(t)\) (plus naps), and a lights‑out time for \(B(t)\). For event weeks, scale volume down while keeping short intensity earlier in the day.
		\item \textbf{Simulate and read out \(P(t)\).} Record \(P_{\max}\), time‑to‑peak, mean \(P\), and a risk fraction \(\mathbb{P}[I>I_{\text{hi}}]\). Compare blocks on these summaries and on the mechanism‑aware trends explained above.
	\end{enumerate}
	
	\subsection{Caveats for this section}
	All “expected checks’’ above should be substantiated by (i) computing weekly Poincaré maps under stationary inputs; and (ii) verifying that the nonnegative invariant set defined by the gates is attractive under those inputs.
	

	\section{Discussion}\label{sec:discussion}
	
	\subsection{Strengths}
	\paragraph{Mechanistic transparency.}
	Each term in \((A)\)–\((I)\) has a physiological role (stimulus, gate, saturation, or clearance), obeys sign rules, and preserves positivity without ad hoc clipping. This makes regimes easy to reason about: moving a session later increases \(B\), which reduces \(q(B)\), which shrinks \emph{specific} clearance terms.
	
	\paragraph{Portability and modularity.}
	The left–center–right separation simplifies both transfer to other sports (retargeting \(w_{\cdot}\) and \(\lambda_{\cdot}\)) and code maintenance (inputs and ODE are decoupled). The same six states can be interrogated with different readiness weightings for distinct event profiles.
	
	\paragraph{Deterministic synthetic athletes.}
	Using three coherent parameter regimes (SA‑E, SA‑S, SA‑R) lets us reason about inter‑individual differences without a probabilistic layer. It also surfaces which couplings matter (e.g., \(\mu_E\) for interference; \(c_S,c_c\) for recovery gating).
	
	\subsection{Weaknesses and threats to validity}
	\paragraph{Deterministic parameters.}
	Point estimates ignore uncertainty. In real deployment, parameter posteriors should be estimated from observed logs (Section~\ref{sec:extensions}).
	
	\paragraph{State aggregation.}
	The six states are coarse (e.g., one \(I\) for heterogeneous tissues). This is deliberate for parsimony, but it limits tissue‑specific prescriptions.
	
	\paragraph{Simplified sleep kernel.}
	The bedtime kernel \(B(t)\) is a short exponential memory, not a full circadian model. It captures “how late,” not chronotype, chronotherapy, or sleep fragmentation.
	
	\paragraph{Measurement mappings.}
	Our observation proxies (e.g., field tests for \(A,N\), HRV/sleep for \(S\)) require calibration choices that we did not execute here due to time constraints (see \S\ref{sec:extensions}).
	
	\subsection{Why we did not implement policy optimization}
	The reference report concluded with a normative “utility function \(\rightarrow\) optimizer \(\rightarrow\) policy’’ loop. In our context the direct analogue would be: choose a regime schedule and sleep hygiene policy to maximize time‑averaged \(P\) subject to risk constraints (\(I\le I_{\text{max}}\)) and time budgets. Implementing this credibly requires (i) a calibrated athlete (parameters estimated with uncertainty); (ii) long‑horizon stability checks (weekly Poincaré analysis); and (iii) clear practical constraints. We did not have time to complete those prerequisites. We therefore stop at mechanism‑driven recommendations from deterministic runs and defer the optimization layer.
	

	\section{Conclusion}\label{sec:conclusion}
	
	\paragraph{What we built.}
	We developed a compact six‑state ODE that couples trainable capacities (\(A,N\)), two‑scale fatigue (\(F_a,F_c\)), sleep debt/quality (\(S\)), and micro‑damage (\(I\)) to exogenous inputs for training, sleep, nutrition, and context. A readiness aggregator \(P(t)\) then supports regime comparison, taper design, and simple risk monitoring.
	
	\paragraph{What we learned deterministically.}
	Even without the full stability analysis of Section 7, the deterministic runs in \S6.3 yield coherent, mechanism‑consistent guidance: move key intensity earlier to protect sleep‑accelerated clearance; place high‑volume endurance in midday for aerobic responders; use alternating microcycles and sleep extension for recovery‑limited profiles; and taper by reducing volume, keeping short intensity, and training earlier in event week.
	
	\paragraph{What we did not have time to compute.}
	We regret not executing (i) the long‑term equilibria/eigenvalue program for each regime (the analog of Section~7 in the reference report), and (ii) a normative optimizer that turns model structure into concrete schedule recommendations. Both are feasible next steps.
	
	\subsection{Extensions and next steps}\label{sec:extensions}
	\paragraph{(E1) Long‑term analysis.}
	For weekly repeating regimes, compute the 7‑day Poincaré map, its fixed point(s), and Jacobian eigenvalues; confirm attraction and compare steady‑state envelopes of \(P\).
	
	\paragraph{(E2) Normative regime design.}
	Formulate a constrained optimization that maximizes \(\bar{P}\) (or event‑week \(P_{\max}\)) subject to \(I(t)\le I_{\text{max}}\), session‑time windows, and total weekly impulse. Use simple NLP or dynamic programming with weekly decision variables.
	
	\paragraph{(E3) Bayesian calibration.}
	Replace point estimates by priors for \(\theta\) (time constants, gates, penalties) and estimate posteriors from athlete logs. The observation model can link \(A,N\) to field tests; \(F_a\) to session RPE or neuromuscular decrements; \(S\) to actigraphy/HRV; \(I\) to soreness and time‑loss incidents. Forward simulation remains our ODE; Section~6 can then be reinterpreted as posterior predictive analysis instead of single‑curve runs.
	
	\paragraph{(E4) Real‑world deployment.}
	Run the same program on actual athletes: encode their schedules, collect sleep/HRV and training logs, calibrate \(\theta\), and compare regime variants prospectively.
	

	\section{Appendix: Reproducibility notes and file map}\label{sec:appendix}
	
	\subsection{Code organization}
		
	This is our Github Repository: \url{https://github.com/erikmnovak/i_msquared_c_2025}
	
	Our deterministic runs use a small Julia module with explicit files:
	\begin{itemize}
		\item \texttt{Parameters.jl}: defines \texttt{ModelParams} (nominal values) and three synthetic‑athlete overrides (SA‑E, SA‑S, SA‑R).
		\item \texttt{Inputs.jl}: regime builders that produce the input streams \(u_E,u_H,u_S\), sleep schedule \(s(t)\), bedtime kernel \(B(t)\), and derived sleep efficiency \(q(B)\).
		\item \texttt{ODE.jl}: the right‑hand side \(f(z,u)\) implementing equations \((A)\)–\((I)\) with recovery and interference gates.
		\item \texttt{Readiness.jl}: readiness \(P(t)\) and summary metrics (peak, time‑to‑peak, mean, risk fraction).
		\item \texttt{Simulation.jl}: solver wrapper (Dormand–Prince) that builds inputs, integrates the ODE, computes \(P(t)\), and prints run summaries.
		\item \texttt{AthleteModel.jl}: convenience include, exports, and a single import point for the module.\footnote{We keep time units in days; the bedtime kernel integrates over the last hours before lights‑out and feeds the nightly sleep‑efficiency factor \(q(B)\).}
	\end{itemize}
	
	\subsection{Minimal recipe to reproduce Section 6.3}
	\begin{enumerate}
		\item Load the module, pick an athlete profile (Nominal/SA‑E/SA‑S/SA‑R), and a regime builder (e.g., early‑AM HIIT, evening intensity, polarized midday, split day, alternating hard/easy, taper, sleep‑extension).
		\item Call the solver wrapper with horizon and sampling step (e.g., 42 days, save every hour). Save \((t,P(t))\) and, if desired, the six states.
		\item Plot \(P(t)\) and compute the summary metrics. Use identical plotting limits for apples‑to‑apples comparisons across regime variants.
	\end{enumerate}
	
	\subsection{Units and ranges}
	Capacities \(A,N\in[0,1]\); fatigue \(F_a,F_c\ge0\); sleep debt \(S\ge0\) (larger is worse); micro‑damage \(I\ge0\). Time constants are in days; the bedtime kernel uses hours internally and contributes to daily clearance via \(q(B)\in(0,1]\). Readiness weights \(w_{\cdot}\) and penalties \(\lambda_{\cdot}\) are dimensionless scalars.
				
	
	\begin{thebibliography}{9}\small
		\bibitem{Banister1975} E.\,W. Banister, T.\,W. Calvert, et al., \emph{A systems model of training for athletic performance}, Can.\ J.\ Appl.\ Sport Sci., 1975.
		\bibitem{Busso2003} T. Busso, \emph{Modeling adaptations to training}, Sports Med., 2003.
		\bibitem{Mujika2003} I. Mujika \& S. Padilla, \emph{Scientific bases for precompetition tapering strategies}, Med. Sci. Sports Exerc., 2003.
		\bibitem{Hickson1980} R. Hickson, \emph{Interference of strength development by simultaneously training for strength and endurance}, Eur. J. Appl. Physiol., 1980.
		\bibitem{Fullagar2015} H. Fullagar et al., \emph{Sleep and athletic performance}, Sports Med., 2015.
		\bibitem{Buchheit2014} M. Buchheit, \emph{Monitoring training status with HR measures}, Sports Med., 2014.
		\bibitem{Gabbett2016} T. Gabbett, \emph{The training--injury prevention paradox: load, risk and performance}, Br. J. Sports Med., 2016.
		\bibitem{Skiba2012} P. Skiba et al., \emph{Modeling the expenditure and reconstitution of work capacity above critical power}, Med. Sci. Sports Exerc., 2012.
	\end{thebibliography}
	
	\newpage
	
	\section{Speech to the NABC}
	
	Good evening everyone, and thank you for the chance to share what our team has been working on.
	
	At its heart, our project asks a simple question: How do training, sleep, and recovery come together to make an athlete ready to perform—without courting injury? We built a compact, coach‑friendly model that treats training and recovery like connected dials. When you turn one, the others move. In plain terms, our model takes in a weekly schedule—what kind of training, when you do it, how you sleep, how well you fuel, and what outside stresses you’re under—and it predicts a day‑by‑day “readiness” curve: higher when the body is primed, lower when fatigue or risk is building. Our full write‑up shows how this plays out across common schedules such as early‑morning HIIT, evening intensity, split days, alternating hard/easy weeks, sleep‑extension blocks, and tapers.
	
	Here are the four biggest takeaways for coaches, athletes, parents, and program leaders:
	
	1. Timing matters—especially at night. The same hard session done late in the evening tends to blunt that night’s sleep quality and pushes down next‑day readiness. When possible, put key intensity earlier in the day, and protect “lights‑out” time so sleep can do its job.
	
	2. Different athletes, different wins. We tested three realistic “profiles”: an endurance‑leaning responder, a power/neuromuscular responder, and a recovery‑limited athlete. The best average readiness came from aligning the block with the athlete: e.g., midday polarized endurance for the aerobic responder, split‑day with PM strength (used sparingly) for the power responder, and sleep‑extension plus a short early‑afternoon nap for the recovery‑limited athlete.
	
	3. Plan recovery like you plan workouts. Readiness climbs not just with training, but with good sleep and steady nutrition that speed up repair. Short tapers that reduce volume while keeping brief intensity reliably nudge readiness to a peak about a week into the taper window—useful for meet and race planning.
	
	4. Track a few simple signals. You don’t need a lab. Keep a running log of: session type and time of day, a 1–10 fatigue or soreness check‑in, bedtime and sleep duration, and a quick note on life stress. Those four lines of data help you spot when to move a hard session earlier, swap in a recovery day, or add a nap policy during heavy blocks.
	
	What can the community do now?
	
	Schools and clubs: Encourage earlier start times for key intensity sessions when logistics allow; make “quiet hours” before bedtime part of team culture during heavy training.
	
	Coaches: Build microcycles that breathe—alternate hard and easy days, and use sleep‑extension blocks before a big push.
	
	Parents and athletes: Treat sleep as training. A consistent wind‑down, cool/dark room, and realistic homework windows help as much as a new pair of spikes.
	
	Organizers: For travel or heat stress weeks, lower evening intensity and plan team naps or quiet rest periods after lunch.
	
	Two final notes. First, we designed this to be sport‑agnostic and practical; the math lives under the hood so coaches can focus on schedules and habits. Second, we’re transparent about limitations: we did not have time to run a full stability analysis or policy optimization this cycle, and we used synthetic athlete profiles rather than large in‑season datasets. Next steps are clear: pilot the model with real teams, and—when time allows—add a Bayesian layer so the parameters personalize themselves to each athlete’s history. Our full report lays out those steps, and we invite interested programs to collaborate.
	
	Thank you. And, here’s to smarter training, better sleep, and healthier seasons.
	
	\newpage
	

	
	\section{AI Usage report}
	
	On principle, I avoided using AI for any actual writing or model creation. However, being a solo participant, I had to use AI on these more repetitive parts of creating the report:
	\begin{enumerate}
		\item Creating most tables.
		\item Writing the skeleton for the code.
		\item Debugging the code.
		\item Creating the pgfplot in section 6.
		\item Creating figure 1.
		\item Searching the literature for relevant results.
	\end{enumerate}
	These were necessary to ensure a clean presentation and that I had a sufficiently ready delivery.
	
	
\end{document}