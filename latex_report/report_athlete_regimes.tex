\documentclass[]{article}

% PACKAGES

% correct encoding and typography
\usepackage[english]{babel}
\usepackage[utf8]{inputenc}
\usepackage[T1]{fontenc}

% math
\usepackage{amsfonts, latexsym, mathtools, amsthm, amssymb, amscd}
\usepackage{euscript}
\usepackage{esvect}
\usepackage{bm}
\usepackage{physics}

% aesthetics color options
\usepackage{graphicx}
\usepackage[dvipsnames]{xcolor}                           

% correct encoding and typography
\usepackage[english]{babel}
\usepackage[utf8]{inputenc}
\usepackage[T1]{fontenc}

\usepackage{tikz}
\usetikzlibrary{arrows.meta,positioning,calc, matrix}


% math
\usepackage{amsfonts, latexsym, mathtools, amsthm, amssymb, amscd}
\usepackage{euscript}
\usepackage{esvect}
\usepackage{bm}
\usepackage{physics}

% aesthetics color options
\usepackage{graphicx}
\usepackage{pdfpages}
\usepackage{microtype}

% to access the last page
\usepackage{lastpage}

% better document formatting
\usepackage{fancyhdr}
\usepackage[a4paper, right=1in, left=1in, bottom=1.2in, top=1.2in, centering]{geometry}
\usepackage{fullpage}

\usepackage{datetime2}

% better stylistic formatting
\usepackage{parskip}
\usepackage{enumitem}
\usepackage{framed}
\usepackage{longtable}
\usepackage{csquotes}

% embedding and referencing
\usepackage{hyperref}
\hypersetup{colorlinks=true,citecolor=blue,urlcolor =black,linkbordercolor={1 0 0}}

\usepackage{booktabs,array,longtable}


%%%%%%%%%%%%%%%%%%%%%%%%%%%%%%%%%%%%%%%%%%%%%%%%%%%%%%%%%

%%%%%%%%%%%%%%%%%%%%%%%%%%%%%%%%%%%%%%%%%%%%%%%%%%%%%%%%%
% Command formatting for ease-of-life

% new math symbols taking no arguments
\newcommand{\minus}{\smallsetminus}
\newcommand{\dotp}{\boldsymbol{\cdot}}

%redefined math symbols taking no arguments
\newcommand{\<}{\langle}
\renewcommand{\>}{\rangle}
\renewcommand{\iff}{\Leftrightarrow}
\renewcommand{\implies}{\Rightarrow}

% Arrows
\newcommand{\from}{\leftarrow}
\newcommand{\ffrom}{\longleftarrow}
\newcommand{\goesto}{\rightsquigarrow}
\newcommand{\into}{\hookrightarrow}

% Basic bbface (blackboard bold) lettering command:
\newcommand{\bbb}[1]{\ensuremath{\mathbb{#1}}}

% Boldface lettering command used for vectors:
\newcommand{\bvec}[1]{\ensuremath{\mathbf{#1}}}

\newcommand{\bN}{\bbb{N}}
\newcommand{\bR}{\bbb{R}}
\newcommand{\bZ}{\bbb{Z}}
\newcommand{\bH}{\bbb{H}}
\newcommand{\oo}{\ensuremath{\emptyset}}
\newcommand{\bQ}{\bbb{Q}}
\newcommand{\bC}{\bbb{C}}
\newcommand{\bF}{\bbb{F}}
\newcommand{\bI}{\bbb{I}}
\newcommand{\bP}{\bbb{P}}
\newcommand{\bE}{\bbb{E}}
\newcommand{\bT}{\bbb{T}}


\newcommand{\vx}{\bvec{x}}
\newcommand{\vy}{\bvec{y}}
\newcommand{\vuu}{\bvec{u}}
\newcommand{\vvv}{\bvec{v}}
\newcommand{\vw}{\bvec{w}}
\newcommand{\vaa}{\bvec{a}}
\newcommand{\vbb}{\bvec{b}}
\newcommand{\vo}{\bvec{o}}
\newcommand{\ve}{\bvec{e}}
\newcommand{\vp}{\bvec{p}}
\newcommand{\vi}{\bvec{i}}
\newcommand{\vz}{\bvec{z}}
\newcommand{\vt}{\bvec{t}}


% Basic cal lettering
\newcommand{\ccc}[1]{\ensuremath{\mathcal{#1}}}

\newcommand{\cE}{\ccc{E}}
\newcommand{\cM}{\ccc{M}}
\newcommand{\cP}{\ccc{P}}
\newcommand{\cV}{\ccc{V}}
\newcommand{\cO}{\ccc{O}}


% Basic frak lettering
\newcommand{\fff}[1]{\ensuremath{\mathfrak{#1}}}

\newcommand{\mm}{\mathfrak{m}}
\newcommand{\pp}{\mathfrak{p}}

% Other lettering
\newcommand{\GL}{\mathit{GL}}
\newcommand{\FL}{{\mathcal{} F}{\ell}_n}
\newcommand{\SO}{\mathit{SO}}
\newcommand{\eps}{\varepsilon}
\newcommand{\ind}{\mathbf{1}}

% some operators and delimiters
\DeclareMathOperator{\proj}{proj}
\DeclareMathOperator{\im}{im}
\DeclareMathOperator{\coker}{coker}
\DeclareMathOperator{\coim}{coim}
\DeclareMathOperator{\Span}{Span}
\DeclareMathOperator{\Stab}{Stab}
\DeclareMathOperator{\Orb}{Orb}
\DeclareMathOperator{\EE}{E}
\DeclareMathOperator{\Fun}{Fun}
\DeclareMathOperator{\CSet}{Set}
\DeclareMathOperator{\Cmap}{map}
\DeclareMathOperator{\colim}{colim}
\DeclareMathOperator{\Hom}{Hom}
\DeclareMathOperator{\res}{res}
\DeclareMathOperator{\Spec}{Spec}
\DeclareMathOperator{\Int}{int}

% Other more involved shortcuts

%for overline math
\newcommand{\ol}[1]{{\overline{#1}}}

%redefined math symbols taking arguments
\renewcommand{\mod}[1]{\ (\mathrm{mod}\ #1)}

%for easy 2 x 2 matrices
\newcommand{\twobytwo}[1]{\left[\begin{array}{@{}cc@{}}#1\end{array}\right]}

%for easy column vectors of size 2
\newcommand{\tworow}[1]{\left[\begin{array}{@{}c@{}}#1\end{array}\right]}

%%%%%%%%%%%%%%%%%%%%%%%%%%%%%%%%%%%%%%%%%%%%%%%%%%%%%%%%%

%%%%%%%%%%%%%%%%%%%%%%%%%%%%%%%%%%%%%%%%%%%%%%%%%%%%%%%%%
%Below are the theorem, definition, example, lemma, etc. body types.

\newtheorem{theorem}{Theorem}[section]
\newtheorem{proposition}[theorem]{Proposition}
\newtheorem{lemma}[theorem]{Lemma}
\newtheorem{corollary}[theorem]{Corollary}
\theoremstyle{definition}
\newtheorem{definition}[theorem]{Definition}
\newtheorem{assumption}[theorem]{Assumption}
\newtheorem{remark}[theorem]{Remark}
\newtheorem{example}[theorem]{Example}


%%%%%%%%%%%%%%%%%%%%%%%%%%%%%%%%%%%%%%%%%%%%%%%%%%%%%%%%%
%Some formatting tidbits for margins, paragraphs, and removing orphans

% make widows and orphans rare
\clubpenalty =10000
\widowpenalty =10000

% formatting of paragraphs and separation
\setlength{\parindent}{0pt}
\setlength{\parskip}{5pt plus 1pt}
\setlength{\headheight}{13.6pt}

\setlength\marginparwidth{2.2in}
\setlength\marginparsep{1mm}

%%%%%%%%%%%%%%%%%%%%%%%%%%%%%%%%%%%%%%%%%%%%%%%%%%%%%%%%%

% Styles and colors
\tikzset{
	state/.style   ={draw,rounded corners,thick,align=center,
		minimum width=3.2cm,minimum height=1.15cm,fill=white},
	input/.style   ={draw,rounded corners,thick,dashed,align=center,
		minimum width=3.2cm,minimum height=1.0cm,fill=gray!10},
	derived/.style ={draw,rounded corners,thick,densely dashed,align=center,
		minimum width=3.2cm,minimum height=1.0cm,fill=gray!5},
	edgeplus/.style={-Stealth,very thick,draw=ForestGreen,shorten >=2pt,shorten <=2pt},
	edgeminus/.style={-Stealth,very thick,draw=BrickRed,shorten >=2pt,shorten <=2pt},
	% If you *do* want tiny '+'/'−' labels back on edges, uncomment:
	% lab/.style={font=\scriptsize,fill=white,inner sep=1pt}
}

%%%%%%%%%%%%%%%%%%%%%%%%%%%%%%%%%%%%%%%%%%%%%%%%%%%%%%%%%


\title{\vspace{-1.0em}A Coupled Six--State Athlete Model for Training, Sleep, Recovery, and Risk\\
	\large (Sections 1--3: Introduction, Overview, Assumptions)}
\author{}
\date{}

\begin{document}

	\maketitle
	\vspace{-2.0em}
	
	\tableofcontents
	
	\newpage
	
	\section{Introduction}
	Athletic performance emerges from the three-pronged tug-of-war between \emph{training stimulus}, \emph{recovery}, and \emph{risk}. 
	
	Our goal is to build a compact, mechanistic model that can evaluate \emph{training--rest regimes} and answer operational questions such as: When does a given microcycle peak performance? How costly is late--evening high intensity on next--day readiness? What taper length best converts prior load into performance at a target event while respecting injury risk?
	
	\paragraph{Problem framing.}
	The problem statement proposes comparing qualitatively distinct regimes (e.g., high--intensity early vs.\ late sessions, split sessions vs.\ single sessions, alternating hard/easy days, dedicated recovery or taper periods). We formalize these as exogenous, time--varying input functions for training, sleep, and context (stress, nutrition), then follow their consequences through a system of coupled ODE states. The model is designed to be \emph{interpretable}, \emph{calibratable on athlete logs}, and \emph{portable} across sports.
	
	\paragraph{Design philosophy and precedent.}
	We draw on established ideas from training--response modeling (fitness--fatigue/impulse--response), tapering, concurrent training interactions, sleep effects on performance, and load--related injury risk \cite{Banister1975,Busso2003,Mujika2003,Hickson1980,Fullagar2015,Buchheit2014,Gabbett2016,Skiba2012}.
	
	
	\section{Brief overview of our dynamic model}
	\label{sec:model-overview}
	
	\subsection*{System architecture and figure}
	Our system is organized as a pipeline with three layers:  
	\emph{(i) exogenous inputs} that the coach/athlete controls (left),  
	\emph{(ii) a six-state ODE core} capturing trainable capacity, fatigue, sleep and risk (center), and  
	\emph{(iii) a derived readiness/output} for decision-making (right). The signed wiring diagram in
	Figure~\ref{fig:overview-wiring} makes these couplings explicit: \textcolor{ForestGreen}{green} = positive effect; \textcolor{BrickRed}{red} = negative effect.
	
	\begin{figure}[h!]
		\centering
		\includegraphics[width=\linewidth]{coupling_diagram_1.png}
		\caption{Left–center–right architecture. Left: five exogenous inputs; Center: six ODE states grouped by row; Right: readiness $P(t)$. Colors denote effect signs.}
		\label{fig:overview-wiring}
	\end{figure}

	
	\subsection*{Left column: exogenous inputs (what the coach controls)}
	Each input is a bounded, time-varying control signal. We keep units flexible and normalize when needed for calibration.
	\begin{description}[leftmargin=1.7em]
		\item[\textbf{Training composition \boldmath$u_E(t),u_H(t),u_S(t)$}.] Session intensity/volume streams for endurance (e.g., Zone time or TRIMP), high‑intensity/anaerobic work (intervals/HIIT), and strength/plyometrics (e.g., tonnage or explosive contacts). These are the \emph{primary} stimuli for adaptation and the main drivers of acute fatigue and micro‑damage \cite{Banister1975,Busso2003,Hickson1980}.
		\item[\textbf{Bedtime proximity kernel \boldmath$B(t)$}.] A short memory of training near lights‑out that reduces sleep efficiency that night \cite{Fullagar2015}. Operationally, $B(t)$ will later be computed by convolving recent intensity with an exponentially decaying kernel that weights the final hours before bedtime more heavily.
		\item[\textbf{Sleep schedule \boldmath$s(t)\!\in\![0,1]$}.] An on/off indicator of sleep opportunity (night sleep and optional naps). During $s(t)=1$ the model pays down fatigue/sleep debt and accelerates tissue repair.
		\item[\textbf{Nutrition/availability \boldmath$n(t)$}.] A compact proxy for energy/protein availability and timing (e.g., carbohydrate after HIIT, protein after strength). We use it to gate remodeling and reduce damage accumulation during sleep and rest.
		\item[\textbf{Context stress \boldmath$x(t)$}.] Non‑training stressors (travel, exams, heat, life stress). This input increases central load and impairs sleep quality; it is an external “tax” on recovery \cite{Buchheit2014}.
	\end{description}

	\subsection*{Center: the six‑state ODE core (what the system does)}
	The central panel contains six dynamical states grouped by theme (rows). We postpone explicit forms until Section~4; here we state what each encodes and how it is observed.
	
	\paragraph{Top row — Trainable capacities.}
	\begin{description}[leftmargin=1.7em]
		\item[\textbf{Aerobic adaptation \boldmath$A(t)$}.] Normalized $(0\!-\!1)$ “engine” for endurance performance (e.g., \%~$\dot{V}\!O_2$ improvements, time‑to‑exhaustion). Stimulated mainly by $u_E$ and $u_H$ with diminishing returns; gated by recovery. Proxies: best‑effort curves, critical‑power modeling, heart‑rate kinetics \cite{Banister1975,Skiba2012}.
		\item[\textbf{Neuromuscular/strength adaptation \boldmath$N(t)$}.] Normalized $(0\!-\!1)$ capacity for force/power (e.g., 1RM, CMJ, sprint split). Stimulated by $u_S$ and partly by $u_H$; subject to endurance–strength interference when $u_E$ is high \cite{Hickson1980}. Also gated by recovery.
	\end{description}
	
	\paragraph{Middle row — Fatigue (two time scales).}
	\begin{description}[leftmargin=1.7em]
		\item[\textbf{Acute fatigue \boldmath$F_a(t)$}.] Fast time scale (hours–days). Rises with session load ($u_E,u_H,u_S$), clears quickly (especially during sleep).
		\item[\textbf{Chronic fatigue \boldmath$F_c(t)$}.] Slow time scale (days–weeks). Accumulates when $F_a$ is repeatedly unresolved (monotony), clears slowly with sustained good sleep and lighter training \cite{Busso2003}.
	\end{description}
	
	\paragraph{Bottom row — Sleep and tissue risk.}
	\begin{description}[leftmargin=1.7em]
		\item[\textbf{Sleep debt / quality \boldmath$S(t)$}.] Larger $S$ means worse cumulative sleep state (more debt/lower quality). Increases while awake and after heavy training; decreases during sleep with an efficiency reduced by $B(t)$ \cite{Fullagar2015}.
		\item[\textbf{Injury micro‑damage / hazard \boldmath$I(t)$}.] A continuous proxy for tissue stress/inflammation (not a discrete injury). Rises with high‑impact/HIIT/strength loading and with fatigue‑mediated poor mechanics; falls with time, sleep, and nutrition \cite{Gabbett2016}.
	\end{description}

	\subsection*{Right column: derived readiness/output (what we optimize)}
	\textbf{Readiness \boldmath$P(t)$} aggregates sport‑specific performance potential from the states above. We use $P(t)$ to compare regimes, design tapers, and schedule recovery days; detailed forms appear in Section~4.
	
	\subsection*{Coupling map (how pieces talk)}
	The directed arrows in Figure~\ref{fig:overview-wiring} implement the following sign‑rules and qualitative nonlinearities:
	\begin{enumerate}[label=\textbf{C\arabic*}. ,leftmargin=1.75em]
		\item \textbf{Training stimulates capacity} (\textcolor{ForestGreen}{+}): $u_E,u_H \!\to\! A$; $u_S,u_H \!\to\! N$ with saturating gains and \emph{diminishing returns}. High $u_E$ mildly interferes with $N$ (concurrent‑training effect) \cite{Hickson1980}.
		\item \textbf{Training creates load} (\textcolor{ForestGreen}{+}): all $u_\cdot \!\to\! F_a$ and, via accumulation, $F_a\!\to\!F_c$.
		\item \textbf{Load creates micro‑damage} (\textcolor{ForestGreen}{+}): $u_\cdot$, $F_a$, and $F_c$ raise $I$ (mechanical + metabolic + poor‑mechanics channels).
		\item \textbf{Sleep repairs} (\textcolor{ForestGreen}{+} into recovery, \textcolor{BrickRed}{-} into debts): $s(t)$ \emph{reduces} $S$, $F_a$, $F_c$ and $I$. But late training worsens that repair: larger $B(t)$ \emph{reduces} the sleep‑driven clearance of $S$, $F_a$, $F_c$, $I$ \cite{Fullagar2015}.
		\item \textbf{Sleep debt throttles adaptation} (\textcolor{BrickRed}{-}): larger $S$ suppresses gains in $A,N$ and increases $F_a,F_c$ (more wakefulness/poorer sleep $\Rightarrow$ higher perceived load) \cite{Busso2003,Fullagar2015}.
		\item \textbf{Micro‑damage suppresses adaptation} (\textcolor{BrickRed}{-}): high $I$ reduces realized gains in $A,N$ and contributes to readiness penalties \cite{Gabbett2016}.
		\item \textbf{Context stress loads the system} (\textcolor{ForestGreen}{+} into $F_c,S$): travel/heat/psychological load raises central fatigue and impairs sleep quality \cite{Buchheit2014}.
		\item \textbf{Nutrition improves remodeling} (\textcolor{BrickRed}{-} into $I$): adequate energy/protein reduces tissue damage and speeds clearance.
		\item \textbf{Readiness aggregation}: $A,N$ contribute positively; $F_a,F_c,S,I$ subtract with task‑specific weights.
	\end{enumerate}
	All couplings will be implemented with smooth, saturating response functions to ensure state positivity and realistic ceilings \cite{Busso2003}.
	
	\subsection*{Regimes as inputs (how we encode the schedules)}
	We represent regimes by specifying the shapes and timing of the five input streams. Below are canonical examples we will test, using the exact left‑column elements of Figure~\ref{fig:overview-wiring}.
	
	\paragraph{R1 — Early‑morning HIIT (7–9 AM), no nap.}
	\emph{Encoding:} A short $u_H$ pulse near wake time; low $B(t)$; $s(t)$ is one nightly block.
	\emph{Expected signature:} Strong $N$ and $A$ stimulus; modest $F_a$ spike; minimal impact on that night’s sleep; next‑day $P$ depends on preceding night’s $S$.
	
	\paragraph{R2 — Evening moderate/high intensity (7–9 PM).}
	\emph{Encoding:} $u_E$ or $u_H$ pulse ending near lights-out $\Rightarrow$ large $B(t)$; standard $s(t)$.  
	\emph{Expected signature:} Reduced sleep‑efficiency that night (slower decay of $S,F_a,F_c,I$); next‑day $P$ depressed; cumulative late‑evening sessions elevate chronic load.
	
	\paragraph{R3 — Split session (light AM endurance + PM strength).}
	\emph{Encoding:} Small morning $u_E$ pulse; larger afternoon/evening $u_S$ pulse raising $B(t)$.  
	\emph{Expected signature:} Good $N$ gains with some interference from $u_E$; higher $I$ and $S$ on PM‑strength days; performance trade‑off between power gains and sleep.
	
	\paragraph{R4 — Alternating days (hard/easy microcycle).}
	\emph{Encoding:} Hard day: large $u_\cdot$ pulses; Easy day: minimal $u_\cdot$, $s(t)$ may include a nap; $B(t)$ small on easy days.  
	\emph{Expected signature:} $F_a$ rises on hard days then decays; $F_c$ stabilizes or falls; $I$ accumulates more slowly; $P$ shows saw‑tooth with higher weekly average.
	
	\paragraph{R5 — Midday training (1–3 PM).}
	\emph{Encoding:} $u_E$ or mixed session far from bedtime $\Rightarrow$ small $B(t)$.  
	\emph{Expected signature:} Balanced load–recovery; relatively low $S$; favorable steady‑state $P$ with low $I$ accrual.
	
	\paragraph{R6 — Taper into event week (volume down, intensity maintained).}
	\emph{Encoding:} Multiply $u_E,u_S$ volumes by a decaying factor; maintain short $u_H$ stimuli; enforce early‑day sessions to keep $B(t)$ small.  
	\emph{Expected signature:} $F_a\downarrow$, then $F_c\downarrow$; $S$ improves; $A,N$ maintained; $I$ decays; peak in $P$ near event \cite{Mujika2003}.
	
	\paragraph{R7 — Sleep‑extension and nap policy.}
	\emph{Encoding:} Increase nightly $s(t)$ duration and add a short post‑lunch nap block; enforce low‑$B(t)$ by moving $u_\cdot$ earlier.  
	\emph{Expected signature:} Faster clearance of $F_a,F_c,I$; sustained reduction in $S$; higher readiness envelope for the same weekly load \cite{Fullagar2015}.
	
	\paragraph{R8 — High‑volume polarized vs.\ pyramidal endurance blocks.}
	\emph{Encoding:} Shift weight among $u_E$ (easy volume) and $u_H$ (interval density) with identical weekly “TRIMP”.  
	\emph{Expected signature:} Comparable $A$ gains but different $F_a,S$ trajectories; polarized blocks target higher $P$ with lower $I$ at the same load.
	
	\medskip
	These regime encodings are \emph{inputs only}; the explicit ODEs that transform them into state trajectories will be written in Section~\ref{sec:model-development}. Our analysis will compare regimes by their steady‑state $P$ envelopes, peaks, time‑to‑peak, and risk measures (e.g., time above $I$ thresholds).
	
	
	
	\section{Assumptions}
	We separate assumptions into: (i) global modeling assumptions; (ii) regime/input assumptions; and (iii) state--specific assumptions that will directly inform Section~4 when we write the ODEs.
	
	\subsection{Global modeling assumptions}
	\begin{enumerate}[label=\textbf{G\arabic*.}]
		\item \textbf{Single ``well--mixed'' athlete:} we model one athlete as a single dynamical unit; tissue and organ micro--heterogeneity are absorbed into parameters.
		\item \textbf{Time scales:} processes evolve on hours--to--weeks; we do not include circannual or multi--year remodeling here.
		\item \textbf{Non--dimensionalization:} states (\(A,N,S,I\)) are scaled to \([0,1]\); \(F_a,F_c\) are nonnegative with practical upper bounds from data.
		\item \textbf{Regularity:} inputs \(u_E,u_H,u_S,s,n,x\) are piecewise continuous and bounded; regime switches are scheduled or threshold--triggered (Section~4).
		\item \textbf{Saturations:} all response functions are monotone and saturating (e.g., Hill/Michaelis--Menten--like) to enforce physiological ceilings and diminishing returns \cite{Busso2003,Mujika2003}.
		\item \textbf{Positivity and invariance:} the ODE right--hand sides are constructed to keep physically meaningful ranges invariant (no negative sleep debt or negative injury, etc.).
		\item \textbf{No explicit delays (first pass):} distributed training effects are approximated by multiple time scales (acute \(\to\) chronic) rather than explicit delay differential equations \cite{Busso2003}.
		\item \textbf{Observables:} we map proxies to states for calibration: critical power/\(W'\) or best efforts to \(A\), jump/1RM surrogates to \(N\), session RPE and neuromuscular decrements to \(F_a\), HRV/sleep metrics to \(R/S\), soreness/incident logs to \(I\) \cite{Buchheit2014,Skiba2012,Fullagar2015}.
		\item \textbf{Noise and shocks:} stochastic shocks (illness, travel) are represented through \(x(t)\); we neglect process noise in the first pass.
	\end{enumerate}
	
	\subsection{Regime and input assumptions}
	\begin{enumerate}[label=\textbf{R\arabic*.}]
		\item \textbf{Training decomposition:} total load is \(u(t)=u_E(t)+u_H(t)+u_S(t)\); each component differs in \emph{how} it stimulates capacity vs.\ damage and in energy cost \cite{Banister1975,Hickson1980}.
		\item \textbf{Sleep window:} \(s(t)=1\) during scheduled sleep (including naps); nightly sleep efficiency is reduced by a \emph{bedtime--proximity} kernel \(B(t)\) that integrates training intensity close to bedtime.
		\item \textbf{Nutrition simplification:} \(n(t)\) represents energy/protein availability; we will later let \(n(t)\) gate recovery and reduce damage accrual.
		\item \textbf{Context stress:} \(x(t)\) aggregates non--training stressors; it increases fatigue and sleep debt and (weakly) raises micro--damage (e.g., travel).
		\item \textbf{Hybrid switching (optional):} regimes are either prescribed on a calendar or triggered by internal thresholds, e.g., if a hazard score from \(F_a,F_c,S,I\) exceeds a limit, switch to recovery.
	\end{enumerate}
	
	\subsection{State--specific assumptions (to guide the ODE forms later)}
	\paragraph{Aerobic adaptation \(A(t)\).}
	\begin{enumerate}[label=\textbf{A\arabic*.}]
		\item Stimulated primarily by \(u_E\) and \(u_H\); the effect is saturating and subject to diminishing returns.
		\item Gains are \emph{gated} by recovery: high \(S\) (poor sleep) and high \(F_c\) reduce effective adaptation \cite{Busso2003,Fullagar2015}.
		\item Detrains slowly toward a baseline in the absence of stimulus.
		\item Elevated damage \(I\) suppresses realized gains (e.g., protective downregulation) \emph{and} can transiently impede training quality.
	\end{enumerate}
	
	\paragraph{Neuromuscular adaptation \(N(t)\).}
	\begin{enumerate}[label=\textbf{N\arabic*.}]
		\item Stimulated by \(u_S\) and, secondarily, by \(u_H\) (shared neuromuscular stress).
		\item Endurance load \(u_E\) causes a modest \emph{interference} with strength/power gains (modeled later as a damping factor) \cite{Hickson1980}.
		\item Gains are gated by \(S\) and \(F_c\) (poor sleep/central fatigue slow synthesis and motor learning).
		\item Detrains with a time constant distinct from \(A\) (typically faster).
		\item Elevated \(I\) directly suppresses \(N\) gains (pain/inflammation limiting heavy work).
	\end{enumerate}
	
	\paragraph{Acute fatigue \(F_a(t)\).}
	\begin{enumerate}[label=\textbf{F\arabic*.}]
		\item Increases with all training components; intensity--heavy work contributes disproportionately \((u_H,u_S)\).
		\item Clears quickly with time and \emph{faster} under good sleep (\(s(t)\)) and good recovery state.
		\item Low energy/nutrition (via \(n(t)\)) and high \(S\) blunt clearance.
	\end{enumerate}
	
	\paragraph{Chronic fatigue \(F_c(t)\).}
	\begin{enumerate}[label=\textbf{C\arabic*.}]
		\item Accumulates from unresolved \(F_a\) (low--pass filtered fatigue).
		\item Clears slowly with time and sleep; sensitive to monotony/psychological factors (absorbed into \(x(t)\)).
		\item High \(F_c\) gates down \(A\) and \(N\) gains \cite{Busso2003}.
	\end{enumerate}
	
	\paragraph{Sleep debt / quality \(S(t)\).}
	\begin{enumerate}[label=\textbf{S\arabic*.}]
		\item Accumulates while awake and with strenuous training days (via arousal and thermoregulation burdens).
		\item Decreases during sleep; the nightly paydown is reduced when \(B(t)\) is large (late training) \cite{Fullagar2015}.
		\item Higher \(S\) raises \(F_a\) and \(F_c\) (worse sleep \(\Rightarrow\) more fatigue) and gates down capacity gains.
	\end{enumerate}
	
	\paragraph{Injury micro--damage \(I(t)\).}
	\begin{enumerate}[label=\textbf{I\arabic*.}]
		\item Increases with mechanical/metabolic stress, particularly \(u_S\) and high--intensity efforts \(u_H\).
		\item Accrual is amplified by high \(F_a\) or \(F_c\) (poor mechanics, compromised tissue resilience).
		\item Clears with time, sleep, and adequate nutrition \(n(t)\) (remodeling).
		\item A (soft) hazard from \(I\) contributes to regime switching/guard conditions in Section~4 and to performance penalties \cite{Gabbett2016}.
	\end{enumerate}
	
	\paragraph{Derived outputs (for later use).}
	We will define sport--specific \emph{readiness} signals \(P_{\text{end}}\) and \(P_{\text{str}}\) as functions of \(A,N,F_a,F_c,S,I\), to be used for evaluation and optimization in later sections.
	
	\section{Model development: a six--state ODE with exogenous controls}
	\label{sec:model-development}
	
	In line with the wiring diagram in Figure~\ref{fig:overview-wiring}, we now move from concepts to a concrete
	six--equation dynamical system driven by five exogenous inputs (training, bedtime proximity, sleep
	opportunity, nutrition, and context stress). The model is intentionally \emph{mechanistic but light}:
	each term has a physiological interpretation, obeys sign rules from Section~\ref{sec:model-overview},
	and uses smooth saturating (logistic/Hill) responses so states remain in meaningful ranges
	(\emph{positivity} and \emph{boundedness}). We keep the time unit as \textbf{days}; sub–day effects (e.g., a 
	late session) enter via the bedtime kernel.
	
	Throughout, we adopt the training–response perspective pioneered by Banister and co‑authors and
	expanded by many others \cite{Banister1975,Busso2003,Skiba2012,Mujika2003}, the interference
	literature for concurrent strength and endurance \cite{Hickson1980}, sleep–performance links \cite{Fullagar2015,Buchheit2014}, and
	load–injury risk ideas \cite{Gabbett2016}.
	
	\subsection{Preliminaries: inputs, normalizations, and helper functions}
	
	\paragraph{Inputs (left column of Fig.~\ref{fig:overview-wiring}).}
	\begin{itemize}
		\item \(u_E(t),u_H(t),u_S(t)\) --- endurance, HIIT/anaerobic, and strength/plyometric session intensity--volume
		streams (e.g., zone minutes/TRIMP; interval load; tonnage or contact count). They are bounded and piecewise
		continuous. We use \(\|u\|_\text{day}=\int_t^{t+1}(u_E+u_H+u_S)\,d\tau\) as the day’s gross load.
		\item \(B(t)\) --- the \textbf{bedtime proximity kernel}. It compresses “how late” training occurred into a scalar
		that modulates that night’s sleep quality. We define
		\[
		B(t)=\int_{-\infty}^t \big(\beta_E u_E(\tau)+\beta_H u_H(\tau)+\beta_S u_S(\tau)\big)\,K_b\!\big(t-\tau\big)\,d\tau,
		\]
		where \(K_b(\Delta)=\exp(-\Delta/\sigma_b)\mathbf{1}_{0<\Delta\le h_b}\) weights the last \(h_b\) hours before lights--out;
		\(\sigma_b\) is a decay constant. Physiologically: high intensity close to bedtime leaves more arousal/heat, impairing
		early sleep \cite{Fullagar2015}.
		\item \(s(t)\in[0,1]\) --- \textbf{sleep opportunity}; \(s=1\) during night sleep or planned naps, else \(0\).
		\item \(n(t)\in[0,1]\) --- \textbf{nutrition/availability}; a compact proxy for energy/protein availability and timing
		(e.g., carbohydrate after HIIT; protein after strength); higher \(n\) improves repair.
		\item \(x(t)\ge0\) --- \textbf{context stress} (travel/heat/psychological load); larger \(x\) adds central load and
		impairs sleep quality \cite{Buchheit2014}.
	\end{itemize}
	
	\paragraph{Gates and saturations.}
	We repeatedly use three small building blocks in our equations:
	
	\begin{align}
		\textit{(i) Recovery gate:}\quad & G_\mathrm{rec}(S,F_c)=\frac{1}{1+c_S S+c_c F_c}\in(0,1],                                            \label{eq:recgate}\\
		\textit{(ii) Interference gate:}\quad & G_\mathrm{int}(u_E)=\frac{1}{1+\mu_E\,u_E}\in(0,1],                                             \label{eq:intgate}\\
		\textit{(iii) Sleep efficiency:}\quad & q(B)=\frac{q_0}{1+\eta\,B(t)}\in(0,q_0], \qquad \eta\ge0, \; 0<q_0\le1.                          \label{eq:sleepeff}
	\end{align}
	
	The \emph{recovery gate} \eqref{eq:recgate} (so named to emphasize that poor sleep/central fatigue throttle supercompensation) embodies the empirically supported idea that poor sleep and
	unresolved central fatigue throttle adaptation \cite{Busso2003,Fullagar2015}. 
	
	The \emph{interference gate}
	\eqref{eq:intgate} (named afater Hickson's concurrent training effect) encodes the classic endurance–strength interference effect \cite{Hickson1980}. 
	
	The
	\emph{sleep efficiency} \eqref{eq:sleepeff} implements “late training hurts tonight’s recovery” via \(B(t)\)
	\cite{Fullagar2015}. Sleep has both a \emph{quantity} (opportunity \(s(t)\)) and \emph{quality} dimension; late, intense sessions degrade
	the latter \cite{Fullagar2015}. A short exponentially weighted memory captures “how late was the hard work”
	without a full circadian submodel. 
	
	We also use standard diminishing–returns saturations for the two other capacities.
	
	\subsection{The six ODEs (center of the figure)}
	\label{sec:six-odes}
	
	We now present the forced system \(\dot{\mathbf{z}}=f(\mathbf{z},\mathbf{u})\) with
	\(\mathbf{z}(t)=[A,N,F_a,F_c,S,I]^\top\) and inputs \(\mathbf{u}(t)=[u_E,u_H,u_S,B,s,n,x]^\top\).
	Time is measured in \emph{days}. Each right–hand side is constructed so that: (i) sign conventions match the wiring diagram in Fig.~\ref{fig:overview-wiring}; (ii) states remain nonnegative (forward‑invariance); (iii) capacities saturate and detraining is first order, consistent with decades of training‑response literature \cite{Banister1975,Busso2003,Mujika2003}; (iv) sleep opportunity and bedtime proximity modulate nightly clearance via the sleep‑efficiency \(q(B)\) \cite{Fullagar2015}; and (v) the endurance–strength interference documented by Hickson is represented parsimoniously \cite{Hickson1980}.
	
	\vspace{0.3em}
	\noindent\textbf{Notation recap and units.} We take \(A,N,S,I,F_a,F_c\ge0\). \(A,N\) are normalized to \([0,1]\) by asymptotes \(K_A,K_N\) (dimensionless). Fatigues \(F_a,F_c\) and sleep‑debt \(S\) are dimensionless load states; injury \(I\) is a dimensionless hazard/micro‑damage proxy. Training streams \(u_E,u_H,u_S\) have units “training impulse per day” (e.g., TRIMP‑like for endurance, interval‑load for HIIT, tonnage/contacts for strength); all gains that multiply them carry day\(^{-1}\) per (unit of impulse). Sleep \(s(t)\in\{0,1\}\) (night sleep and naps); \(B(t)\ge0\) is the bedtime‑proximity kernel (Sec.~4.1); \(n(t)\in[0,1]\) is a nutrition/availability proxy; \(x(t)\ge0\) is context stress (travel/heat/psych load).
	
	\paragraph{Top row: trainable capacities.}
	\begin{align}
		\dot A &= 
		\underbrace{k_A\big(\alpha_E u_E+\alpha_H u_H\big)}_{\substack{\text{endurance and HIIT stimuli} \\ \text{(units: day}^{-1}\!)}} 
		\,\underbrace{G_\mathrm{rec}(S,F_c)}_{\text{recovery gate}}\,
		\underbrace{\Big(1-\frac{A}{K_A}\Big)}_{\text{diminishing returns}}
		\;-\;\underbrace{\frac{A}{\tau_A}}_{\text{detraining}}
		\;-\;\underbrace{\theta_{AI}\,I}_{\text{injury penalty}} ,
		\label{eq:A-expanded}\\[0.25em]
		\dot N &= 
		\underbrace{k_N\big(\alpha_S u_S+\alpha_{HN} u_H\big)}_{\substack{\text{strength + HIIT stimuli}\\\text{(day}^{-1}\!)}} 
		\,G_\mathrm{rec}(S,F_c)\,
		\underbrace{G_\mathrm{int}(u_E)}_{\text{concurrent interference}}
		\Big(1-\frac{N}{K_N}\Big)
		\;-\;\frac{N}{\tau_N}
		\;-\;\theta_{NI}\,I .
		\label{eq:N-expanded}
	\end{align}
	
	\emph{Term‑by‑term rationale.}
	\begin{enumerate}[label=\textbf{C.\arabic*},leftmargin=1.6em]
		\item \emph{Stimulus terms.} Endurance \(u_E\) and HIIT \(u_H\) are the primary drivers of central/aerobic adaptation \(A\); strength \(u_S\) (and, secondarily, HIIT) drive neuromuscular adaptation \(N\). The linear dependence in the low‑to‑moderate range reflects the impulse‑response tradition \cite{Banister1975,Busso2003}; higher‑order effects are carried by the saturation \((1-A/K_A)\) and \((1-N/K_N)\).
		\item \emph{Diminishing returns.} The logistic–like factor ensures that identical stimuli produce smaller increments when close to capacity—empirically consistent with taper/peaking observations \cite{Mujika2003}.
		\item \emph{Recovery gate \(G_\mathrm{rec}(S,F_c)=1/(1+c_S S+c_c F_c)\).} Poor sleep state \(S\) and unresolved central fatigue \(F_c\) throttle supercompensation; the multiplicative gate makes this explicit and keeps units tidy (dimensionless factor \( \in (0,1]\)) \cite{Busso2003,Fullagar2015}.
		\item \emph{Detraining.} In the absence of stimulus, \(A\) and \(N\) decay exponentially with time constants \(\tau_A,\tau_N\) (20–90 days typical), matching longitudinal observations \cite{Busso2003}.
		\item \emph{Injury penalty.} A high hazard state \(I\) reduces \emph{realized} gains (pain/inflammation limiting quality and protective down‑regulation) \cite{Gabbett2016}. We model this as a linear sink; in Sec.~5 we will check sensitivity to \(\theta_{AI},\theta_{NI}\).
		\item \emph{Interference gate \(G_\mathrm{int}(u_E)=1/(1+\mu_E u_E)\).} The classic endurance‑strength interference is represented as a simple monotone damping of strength gains when concurrent endurance load is high; \(\mu_E\) sets how much endurance “soaks up” resources otherwise available to \(N\) \cite{Hickson1980}. 
	\end{enumerate}
	
	\emph{Sanity checks and limits.}
	(i) If \(u_E=u_H=0\), \(A\) detains to baseline with time constant \(\tau_A\); similarly for \(N\) if \(u_S=u_H=0\).  
	(ii) If sleep/recovery are excellent (\(S=F_c=0\)), the gate is \(1\) and the model reduces to a classic stimulus–response with saturation.  
	(iii) If \(I\) is large, both capacities stall—this is intentional and encourages recovery regimes in Sec.~6.
	
	\paragraph{Middle row: fatigue on two time scales.}
	\begin{align}
		\dot F_a &= 
		\underbrace{\gamma_E u_E+\gamma_H u_H+\gamma_S u_S}_{\text{training\,$\to$\,acute load}}
		\;-\;\underbrace{\big(\tfrac{1}{\tau_{fa}}+\rho_a\,s(t)\,q(B)\big)F_a}_{\substack{\text{baseline + sleep‑accelerated}\\\text{clearance (reduced by late sessions)}}},
		\label{eq:Fa-expanded}\\[0.25em]
		\dot F_c &= 
		\underbrace{\varepsilon\,F_a}_{\text{unresolved fatigue spills over}}
		\;+\;\underbrace{\xi_c\,x(t)}_{\text{context stress}}
		\;-\;\underbrace{\big(\tfrac{1}{\tau_{fc}}+\rho_c\,s(t)\,q(B)\big)F_c}_{\text{slow clearance, helped by sleep}} .
		\label{eq:Fc-expanded}
	\end{align}
	
	\emph{Design choices and evidence.}
	\begin{itemize}[leftmargin=1.6em]
		\item \emph{Additive inputs, multiplicative dissipation.} All training modalities raise acute fatigue \(F_a\). Both \(F_a\) and \(F_c\) clear with baseline first‑order dissipation (\(\tau_{fa},\tau_{fc}\)) \emph{plus} a sleep‑accelerated component, scaled by \(s(t)\,q(B)\). Making clearance proportional to the current level (\(\propto F_a,F_c\)) guarantees nonnegativity: if a state hits zero, it cannot go negative (forward‑invariance).
		\item \emph{Late‑session penalty.} Sleep efficiency \(q(B)=q_0/(1+\eta B)\) is reduced when late training makes \(B\) large; hence, clearance at night is smaller after evening intensity \cite{Fullagar2015}.
		\item \emph{From acute to chronic.} The spillover parameter \(\varepsilon\) is the “low‑pass filter” linking day‑to‑day load to weeks‑scale central fatigue; it reproduces the fast/slow components in impulse‑response models \cite{Busso2003}.
		\item \emph{Context stress \(x(t)\).} Travel, heat, exams and logistics add central load and degrade sleep \cite{Buchheit2014}; here \(x(t)\) feeds \(F_c\) via \(\xi_c\), and will also enter \(S\) below.
	\end{itemize}
	
	\emph{Sanity checks and limits.}
	(i) If no sleep is taken (\(s=0\)) both fatigues still clear slowly (via \(1/\tau\)), preventing blow‑up.  
	(ii) If an athlete sleeps well and early (so \(s=1\) and \(B\approx0\)), night clearance is maximal \(\approx(1/\tau+\rho)\,F\).  
	(iii) If there is no training nor stress, \(F_a,F_c\to0\) exponentially.
	
	\paragraph{Bottom row: sleep debt/quality and micro‑damage (risk).}
	\begin{align}
		\dot S &= 
		\underbrace{\lambda_w\big(1-s(t)\big)}_{\text{wake accrual}}
		\;+\;\underbrace{\lambda_T\big(\gamma_E u_E+\gamma_H u_H+\gamma_S u_S\big)}_{\text{training raises need for sleep}}
		\;+\;\underbrace{\xi_s\,x(t)}_{\text{stress impairs sleep}}
		\;-\;\underbrace{\big(\tfrac{1}{\tau_S}+\mu_s\,s(t)\,q(B)\big)S}_{\text{baseline + sleep‑accelerated paydown}},
		\label{eq:S-expanded}\\[0.25em]
		\dot I &= 
		\underbrace{\psi_0\big(\kappa_E u_E+\kappa_H u_H+\kappa_S u_S\big)}_{\text{mechanical/metabolic stress}}
		\underbrace{\big(1+\chi_a F_a+\chi_c F_c\big)}_{\text{fatigue amplifies damage}}
		\;-\;\underbrace{\big(\tfrac{1}{\tau_I}+\psi_s\,s(t)\,q(B)+\psi_n\,n(t)\big)\,I}_{\text{time, sleep, nutrition repair}} .
		\label{eq:I-expanded}
	\end{align}
	
	\emph{Design choices and evidence.}
	\begin{itemize}[leftmargin=1.6em]
		\item \emph{Sleep debt \(S\).} While awake, debt accumulates at \(\lambda_w\); hard training days add an extra term \(\lambda_T(\ldots)\) (arousal/thermoregulation burden). During sleep, debt is paid down at a rate that is larger when sleep is available (\(s=1\)) and efficient (\(q(B)\) close to \(q_0\)). Late sessions (\(B\!\uparrow\)) reduce the nightly paydown \cite{Fullagar2015}.
		\item \emph{Micro‑damage \(I\).} Tissue stress rises with modality‑weighted load \(\kappa_E,\kappa_H,\kappa_S\); fatigue multiplies the effect (\(\chi_a,\chi_c\)) because poor mechanics and slower remodeling under fatigue are well documented \cite{Gabbett2016}. Repair is first‑order with time and is accelerated by sleep (via \(q(B)\)) and by nutrition \(n(t)\) (energy/protein availability).
		\item \emph{Nutrition as clearance.} Modeling \(n(t)\) as an additive clearance coefficient \(\psi_n n(t)\) keeps dimensions clean and matches the idea that adequate fueling speeds remodeling rather than instantaneously erasing damage.
	\end{itemize}
	
	\emph{Sanity checks and limits.}
	(i) If an athlete extends sleep opportunity (larger \(s\)) and trains earlier (smaller \(B\)), the term \((\mu_s s q(B))S\) pulls \(S\) down faster.  
	(ii) If nutrition is chronically low (\(n\approx0\)), \(I\) clears slowly, raising injury risk for the same external load.
	
	\paragraph{Readiness aggregator (right column).}
	We combine states into a sport‑specific readiness signal
	\begin{equation}
		P(t)\;=\;w_{\mathrm{end}}A+w_{\mathrm{str}}N\;-\;\lambda_a F_a-\lambda_c F_c-\lambda_s S-\lambda_i I,
		\label{eq:P-expanded}
	\end{equation}
	where weights encode the event profile (e.g., \(w_{\mathrm{end}}\gg w_{\mathrm{str}}\) for a marathon). The sign structure matches Fig.~\ref{fig:overview-wiring}. In Sec.~5 we will calibrate \(\lambda_\cdot,w_\cdot\) so that \(P(t)\) predicts best‑effort tests; in Sec.~6 we will use time‑averaged \(P\) and its peak timing to rank regimes and to design tapers \cite{Mujika2003}.
	
	\subsubsection*{Why these forms? (expanded justification)}
	\paragraph{(J1) Positivity and boundedness without ad hoc clipping.}
	All sinks in \eqref{eq:Fa-expanded}–\eqref{eq:I-expanded} are proportional to the current state, so the nonnegative orthant is forward‑invariant: if a state hits zero, it cannot cross negative. For capacities, the saturations \((1-A/K_A),(1-N/K_N)\) and linear detraining keep \(A\in[0,K_A]\), \(N\in[0,K_N]\) if initialized inside the interval—no artificial “max/min” operators are needed.
	
	\paragraph{(J2) Clear separation between \emph{stimulus}, \emph{modulation}, and \emph{dissipation}.}
	Each equation decomposes as: \(\text{gain from inputs}\times\text{gates/saturations}\;-\;\text{(baseline + sleep + nutrition) clearance}\).
	This mirrors the impulse–response view \cite{Banister1975,Busso2003}, isolates where sleep acts (always in the clearance terms through \(q(B)\)), and makes the late–session penalty precise.
	
	\paragraph{(J3) Bedtime proximity as a short memory instead of a full circadian submodel.}
	The kernel \(B(t)=\int (\beta_E u_E+\beta_H u_H+\beta_S u_S)\,K_b(t-\tau)\,d\tau\) (Sec.~4.1) captures “how late was the hard work?” while keeping the model small. The concavity \(q(B)=q_0/(1+\eta B)\) ensures diminishing sleep impairment as \(B\) increases (a large late session is worse than two modest sessions).
	
	\paragraph{(J4) Interference via a \emph{gate} rather than via cross‑terms in \(\dot N\).}
	We use \(G_\mathrm{int}(u_E)=1/(1+\mu_E u_E)\) to encode the endurance–strength interference \cite{Hickson1980}. This makes the trade‑off explicit, dimensionless, and easy to calibrate from blocks that vary the amount of concurrent endurance.
	
	\paragraph{(J5) Minimal parameter collinearity.}
	Stimulus gains \(k_A,k_N\) are separated from modality weights \(\alpha_{\cdot}\), the recovery gate uses distinct couplings \(c_S,c_c\), and sleep efficiency \(q(B)\) has its own baseline/penalty \((q_0,\eta)\). This reduces confounding in calibration (Sec.~5).
	
	\subsubsection*{Dimensional consistency and quick reference}
	Table~\ref{tab:dimensions} shows the units of the main multipliers so each product in \eqref{eq:A-expanded}–\eqref{eq:I-expanded} is in day\(^{-1}\) times a state (as required for an ODE in days).
	\begin{center}
		\begin{tabular}{lll}
			\hline
			Term & Role & Units \\
			\hline
			$k_A\,\alpha_E u_E$,\; $k_A\,\alpha_H u_H$;\; $k_N\,\alpha_S u_S$,\; $k_N\,\alpha_{HN} u_H$ & stimulus gains & day$^{-1}$ \\
			$1/\tau_A,1/\tau_N,1/\tau_{fa},1/\tau_{fc},1/\tau_S,1/\tau_I$ & baseline decay & day$^{-1}$ \\
			$\rho_a s q(B),\rho_c s q(B),\mu_s s q(B),\psi_s s q(B),\psi_n n$ & sleep/nutrition clearance & day$^{-1}$ \\
			$\gamma_E u_E,\gamma_H u_H,\gamma_S u_S$ & training $\to$ fatigue & day$^{-1}$ \\
			$\varepsilon$ & acute $\to$ chronic fatigue spillover & day$^{-1}$ \\
			$\psi_0(\kappa_E u_E+\kappa_H u_H+\kappa_S u_S)$ & load $\to$ damage & day$^{-1}$ \\
			$c_S S+c_c F_c$ & recovery gate (dimensionless) & -- \\
			$\mu_E u_E$ & interference gate (dimensionless) & -- \\
			\hline
		\end{tabular}
	\end{center}
	\label{tab:dimensions}
	
	\subsubsection*{Limiting scenarios (for intuition and testing)}
	\begin{itemize}[leftmargin=1.6em]
		\item \textbf{No training week} ($u_\cdot\!=\!0$), \textbf{early sleep} ($B\!=\!0$): $F_a,F_c,S,I\!\downarrow$; $A,N$ detrain slowly. Useful for validating baseline time constants \(\tau_\cdot\).
		\item \textbf{Late‑evening HIIT block}: $u_H$ near bedtime $\Rightarrow B\!\uparrow$; \(\,q(B)\!\downarrow\) reduces nightly clearance in \eqref{eq:Fa-expanded}–\eqref{eq:I-expanded}, raising next‑day \(F_a,F_c,S,I\) and depressing capacity gains through the recovery gate in \eqref{eq:A-expanded}–\eqref{eq:N-expanded} \cite{Fullagar2015}.
		\item \textbf{Midday polarized block}: high \(u_E\) and small \(u_H\) at midday keep \(B\) small; \(A\!\uparrow\) with limited interference on \(N\), predicting higher average \(P\).
	\end{itemize}
	
	\subsubsection*{Identifiability notes for calibration (pointer to Sec.~5)}
	Estimate \(\tau_{fa},\tau_{fc},\tau_S\) from decay after rest days; \(\eta\) by contrasting nights after late vs.\ midday sessions; \(\mu_E\) from weeks that vary concurrent endurance; \(\chi_a,\chi_c\) from how \(I\) escalates under monotony (elevated \(F_c\)) \cite{Buchheit2014}. We keep gates simple (rational functions) precisely to make this calibration feasible.
		
	\subsection{Well‑posedness and qualitative properties}
	The right–hand side of our ODE system is locally Lipschitz in the states for bounded inputs, so
	solutions exist and are unique. For \(A,N\), the saturations and linear decay give \(0\le A\le K_A\), \(0\le N\le K_N\) when initialized in range. For \(F_a,F_c,S,I\), the linear dissipation terms imply the nonnegative orthant is
	positively invariant and each state is ultimately bounded by an affine function of the input magnitudes
	(\(\|u\|_\text{day},x\)). Under stationary, \(T\)-periodic, or weekly repeating inputs we can study equilibria or the Poincaré map.
		
	\subsection{From regimes to inputs (how schedules drive the ODE)}
	Each regime from the problem statement becomes a specification of the input streams:
	timestamps and magnitudes for $(u_E,u_H,u_S)$, the sleep window $s(t)$ (and naps), and the
	lights--out time used to compute $B(t)$. Nutrition $n(t)$ and context stress $x(t)$ can be held at
	nominal profiles or varied to match scenarios (travel, heat). Representative codings:
	\begin{itemize}
		\item \textbf{Early AM HIIT:} a short $u_H$ pulse soon after wake; standard $s(t)$; small $B(t)$.
		\item \textbf{Evening intensity:} $u_E$ or $u_H$ pulse that ends near lights--out; large $B(t)$ reduces $q(B)$ that night.
		\item \textbf{Split day (AM endurance, PM strength):} two pulses with the PM $u_S$ increasing $B(t)$; useful to probe
		the $G_\mathrm{int}$ effect on $N$.
		\item \textbf{Alternating hard/easy days:} large pulses on hard days, minimal load and optional nap ($s=1$ block) on easy days.
		\item \textbf{Midday training:} pulses centered 6--8\,h before bedtime to keep $B(t)$ small.
		\item \textbf{Taper:} multiply volumes by a decaying factor while retaining brief $u_H$ to preserve $N$/$A$
		(``maintain intensity, reduce volume'' \cite{Mujika2003}); schedule earlier in the day to minimize $B(t)$.
		\item \textbf{Sleep extension/nap policy:} lengthen nightly $s(t)$ and add a short post‑lunch nap; ensure late‑day
		sessions are moved earlier (low $B$). Expect lower $S$, faster $F_a/F_c$ clearance, and lower $I$ \cite{Fullagar2015}.
	\end{itemize}

	\section{Model summary and parameter inventory}
	\label{sec:final-model}
	
	This section restates the full model in a compact form and inventories every state, parameter, and input used
	in the six–equation ODE. \emph{For readability we suppress the explicit time dependence}: all symbols
	$A,N,F_a,F_c,S,I,u_E,u_H,u_S,s,B,n,x$ below should be read as functions of time $t$ unless otherwise stated.
	
	\subsection{Compact statement of the six ODEs}
	\label{sec:compact-odes}
	\begingroup\small
	\begin{align}
		\dot A &= k_A(\alpha_E u_E+\alpha_H u_H)\,G_\mathrm{rec}(S,F_c)\Bigl(1-\tfrac{A}{K_A}\Bigr)
		\;-\;\tfrac{A}{\tau_A}\;-\;\theta_{AI} I, \tag{$\mathcal{A}$} \label{eq:compactA}\\
		\dot N &= k_N(\alpha_S u_S+\alpha_{HN} u_H)\,G_\mathrm{rec}(S,F_c)\,G_\mathrm{int}(u_E)\Bigl(1-\tfrac{N}{K_N}\Bigr)
		\;-\;\tfrac{N}{\tau_N}\;-\;\theta_{NI} I, \tag{$\mathcal{N}$} \label{eq:compactN}\\
		\dot F_a &= \gamma_E u_E+\gamma_H u_H+\gamma_S u_S
		\;-\;\Bigl(\tfrac{1}{\tau_{fa}}+\rho_a\, s\,q(B)\Bigr)F_a, \tag{$\mathcal{F}_a$} \label{eq:compactFa}\\
		\dot F_c &= \varepsilon F_a + \xi_c\,x
		\;-\;\Bigl(\tfrac{1}{\tau_{fc}}+\rho_c\, s\,q(B)\Bigr)F_c, \tag{$\mathcal{F}_c$} \label{eq:compactFc}\\
		\dot S &= \lambda_w(1-s)+\lambda_T(\gamma_E u_E+\gamma_H u_H+\gamma_S u_S)+\xi_s\,x
		\;-\;\Bigl(\tfrac{1}{\tau_S}+\mu_s\, s\,q(B)\Bigr)S, \tag{$\mathcal{S}$} \label{eq:compactS}\\
		\dot I &= \psi_0(\kappa_E u_E+\kappa_H u_H+\kappa_S u_S)\bigl(1+\chi_a F_a+\chi_c F_c\bigr)
		\;-\;\Bigl(\tfrac{1}{\tau_I}+\psi_s\, s\,q(B)+\psi_n\, n\Bigr)I. \tag{$\mathcal{I}$} \label{eq:compactI}
	\end{align}
	\endgroup
	
	\subsection{Inputs and helper relations (definitions)}
	\label{sec:inputs-helpers}
	\begingroup\small
	\begin{align}
		\text{Bedtime kernel: } \quad
		B &= \int_{0}^{h_b}\!\!\bigl(\beta_E u_E(t-\Delta)+\beta_H u_H(t-\Delta)+\beta_S u_S(t-\Delta)\bigr)\,K_b(\Delta)\,d\Delta, 
		\quad K_b(\Delta)=e^{-\Delta/\sigma_b}, \\
		\text{Sleep efficiency: } \quad
		q(B) &= \frac{q_0}{1+\eta B}, 
		\qquad 0<q_0\le 1,\ \eta\ge 0,\\
		\text{Recovery gate: } \quad
		G_\mathrm{rec}(S,F_c) &= \frac{1}{1+c_S S+c_c F_c},\\
		\text{Interference gate: } \quad
		G_\mathrm{int}(u_E) &= \frac{1}{1+\mu_E u_E}.
	\end{align}
	\endgroup
	
	\subsection{Output (readiness)}
	\label{sec:output-P}
	\begin{equation}
		P \;=\; w_{\mathrm{end}}A+w_{\mathrm{str}}N-\lambda_a F_a-\lambda_c F_c-\lambda_s S-\lambda_i I.
	\end{equation}
	
	\subsection{Variables and parameters (grouped inventory)}
	\label{sec:inventory}
	To match the style of the reference report’s model summary, Table~\ref{tab:inventory} lists the six equations and all symbols they introduce or require. We include the input/helper relations and the readiness output for completeness. Units follow Section~\ref{sec:model-development}.
	\begin{center}
		\renewcommand{\arraystretch}{1.18}
		\setlength{\tabcolsep}{6pt}
		\begin{longtable}{>{\raggedright\arraybackslash}p{0.20\linewidth} >{\raggedright\arraybackslash}p{0.22\linewidth} >{\raggedright\arraybackslash}p{0.52\linewidth}}
			\caption{Variables and parameters used in the six–equation ODE and associated input/output relations (time dependence suppressed for readability).}
			\label{tab:inventory}\\
			\toprule
			\textbf{Equation} & \textbf{Symbol} & \textbf{Definition / role (units)} \\
			\midrule
			\endfirsthead
			\multicolumn{3}{l}{\small\itshape Table~\thetable\ (continued): Variables and parameters used in the model.}\\
			\toprule
			\textbf{Equation} & \textbf{Symbol} & \textbf{Definition / role (units)} \\
			\midrule
			\endhead
			\midrule
			\multicolumn{3}{r}{\small\itshape Continued on next page} \\
			\bottomrule
			\endfoot
			\bottomrule
			\endlastfoot
			
			\textbf{($\mathcal{A}$) Aerobic adaptation} & $A$ & Aerobic/endurance capacity (normalized $0\!-\!1$).\\
			& $k_A$ & Adaptation gain (day$^{-1}$ per unit of stimulus).\\
			& $\alpha_E,\alpha_H$ & Weights from endurance/HIIT stimulus to $A$ (dimensionless).\\
			& $G_\mathrm{rec}(S,F_c)$ & Recovery gate $=1/(1+c_S S+c_c F_c)$ (dimensionless).\\
			& $c_S,c_c$ & Couplings of sleep debt and chronic fatigue in recovery gate (dimensionless).\\
			& $K_A$ & Aerobic asymptote (often normalized to $1$).\\
			& $\tau_A$ & Detraining time constant of $A$ (days).\\
			& $\theta_{AI}$ & Injury penalty on realized gains (day$^{-1}$).\\
			\addlinespace[1mm]
			
			\textbf{($\mathcal{N}$) Neuromuscular adaptation} & $N$ & Strength/power capacity (normalized $0\!-\!1$).\\
			& $k_N$ & Adaptation gain (day$^{-1}$ per unit of stimulus).\\
			& $\alpha_S,\alpha_{HN}$ & Weights from strength/HIIT stimulus to $N$ (dimensionless).\\
			& $G_\mathrm{rec}(S,F_c)$ & Recovery gate as above.\\
			& $G_\mathrm{int}(u_E)$ & Interference gate $=1/(1+\mu_E u_E)$ (dimensionless).\\
			& $\mu_E$ & Strength of endurance–strength interference (dimensionless).\\
			& $K_N$ & Neuromuscular asymptote (often normalized to $1$).\\
			& $\tau_N$ & Detraining time constant of $N$ (days).\\
			& $\theta_{NI}$ & Injury penalty on realized gains (day$^{-1}$).\\
			\addlinespace[1mm]
			
			\textbf{($\mathcal{F}_a$) Acute fatigue} & $F_a$ & Acute/session fatigue (dimensionless).\\
			& $\gamma_E,\gamma_H,\gamma_S$ & Modality gains: training $\to F_a$ (day$^{-1}$ per input unit).\\
			& $\tau_{fa}$ & Baseline acute fatigue clearance (days).\\
			& $\rho_a$ & Sleep–accelerated clearance coefficient for $F_a$ (day$^{-1}$).\\
			& $s$ & Sleep opportunity indicator (1 during sleep; 0 otherwise).\\
			& $q(B)$ & Sleep efficiency, see “helpers” (dimensionless).\\
			\addlinespace[1mm]
			
			\textbf{($\mathcal{F}_c$) Chronic fatigue} & $F_c$ & Chronic/central fatigue (dimensionless).\\
			& $\varepsilon$ & Spillover from $F_a$ to $F_c$ (day$^{-1}$).\\
			& $\xi_c$ & Context stress $\to F_c$ gain (day$^{-1}$ per unit $x$).\\
			& $\tau_{fc}$ & Baseline chronic fatigue clearance (days).\\
			& $\rho_c$ & Sleep–accelerated clearance coefficient for $F_c$ (day$^{-1}$).\\
			& $x$ & Context stress input (travel/heat/psych load; exogenous).\\
			\addlinespace[1mm]
			
			\textbf{($\mathcal{S}$) Sleep debt / quality} & $S$ & Sleep debt / sleep quality state (larger $S$ is worse).\\
			& $\lambda_w$ & Wake accrual rate of sleep debt (debt units per day).\\
			& $\lambda_T$ & Training–day accrual scaling (debt per unit of training load).\\
			& $\xi_s$ & Context stress $\to S$ gain (per unit $x$).\\
			& $\tau_S$ & Baseline sleep debt decay (days).\\
			& $\mu_s$ & Sleep–accelerated paydown for $S$ (day$^{-1}$).\\
			& $s,\,q(B)$ & Sleep opportunity and efficiency as above.\\
			\addlinespace[1mm]
			
			\textbf{($\mathcal{I}$) Injury micro–damage} & $I$ & Micro–damage / injury hazard proxy (dimensionless).\\
			& $\psi_0$ & Base damage gain (day$^{-1}$ per input unit).\\
			& $\kappa_E,\kappa_H,\kappa_S$ & Modality weights in load $\to I$ (dimensionless).\\
			& $\chi_a,\chi_c$ & Fatigue amplification of damage (dimensionless).\\
			& $\tau_I$ & Baseline repair/remodeling time constant (days).\\
			& $\psi_s$ & Sleep–accelerated repair gain (day$^{-1}$).\\
			& $\psi_n$ & Nutrition–accelerated repair gain (day$^{-1}$).\\
			& $n$ & Nutrition/availability input (0–1).\\
			\addlinespace[1mm]
			
			\textbf{Inputs and helper relations} & $u_E,u_H,u_S$ & Training composition (endurance/HIIT/strength); exogenous, piecewise‑continuous.\\
			& $B$ & Bedtime kernel (recent late‑day training), $B=\int_0^{h_b}(\beta_E u_E+\beta_H u_H+\beta_S u_S)K_b(\Delta)d\Delta$.\\
			& $\beta_E,\beta_H,\beta_S$ & Modality weights in the bedtime kernel (dimensionless).\\
			& $K_b(\Delta)$ & Memory kernel $e^{-\Delta/\sigma_b}$ on $\Delta\in(0,h_b]$ (decay $\sigma_b$, horizon $h_b$).\\
			& $q(B)$ & Sleep efficiency $=q_0/(1+\eta B)$; $0<q_0\le1$, $\eta\ge0$.\\
			& $G_\mathrm{rec}(S,F_c)$ & Recovery gate $=1/(1+c_S S+c_c F_c)$.\\
			& $G_\mathrm{int}(u_E)$ & Interference gate $=1/(1+\mu_E u_E)$.\\
			& $s$ & Sleep opportunity schedule (0/1; nights and naps).\\
			& $x$ & Context stress (travel/heat/psychological load).\\
			& $n$ & Nutrition/availability proxy (0–1).\\
			\addlinespace[1mm]
			
			\textbf{Output (readiness)} & $P$ & Readiness $=w_{\mathrm{end}}A+w_{\mathrm{str}}N-\lambda_a F_a-\lambda_c F_c-\lambda_s S-\lambda_i I$.\\
			& $w_{\mathrm{end}},w_{\mathrm{str}}$ & Positive weights for endurance vs.\ strength contributions.\\
			& $\lambda_a,\lambda_c,\lambda_s,\lambda_i$ & Penalty weights on fatigue, debt, and damage.\\
		\end{longtable}
	\end{center}
	
	\paragraph{Notes on usage.}
	The **inputs** $u_E,u_H,u_S,s,n,x$ are set by the chosen regime (Section~\ref{sec:model-overview}); the
	**helpers** $B,q,G_\mathrm{rec},G_\mathrm{int}$ are computed from them and from the states; and the six ODEs
	\eqref{eq:compactA}–\eqref{eq:compactI} evolve the **states**. The right‑hand side is locally Lipschitz for bounded inputs and keeps states nonnegative by construction.
	
	\section{Parameter estimation and prior specification}
	\label{sec:estimation}
	
	In this section we specify \emph{priors} for every parameter that appears in the six ODEs
	(\S\ref{sec:six-odes}) and in the auxiliary relations (bedtime kernel, gates, sleep efficiency, readiness).
	Our intent is twofold: (i) make explicit the physiological knowledge and modeling compromises behind
	each numerical choice; (ii) provide weakly‑to‑moderately informative priors that regularize calibration
	without constraining legitimate athlete‑to‑athlete variation.
	
	Throughout this section we omit the explicit time argument (e.g., we write $A$ rather than $A(t)$) for
	readability; all states and inputs are functions of time unless stated otherwise.
	
	\subsection{Data sources and normalization}
	\label{sec:data-normalization}
	We calibrate against standard practitioner data streams:
	(i) session composition $\{u_E,u_H,u_S\}$ from training logs (zone minutes/TRIMP, interval prescriptions, tonnage or contact counts),
	(ii) sleep opportunity $s$ and bedtime (from diaries/wearables) to compute $B$,
	(iii) nutrition $n$ (binary or graded availability; e.g., carbohydrate/protein targeting),
	(iv) context stress $x$ (travel, heat, self‑report),
	(v) performance proxies for $A,N$ (best‑effort curves / critical‑power metrics \cite{Skiba2012}, CMJ/1RM or sprint splits),
	(vi) recovery markers for $F_a,F_c,S$ (sRPE/monotony, HRV and resting HR trends \cite{Buchheit2014}, actigraphy‑derived sleep efficiency),
	(vii) tissue status for $I$ (soreness, acute:chronic load patterns \cite{Gabbett2016}).
	
	To reduce unit confounds we standardize the daily training inputs so that a “typical hard day” produces
	$\int(u_E+u_H+u_S)\,dt\approx 1$. Priors below are expressed on that scale; when a practitioner chooses
	a different scale, the gains adapt during calibration.
	
	\subsection{Literature anchors used for priors}
	We align with the training‑response canon (fast/slow components, diminishing returns, first‑order detraining) \cite{Banister1975,Busso2003},
	the taper literature \cite{Mujika2003}, the endurance–strength interference evidence \cite{Hickson1980},
	sleep–performance links (sleep quantity \emph{and} efficiency) \cite{Fullagar2015,Buchheit2014},
	and load–injury associations \cite{Gabbett2016}. We purposefully encode these ideas as simple,
	smooth gates (rational/Hill‑like) to keep the system identifiable (\S\ref{sec:identifiability-plan}).
	
	\subsection{Priors for the six ODEs}
	\label{sec:priors-odes}
	We denote $\mathrm{LN}(\mu,\sigma^2)$ a log‑normal with $\log$‑space mean $\mu$ and s.d.\ $\sigma$;
	$\mathrm{Beta}(a,b)$ on $[0,1]$; $\mathrm{Dir}(\boldsymbol{\alpha})$ a Dirichlet; $\mathcal{N}^+(m,s^2)$ a half‑normal.
	
	\paragraph{Top row — capacities $A,N$ (\eqref{eq:S1}–\eqref{eq:S2}).}
	\begin{itemize}[leftmargin=1.5em]
		\item $k_A,k_N$ (stimulus gains, day$^{-1}$ per load): $\mathrm{LN}(\log 0.03,\,0.5^2)$.
		\emph{Rationale:} week‑to‑month scale improvements of 5–20\% under sustained load are typical in
		Banister‑style models \cite{Banister1975,Busso2003}.
		\item $(\alpha_E,\alpha_H)$ and $(\alpha_S,\alpha_{HN})$ (modality weights): Dirichlet priors on each pair, 
		$\mathrm{Dir}(2,2)$ (uninformative symmetric) with the \emph{soft} expectation that $u_E,u_H$ dominate $A$ and
		$u_S,u_H$ dominate $N$ (\cite{Hickson1980} supports HIIT aiding both).
		\item $K_A,K_N$ (asymptotes): fix to $1$ by normalization; if left free, use $\mathrm{LN}(\log 1,0.1^2)$.
		\item $\tau_A,\tau_N$ (detraining time constants): $\mathrm{LN}(\log 50,0.4^2)$ and $\mathrm{LN}(\log 35,0.4^2)$ days.
		\emph{Rationale:} multi‑week decay after cessation \cite{Busso2003,Mujika2003}.
		\item $\theta_{AI},\theta_{NI}$ (injury penalty on realized gains): $\mathcal{N}^+(0.05,\,0.05^2)$ day$^{-1}$.
		Small, but allows stalls in adaptation when $I$ is high \cite{Gabbett2016}.
		\item $G_\mathrm{int}(u_E)=1/(1+\mu_E u_E)$ (interference): $\mu_E\sim\mathrm{LN}(\log 0.3,\,0.5^2)$.
		\emph{Rationale:} concurrent endurance can depress strength gains by 10–40\% at high endurance
		loads \cite{Hickson1980}.
		\item $G_\mathrm{rec}(S,F_c)=1/(1+c_S S+c_c F_c)$: $c_S,c_c\sim\mathrm{LN}(\log 0.5,\,0.6^2)$ (dimensionless).
		\emph{Rationale:} poor sleep and central fatigue throttle supercompensation \cite{Busso2003,Fullagar2015}.
	\end{itemize}
	
	\paragraph{Middle row — fatigue $F_a,F_c$ (\eqref{eq:S3}–\eqref{eq:S4}).}
	\begin{itemize}[leftmargin=1.5em]
		\item $(\gamma_E,\gamma_H,\gamma_S)$ (training $\to F_a$, day$^{-1}$ per load): independent $\mathrm{LN}(\log m,0.35^2)$ with
		means $m=(0.6,0.9,0.5)$, reflecting that HIIT generally spikes acute load most.
		\item $\tau_{fa}$ (acute clearance): $\mathrm{LN}(\log 1.0,0.35^2)$ days (median 1 day; 95\% $\approx$ 0.5–2 days).
		\item $\tau_{fc}$ (chronic clearance): $\mathrm{LN}(\log 10,0.5^2)$ days (95\% $\approx$ 4–25 days), consistent with
		fast/slow components in training‑response models \cite{Busso2003}.
		\item $\rho_a,\rho_c$ (sleep‑accelerated clearances): $\mathcal{N}^+(0.5,\,0.25^2)$ day$^{-1}$.
		Sleep increases clearance rates; magnitude tempered by $q(B)$ \cite{Fullagar2015}.
		\item $\varepsilon$ (spillover $F_a\!\to\!F_c$): $\mathrm{Beta}(2.5,12)$ (mean $\approx 0.17$ day$^{-1}$), capturing that only a fraction of unresolved fatigue becomes chronic \cite{Busso2003}.
		\item $\xi_c$ (context stress $\to F_c$): $\mathcal{N}^+(0.2,0.15^2)$; $\xi_a$ (into $F_a$) $\mathcal{N}^+(0.05,0.05^2)$, often near zero.
		Travel/heat/psych load primarily burdens the slow component \cite{Buchheit2014}.
	\end{itemize}
	
	\paragraph{Bottom row — sleep debt $S$ and micro‑damage $I$ (\eqref{eq:S5}–\eqref{eq:S6}).}
	\begin{itemize}[leftmargin=1.5em]
		\item $\lambda_w$ (wake accrual): $\mathrm{LN}(\log 0.3,0.4^2)$ day$^{-1}$; $\lambda_T$ (training‑day accrual gain): $\mathrm{LN}(\log 0.2,0.5^2)$.
		\emph{Rationale:} debt rises during wakefulness; heavy training increases need \cite{Fullagar2015}.
		\item $\tau_S$ (baseline paydown): $\mathrm{LN}(\log 6,0.4^2)$ days; $\mu_s$ (sleep‑accelerated paydown gain): $\mathcal{N}^+(0.3,0.15^2)$.
		\item $\psi_0$ (base damage gain): $\mathrm{LN}(\log 0.4,0.5^2)$ day$^{-1}$; modality weights
		$(\kappa_E,\kappa_H,\kappa_S)\sim\mathrm{Dir}(2,3,3)$ to reflect higher damage from eccentric/HIIT loads.
		\item $(\chi_a,\chi_c)$ (fatigue amplification of damage): independent $\mathrm{Beta}(2,6)$ (mean $\approx 0.25$), encoding poorer mechanics/remodeling with fatigue \cite{Gabbett2016}.
		\item $\tau_I$ (repair time): $\mathrm{LN}(\log 14,0.5^2)$ days; $\psi_s$ (sleep‑accelerated repair): $\mathcal{N}^+(0.25,0.15^2)$;
		$\psi_n$ (nutrition‑accelerated repair): $\mathcal{N}^+(0.2,0.1^2)$.
	\end{itemize}
	
	\subsection{Priors for auxiliary relations (left/right columns)}
	\label{sec:priors-aux}
	\paragraph{Bedtime kernel and sleep efficiency.}
	We compute \(B=\int_0^{h_b}(\beta_E u_E+\beta_H u_H+\beta_S u_S)\,e^{-\Delta/\sigma_b}\,d\Delta\).
	\begin{itemize}[leftmargin=1.5em]
		\item Horizon $h_b$: $\mathrm{LN}(\log 6\ \mathrm{h},\,0.25^2)$ (typical “final six hours” window matters most for sleep),
		\item Decay $\sigma_b$: $\mathrm{LN}(\log 1.5\ \mathrm{h},\,0.3^2)$,
		\item Modality weights $(\beta_E,\beta_H,\beta_S)\sim\mathrm{Dir}(1.5,2.5,2.0)$ (late‐day HIIT/strength penalized more \cite{Fullagar2015}).
		\item Sleep efficiency \(q(B)=q_0/(1+\eta B)\): $q_0\sim\mathrm{Beta}(18,3)$ (mean~$\approx 0.86$), $\eta\sim\mathrm{LN}(\log 1.0,0.6^2)$ so a large late session (kernel mass $\approx1$) can lower efficiency by 10–30\% \cite{Fullagar2015}.
	\end{itemize}
	
	\paragraph{Readiness aggregation $P=w_{\rm end}A+w_{\rm str}N-\phi_aF_a-\phi_cF_c-\phi_sS-\phi_iI$.}
	\begin{itemize}[leftmargin=1.5em]
		\item Positive weights $(w_{\rm end},w_{\rm str})\sim\mathrm{Dir}(2,2)$ (sport‑specific).
		\item Penalties $(\phi_a,\phi_c,\phi_s,\phi_i)\sim\mathcal{N}^+(0.5,0.3^2)$ with a soft prior ordering $\phi_i\ge\phi_c\ge\phi_a$ (tissue risk penalized most).
	\end{itemize}
	
	\subsection{Initial conditions}
	\label{sec:initials}
	We place weak priors: $A_0,N_0\sim\mathrm{Beta}(4,2)$ (typical preseason $\approx 0.7$ of asymptote),
	$F_{a0},F_{c0},S_0,I_0\sim\mathcal{N}^+(0.2,0.2^2)$. These can be sharpened using one week of baseline
	data before simulations begin.
	
	\subsection{Identifiability and staged calibration plan}
	\label{sec:identifiability-plan}
	We adopt a two‑stage strategy (common in training‑response modeling \cite{Busso2003} and mirrored in the
	reference report’s staged approach). :contentReference[oaicite:2]{index=2}
	\begin{enumerate}[leftmargin=1.5em]
		\item \textbf{Stage A: exogenous and clearance structure.} Fix input normalization; estimate sleep kernel
		$(h_b,\sigma_b,\boldsymbol{\beta})$ from late‑vs‑midday sessions and their effect on actigraphy efficiency; fit
		$\{q_0,\eta\}$ from the same nights. With those fixed, estimate \emph{clearance} parameters 
		$\{\tau_{fa},\rho_a,\tau_{fc},\rho_c,\tau_S,\mu_s,\tau_I,\psi_s,\psi_n\}$ from recovery trajectories during low‑load days.
		\item \textbf{Stage B: adaptation and couplings.} With clearances set, estimate stimulus gains
		$\{k_A,k_N\}$ and modality weights $\{\alpha_\cdot\}$ from blocks emphasizing $u_E/u_H$ (for $A$) and $u_S/u_H$ (for $N$),
		then $\mu_E$ (interference) by comparing strength‐focused blocks with/without concurrent endurance \cite{Hickson1980}.
		Estimate load‑to‑fatigue gains $\{\gamma_\cdot\}$ and spillover $\varepsilon$ from day‑to‑week dynamics under varied microcycles.
		Finally, fit readiness weights $(w_\cdot,\phi_\cdot)$ using best‑effort or competition outcomes as targets \cite{Mujika2003}.
	\end{enumerate}
	We recommend Bayesian calibration (e.g., NUTS) with the priors above and a simple measurement
	model: observed proxies $y_A,y_N$ are noisy maps of $A,N$ (Gaussian noise on a logit transform),
	$y_{F_a},y_{F_c}$ from sRPE/monotony, $y_S$ from sleep debt/efficiency, $y_I$ from soreness and
	acute:chronic ratio features \cite{Gabbett2016}. This separates process noise from measurement noise,
	improving identifiability.
	
	\subsection{Prior summary tables}
	\label{sec:prior-tables}
	\renewcommand{\arraystretch}{1.15}
	\setlength{\LTpre}{6pt}\setlength{\LTpost}{6pt}
	\begin{longtable}{p{0.24\linewidth}p{0.46\linewidth}p{0.24\linewidth}}
		\caption{Prior distributions for parameters (time dependence suppressed; units in days or day$^{-1}$ as indicated).}\\
		\toprule
		\textbf{Block / Parameter} & \textbf{Interpretation} & \textbf{Prior (95\% range)} \\
		\midrule
		\endfirsthead
		\toprule
		\textbf{Block / Parameter} & \textbf{Interpretation} & \textbf{Prior (95\% range)} \\
		\midrule
		\endhead
		\midrule\multicolumn{3}{r}{\emph{Table continues}}\\\midrule
		\endfoot
		\bottomrule
		\endlastfoot
		\textbf{Capacities} $k_A,k_N$ & stimulus $\to$ adaptation gains (day$^{-1}$/load) & $\mathrm{LN}(\log 0.03,0.5^2)$ \\
		$K_A,K_N$ & asymptotes (normalized) & fixed 1 (or $\mathrm{LN}(\log 1,0.1^2)$) \\
		$\alpha_E,\alpha_H$; $\alpha_S,\alpha_{HN}$ & modality weights & $\mathrm{Dir}(2,2)$ (each pair) \\
		$\tau_A,\tau_N$ & detraining time constants (d) & $\mathrm{LN}(\log 50,0.4^2)$; $\mathrm{LN}(\log 35,0.4^2)$ \\
		$\theta_{AI},\theta_{NI}$ & injury penalties on gains & $\mathcal{N}^+(0.05,0.05^2)$ \\
		$c_S,c_c$ & recovery‑gate couplings & $\mathrm{LN}(\log 0.5,0.6^2)$ \\
		$\mu_E$ & interference gate sensitivity & $\mathrm{LN}(\log 0.3,0.5^2)$ \\
		\midrule
		\textbf{Fatigue} $\gamma_E,\gamma_H,\gamma_S$ & load $\to F_a$ gains (day$^{-1}$/load) & $\mathrm{LN}(\log[0.6,0.9,0.5],0.35^2)$ \\
		$\tau_{fa},\tau_{fc}$ & clearances (d) & $\mathrm{LN}(\log 1,0.35^2)$; $\mathrm{LN}(\log 10,0.5^2)$ \\
		$\rho_a,\rho_c$ & sleep‑accelerated clearances & $\mathcal{N}^+(0.5,0.25^2)$ \\
		$\varepsilon$ & $F_a\!\to\!F_c$ spillover & $\mathrm{Beta}(2.5,12)$ \\
		$\xi_a,\xi_c$ & stress $\to F_a,F_c$ & $\mathcal{N}^+(0.05,0.05^2)$; $\mathcal{N}^+(0.2,0.15^2)$ \\
		\midrule
		\textbf{Sleep \& Damage} $\lambda_w,\lambda_T$ & debt accrual (wake / training) & $\mathrm{LN}(\log 0.3,0.4^2)$; $\mathrm{LN}(\log 0.2,0.5^2)$ \\
		$\tau_S,\mu_s$ & debt paydown (baseline / sleep) & $\mathrm{LN}(\log 6,0.4^2)$; $\mathcal{N}^+(0.3,0.15^2)$ \\
		$\psi_0$ & base damage gain & $\mathrm{LN}(\log 0.4,0.5^2)$ \\
		$(\kappa_E,\kappa_H,\kappa_S)$ & modality weights to damage & $\mathrm{Dir}(2,3,3)$ \\
		$\chi_a,\chi_c$ & fatigue amplification & $\mathrm{Beta}(2,6)$ \\
		$\tau_I,\psi_s,\psi_n$ & repair (baseline / sleep / nutrition) & $\mathrm{LN}(\log 14,0.5^2)$; $\mathcal{N}^+(0.25,0.15^2)$; $\mathcal{N}^+(0.2,0.1^2)$ \\
		\midrule
		\textbf{Kernel \& Sleep eff.} $h_b,\sigma_b$ & bedtime horizon/decay (h) & $\mathrm{LN}(\log 6,0.25^2)$; $\mathrm{LN}(\log 1.5,0.3^2)$ \\
		$(\beta_E,\beta_H,\beta_S)$ & late‑day modality weights & $\mathrm{Dir}(1.5,2.5,2.0)$ \\
		$q_0,\eta$ & baseline eff./penalty in $q(B)$ & $\mathrm{Beta}(18,3)$; $\mathrm{LN}(\log 1.0,0.6^2)$ \\
		\midrule
		\textbf{Readiness} $(w_{\rm end},w_{\rm str})$ & positive weights & $\mathrm{Dir}(2,2)$ \\
		$\phi_a,\phi_c,\phi_s,\phi_i$ & penalties (fatigue/debt/damage) & $\mathcal{N}^+(0.5,0.3^2)$ (soft order $\phi_i\ge\phi_c\ge\phi_a$) \\
	\end{longtable}
	
	\subsection{Sensitivity of priors (what matters most)}
	The prior mass on the \emph{clearance} parameters ($\tau_{fa},\tau_{fc},\tau_S,\tau_I$ and their sleep‑accelerated
	gains) has the strongest qualitative influence on weekly dynamics; the prior on $\mu_E$ meaningfully
	moderates strength gains when endurance volume is high (\cite{Hickson1980}); and the bedtime kernel
	hyperparameters $(h_b,\sigma_b,\boldsymbol{\beta},\eta)$ set how punitive late sessions are for next‑day readiness
	(\cite{Fullagar2015}). We therefore recommend tighter priors (smaller s.d.) on those blocks if the data
	are sparse.
	
	\subsection{What we \emph{did not} include (and why)}
	We purposely avoided circadian submodels and hormone kinetics; their parameters are poorly identified
	from routine team data and would add fragility. Instead, the bedtime kernel $B$ plus $q(B)$ offers a
	transparent, low‑parameter proxy for the same phenomenon, consistent with the minimal, auditable style
	of the reference model. :contentReference[oaicite:3]{index=3}
	
	\subsection{Likelihood specification and observation model}
	\label{sec:likelihood}
	
	This section defines how real (or synthetic) observations are linked to the latent states
	\(\mathbf{z}=[A,N,F_a,F_c,S,I]^\top\) produced by the ODE in \S\ref{sec:six-odes}.  We separate:
	(i) the \emph{forward model} (ODE integration given parameters and inputs), (ii) a small \emph{process noise}
	that captures model mismatch, and (iii) \emph{measurement models} (observation likelihoods) for each data stream.
	We then write the joint likelihood and hyperpriors for the noise terms.
	
	\vspace{0.5ex}
	\paragraph{Time grid, inputs, and helper signals.}
	We integrate the six ODEs with step \(\Delta t=1\) day unless noted otherwise. Training inputs \(u_E,u_H,u_S\)
	are daily aggregates normalized so that a “typical hard day” satisfies
	\(\int (u_E+u_H+u_S)\,dt\approx 1\) (Sec.~\ref{sec:data-normalization}). The bedtime proximity kernel and
	sleep efficiency, needed by the ODE, are re-stated for completeness:
	\[
	B=\!\int_{0}^{h_b}\!\bigl(\beta_E u_E(t-\Delta)+\beta_H u_H(t-\Delta)+\beta_S u_S(t-\Delta)\bigr)
	e^{-\Delta/\sigma_b}\,d\Delta,\qquad
	q(B)=\frac{q_0}{1+\eta B}.
	\]
	When multiple training sessions occur per day, we compute \(B\) using the within‑day timing and then
	average to the day boundary before applying \(q(B)\).
	
	\paragraph{Forward model (deterministic skeleton).}
	Given parameters \(\theta\), inputs \(\mathbf{u}\), and initial state \(\mathbf{z}_0\), the ODE flow \(\Phi_{\Delta t}\) produces
	\(\tilde{\mathbf{z}}_{k+1}=\Phi_{\Delta t}(\mathbf{z}_k,\mathbf{u}_k;\theta)\) on the daily grid \(t_k\).
	
	\paragraph{Process noise (innovation).}
	To tolerate model misspecification we include small, mean‑zero innovations on each state:
	\begin{equation}
		\mathbf{z}_{k+1}=\Phi_{\Delta t}(\mathbf{z}_k,\mathbf{u}_k;\theta)+\boldsymbol{\varepsilon}_k,\qquad
		\boldsymbol{\varepsilon}_k\sim\mathcal{N}\!\bigl(\mathbf{0},\operatorname{diag}(\sigma_A^2,\sigma_N^2,\sigma_{Fa}^2,\sigma_{Fc}^2,\sigma_S^2,\sigma_I^2)\bigr).
		\label{eq:state-innov}
	\end{equation}
	These \(\sigma_\cdot\) are \emph{process} (not measurement) standard deviations; priors appear below.  Setting
	them near zero recovers a purely deterministic state evolution.
	
	\paragraph{Observation calendar.}
	Each channel has its own set of observation times:
	\(\mathcal{T}_A,\mathcal{T}_N,\mathcal{T}_{Fa},\mathcal{T}_{Fc},\mathcal{T}_{S},\mathcal{T}_{I}\).  The model handles irregular sampling and missing values naturally by omitting
	absent terms from the likelihood.
	
	\subsubsection*{Measurement models (per channel)}
	Below, \(y_{\cdot,k}\) denotes an observation at day \(t_k\). We use link functions that match what practitioners actually record, and we make sign conventions explicit so posterior coefficients are interpretable.
	
	\paragraph{Capacity observations \(A,N\) (field tests).}
	Practically, coaches observe performances that are monotone in \(A\) and \(N\): e.g., critical‑power/pace tests or best‑effort segments for \(A\) and CMJ/1RM/sprint splits for \(N\) \cite{Skiba2012,Banister1975,Busso2003}.
	We define test scores scaled to \([0,1]\) (by sport‑specific reference tables) and work on the logit scale:
	\begin{align}
		y_{A,k}^{(\mathrm{logit})} &= \operatorname{logit}(y_{A,k}) \sim \mathcal{N}\!\big(\operatorname{logit}(A_{t_k})+\alpha_{A0},\,\tau_A^{-1}\big),\quad k\in\mathcal{T}_A, \label{eq:likeA}\\
		y_{N,k}^{(\mathrm{logit})} &= \operatorname{logit}(y_{N,k}) \sim \mathcal{N}\!\big(\operatorname{logit}(N_{t_k})+\alpha_{N0},\,\tau_N^{-1}\big),\quad k\in\mathcal{T}_N. \label{eq:likeN}
	\end{align}
	Offsets \(\alpha_{A0},\alpha_{N0}\) absorb mapping biases between tests and latent capacities; precisions \(\tau_A,\tau_N\) are measurement precisions on the logit scale.
	
	\paragraph{Acute fatigue \(F_a\) (session‑level load proxy).}
	We link \(F_a\) to session RPE, sRPE*duration, or standardized acute‑load residuals. Let \(y_{Fa,k}\) be a z‑score of session‑level perceived exertion:
	\begin{equation}
		y_{Fa,k}\sim \mathcal{N}\big(\beta_{a0}+\beta_{a1} F_{a,t_k},\,\tau_{Fa}^{-1}\big),\qquad \beta_{a1}>0,\quad k\in\mathcal{T}_{Fa}.
		\label{eq:likeFa}
	\end{equation}
	The positive slope encodes that higher \(F_a\) corresponds to higher perceived load the same day or next morning.
	
	\paragraph{Chronic fatigue \(F_c\) (autonomic markers).}
	We map to HRV‑based recovery indices or resting‑heart‑rate (RHR) trends, which respond on day‑to‑week scales \cite{Buchheit2014}. Using a z‑score of morning HRV (\(y_{Fc,k}\), where higher is “better”):
	\begin{equation}
		y_{Fc,k}\sim \mathcal{N}\big(\beta_{c0}-\beta_{c1} F_{c,t_k},\,\tau_{Fc}^{-1}\big),\qquad \beta_{c1}>0,\quad k\in\mathcal{T}_{Fc}.
		\label{eq:likeFc}
	\end{equation}
	The negative sign reflects the empirical HRV–fatigue relationship.
	
	\paragraph{Sleep debt \(S\) (actigraphy/diary efficiency).}
	Let \(e_k\in(0,1)\) be nightly sleep efficiency or quality (fraction of time asleep while in bed) from actigraphy \cite{Fullagar2015}.
	We model a Beta likelihood with logit link (“efficiency decreases as \(S\) increases”):
	\begin{align}
		m_k &= \operatorname{logit}^{-1}\big(\delta_0 - \delta_1 S_{t_k}\big),\qquad \delta_1>0, \nonumber\\
		e_k &\sim \mathrm{Beta}\big(\kappa_S m_k,\;\kappa_S (1-m_k)\big),\qquad k\in\mathcal{T}_S. \label{eq:likeS}
	\end{align}
	Here \(\kappa_S\) is a concentration (higher \(\kappa_S\Rightarrow\) lower dispersion). If total sleep minutes are also observed, a second Gaussian term on log‑minutes with slope \(-\delta_1'\, S_{t_k}\) can be added (independent conditional on \(S\)).
	
	\paragraph{Micro‑damage \(I\) (soreness + injury events).}
	We support two channels: (i) soreness VAS (0–10) and (ii) rare injury events. For soreness \(y_{I,k}^{(\mathrm{VAS})}\):
	\begin{equation}
		y_{I,k}^{(\mathrm{VAS})}\sim \mathcal{N}\big(\gamma_{I0}+\gamma_{I1} I_{t_k},\,\tau_{I,\mathrm{VAS}}^{-1}\big),\qquad \gamma_{I1}>0.
		\label{eq:likeI-vas}
	\end{equation}
	For time‑loss injury incidence over day \([t_k,t_{k+1})\) we use a Bernoulli with log‑link intensity increasing in \(I\) \cite{Gabbett2016}:
	\begin{align}
		\lambda_k &= \lambda_0\,\exp\{\gamma_{I} I_{t_k}\}, \qquad p_k \;=\; 1-\exp\{-\lambda_k \Delta t\}, \nonumber\\
		y_{I,k}^{(\mathrm{inj})} &\sim \mathrm{Bernoulli}(p_k),\qquad k\in\mathcal{T}_{I,\mathrm{inj}}.
		\label{eq:likeI-injury}
	\end{align}
	A zero‑inflated Poisson alternative fits aggregated counts over weeks; the Bernoulli formulation leverages the “rare‑event as Poisson” approximation on the daily grid.
	
	\subsubsection*{Noise hyperpriors (measurement and process)}
	We complete the likelihood with weakly informative priors that match the scales used above:
	\begin{align*}
		&\textbf{Measurement precisions:}\quad
		\tau_A,\tau_N,\tau_{Fa},\tau_{Fc},\tau_{I,\mathrm{VAS}}\sim \mathrm{Gamma}(2,\,2)\ \text{(means 1; adjust per dataset)},\\
		&\textbf{Beta concentration:}\quad \kappa_S\sim \mathrm{Gamma}(3,\,0.5)\ \text{(mean 6; moderately concentrated)},\\
		&\textbf{Linear coefficients:}\quad \alpha_{A0},\alpha_{N0},\beta_{a0},\beta_{c0},\gamma_{I0}\sim \mathcal{N}(0,\,1),\\
		&\beta_{a1},\beta_{c1},\delta_1,\gamma_{I1}\sim \mathcal{N}^+(0.5,\,0.3^2),\qquad \delta_0\sim \mathcal{N}(\operatorname{logit}(0.85),\,0.4^2),\\
		&\lambda_0\sim \mathrm{Gamma}(1.5,\,15)\ \text{(rare baseline injuries)},\qquad \gamma_I\sim \mathcal{N}^+(0.6,\,0.3^2),\\
		&\textbf{Process s.d.\ in \eqref{eq:state-innov}:}\quad
		\sigma_A,\sigma_N,\sigma_{Fa},\sigma_{Fc},\sigma_S,\sigma_I\sim \mathcal{N}^+(0.05,\,0.05^2).
	\end{align*}
	Signs are encoded by half‑normals \(\mathcal{N}^+\) when physiology dictates monotonicity (e.g., higher \(F_c\) lowers HRV).
	
	\subsubsection*{Joint likelihood (factorization)}
	Let \(\mathbf{Z}=\{\mathbf{z}_k\}_{k=0}^{K}\) be the latent daily states and \(\mathbf{Y}\) the multichannel observations. Conditional on parameters \(\theta\), inputs, and initial state \(\mathbf{z}_0\), the joint density factors as
	\begin{align}
		p(\mathbf{Y},\mathbf{Z}\mid \theta)
		&= p(\mathbf{z}_0)\,\prod_{k=0}^{K-1}
		\underbrace{\mathcal{N}\!\big(\mathbf{z}_{k+1}\,;\,\Phi_{\Delta t}(\mathbf{z}_k,\mathbf{u}_k;\theta),\,\Sigma_{\mathrm{proc}}\big)}_{\text{process model \eqref{eq:state-innov}}}\nonumber\\
		&\quad\times \prod_{k\in\mathcal{T}_A} \mathcal{N}\!\big(y_{A,k}^{(\mathrm{logit})}\,;\operatorname{logit}(A_{t_k})+\alpha_{A0},\,\tau_A^{-1}\big)\nonumber\\
		&\quad\times \prod_{k\in\mathcal{T}_N} \mathcal{N}\!\big(y_{N,k}^{(\mathrm{logit})}\,;\operatorname{logit}(N_{t_k})+\alpha_{N0},\,\tau_N^{-1}\big)\nonumber\\
		&\quad\times \prod_{k\in\mathcal{T}_{Fa}} \mathcal{N}\!\big(y_{Fa,k}\,;\beta_{a0}+\beta_{a1}F_{a,t_k},\,\tau_{Fa}^{-1}\big)\nonumber\\
		&\quad\times \prod_{k\in\mathcal{T}_{Fc}} \mathcal{N}\!\big(y_{Fc,k}\,;\beta_{c0}-\beta_{c1}F_{c,t_k},\,\tau_{Fc}^{-1}\big)\nonumber\\
		&\quad\times \prod_{k\in\mathcal{T}_{S}} \mathrm{Beta}\!\big(e_k\,;\kappa_S m_k,\kappa_S(1-m_k)\big) \nonumber\\
		&\quad\times \prod_{k\in\mathcal{T}_{I,\mathrm{VAS}}}\mathcal{N}\!\big(y_{I,k}^{(\mathrm{VAS})}\,;\gamma_{I0}+\gamma_{I1}I_{t_k},\,\tau_{I,\mathrm{VAS}}^{-1}\big)\nonumber\\
		&\quad\times \prod_{k\in\mathcal{T}_{I,\mathrm{inj}}}\mathrm{Bernoulli}\!\big(y_{I,k}^{(\mathrm{inj})}\,;p_k\big).
		\label{eq:joint-like}
	\end{align}
	Here \(\Sigma_{\mathrm{proc}}=\operatorname{diag}(\sigma_A^2,\sigma_N^2,\sigma_{Fa}^2,\sigma_{Fc}^2,\sigma_S^2,\sigma_I^2)\), and \(m_k\) and \(p_k\) are defined in \eqref{eq:likeS}–\eqref{eq:likeI-injury}.
	
	\subsubsection*{Remarks on identifiability and transformations}
	\begin{itemize}[leftmargin=1.3em]
		\item \emph{Boundary states:} We work with \(\operatorname{logit}(A),\operatorname{logit}(N)\) in the measurement model to respect \([0,1]\) bounds; the ODE keeps \(A,N\in[0,1]\) by construction.
		\item \emph{Scaling:} If the practitioner uses a different normalization for \(u_\cdot\), the gains \(k_\cdot,\gamma_\cdot,\psi_0\) absorb the scale. This is handled during calibration; the likelihood remains unchanged.
		\item \emph{Multi‑athlete hierarchical option:} For multiple athletes \(i\), add athlete‑specific offsets \(\alpha_{A0}^{(i)},\alpha_{N0}^{(i)}\) and slopes \(\beta_{\cdot1}^{(i)}\) with group‑level priors; this stabilizes individual fits when some channels are sparse.
	\end{itemize}
	
	\subsubsection*{Calibration notes (posterior computation)}
	We recommend HMC/NUTS with an adjoint‑sensitivity ODE solver (for differentiability). Daily integration with a fixed‑step RK4 or Dormand–Prince (tol \(10^{-6}\)) is sufficient. We monitor \(\widehat{R}\) and effective sample size and perform posterior predictive checks (PPCs) for all channels:
	(i) simulate \(\mathbf{Z}\) forward from posterior draws, (ii) simulate \(\mathbf{Y}\) from \eqref{eq:joint-like}, and (iii) compare to held‑out data.
	
	\subsubsection*{Literature anchors for the links}
	The choice of channels and signs follows sports‑science practice: capacity–performance mapping through critical‑power/CMJ tests \cite{Skiba2012,Banister1975,Busso2003}, HRV/RHR as slow‑fatigue proxies \cite{Buchheit2014}, sleep efficiency as a function of late‑day training \cite{Fullagar2015}, and load‑related injury risk increasing with micro‑damage \cite{Gabbett2016}.  The section’s layout mirrors the clear “observation model” style used by the reference report.  \emph{(See its Section 6 for stylistic comparison.)} {\footnotesize\mbox{}} \hfill \emph{Reference‑report style cue:} :contentReference[oaicite:1]{index=1}
	
	
	
	
	
	
	
	
	\begin{thebibliography}{9}\small
		\bibitem{Banister1975} E.\,W. Banister, T.\,W. Calvert, et al., \emph{A systems model of training for athletic performance}, Can.\ J.\ Appl.\ Sport Sci., 1975.
		\bibitem{Busso2003} T. Busso, \emph{Modeling adaptations to training}, Sports Med., 2003.
		\bibitem{Mujika2003} I. Mujika \& S. Padilla, \emph{Scientific bases for precompetition tapering strategies}, Med. Sci. Sports Exerc., 2003.
		\bibitem{Hickson1980} R. Hickson, \emph{Interference of strength development by simultaneously training for strength and endurance}, Eur. J. Appl. Physiol., 1980.
		\bibitem{Fullagar2015} H. Fullagar et al., \emph{Sleep and athletic performance}, Sports Med., 2015.
		\bibitem{Buchheit2014} M. Buchheit, \emph{Monitoring training status with HR measures}, Sports Med., 2014.
		\bibitem{Gabbett2016} T. Gabbett, \emph{The training--injury prevention paradox: load, risk and performance}, Br. J. Sports Med., 2016.
		\bibitem{Skiba2012} P. Skiba et al., \emph{Modeling the expenditure and reconstitution of work capacity above critical power}, Med. Sci. Sports Exerc., 2012.
	\end{thebibliography}
	
	
	
	
\end{document}